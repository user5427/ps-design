%%%%%
%%%%%  Use LUALATEX, not LATEX.
%%%%%
%%%%
\documentclass[]{VUMIFTemplateClass}

\usepackage{indentfirst}
\usepackage{amsmath, amsthm, amssymb, amsfonts}
\usepackage{mathtools}
\usepackage{physics}
\usepackage{graphicx}
\usepackage{verbatim}
\usepackage[hidelinks]{hyperref}
\usepackage{color,algorithm,algorithmic}
\usepackage[nottoc]{tocbibind}
\usepackage{tocloft}

\usepackage{titlesec}
\newcommand{\sectionbreak}{\clearpage}

\makeatletter
\renewcommand{\fnum@algorithm}{\thealgorithm}
\makeatother
\renewcommand\thealgorithm{\arabic{algorithm} algorithm}

\usepackage{biblatex}
\bibliography{bibliografija}
%% to change the numbering (numeric or alphabetic) of bibliographic sources, make the change in VUMIFTemplateClass.cls, line 139

% Author's MACROS
\newcommand{\EE}{\mathbb{E}\,} % Mean
\newcommand{\ee}{{\mathrm e}}  % nice exponent
\newcommand{\RR}{\mathbb{R}}


\studyprogramme{Software engineering} %Write your study programme (example – Software engineering, Financial and Actuarial Mathematics, etc.)
% \worktype{\{Work type\}} % Bachelor's thesis or Master's thesis
\worktitle{Human computer interaction}
\secondworktitle{Žmogaus kompiuterio sąveika}
\workauthor{Tadas Riksas, Darius Spruogis, Gustas Mickus, Kajus Bicka}

%There may be more than one author, in which case each author is written from a new line which is added in Titlepage.tex or LongerTitlePage.tex
%\secondauthor{Name Surname} %If present, otherwise delete

\supervisor{Kristina Lapin}
\reviewer{Kristina Lapin} %If present, otherwise delete
% \scientificadvisor{pedagogical/scientific title Name Surname} %If present, otherwise delete

\begin{document}
\selectlanguage{english}

\onehalfspacing
\begin{titlepage}
\vskip 20pt
\begin{center}
\includegraphics[scale=0.55]{images/MIF.png}
\end{center}

\makeatletter

\vskip 20pt
\centerline{\bf \large \textbf{VILNIUS UNIVERSITY}}
\vskip 10pt
\centerline{\large \textbf{FACULTY OF MATHEMATICS AND INFORMATICS}}
\vskip 10pt
\centerline{\large \textbf{\MakeUppercase{\@studyprogramme \space study programme}}}

\vskip 80pt
\centerline{\Large \@worktype}
\vskip 20pt
\begin{center}
    {\bf \LARGE \@worktitle}
\end{center}
\begin{center}
    {\bf \Large \@secondworktitle}
\end{center}
\vskip 80pt

\centering{\Large \@workauthor}
\@ifundefined{@secondauthor}{}
{
\vskip 10pt
\centering{\Large \@secondauthor}
}
\vskip 20pt

\centering{
    \begin{tabular}{rcp{.7\textwidth}}
        {\Large Supervisor} & {\Large :} & {\Large \@supervisor}\\[10pt]
        \@ifundefined{@scientificadvisor}{}
            {
                {\Large Scientific advisor} & {\Large :} & {\Large \@scientificadvisor}\\[10pt]
            }
        \@ifundefined{@reviewer}{}
            {
                {\Large Reviewer} & {\Large :} & {\Large \@reviewer}\\[10pt]
            }
    \end{tabular}}


\vskip 110pt

\centerline{\large \textbf{Vilnius}}
\centerline{\large \textbf{\the\year{}}}

\makeatother

\newpage
\end{titlepage}
%\newgeometry{top=2cm,bottom=2cm,right=2cm,left=3cm}
\setcounter{page}{2}


\singlespacing
\selectlanguage{english}
% list of figures, delete if not needed
% \listoffigures 

%list of tables, delete if not needed
% \listoftables

%Turinys
\tableofcontents
\onehalfspacing


% %Section for abbreviations
% \sectionnonum{List of abbreviations} %Leave if necessary
% This section is for when abbreviations are used. For example:

% \begin{tabular}{rcp{.7\textwidth}}
%     {u.d.i.r.v.} & {} & {uniformly distributed independent random variables}
% \end{tabular}

\section{Introduction}
\subsection{Project Title}

\textbf{Title:} UniServe

\textbf{Explanation:} "Uni" represents \emph{universal} inidicating the platform's flexibility across a variety of business models 
(reservation based businesses such as massage services or order based businesses such as restaurants). 
"Serve" refers to \emph{service}, emphasizing the platform’s role in providing core functionalities for these types of businesses.

\subsection{Problem statement}


% Decades ago, it was still common to go to restaurants and expect menus to be
% handed out on-site, or to call by phone to register an appointment. However, as
% we move into an age of ever-accelerating communication and information exchange,
% these methods have become stubborn and slow. More customers and businesses recognize a need for
% faster communication and more accurate information exchange.

% Reliance on traditional methods slows down information exchange between
% employees, introduces human error, and creates friction in customer
% interactions. This leads to employee dissatisfaction, increased workload, as
% well as a decrease in operational efficiency.

% Slower communications and unstandardized registrations or menus introduce friction
% in customer interactions, makes customers more reluctant to engage with the
% business and leads to overall decrease in customer satisfaction.


% \textbf{--updated text--}

% Traditional methods like handing out menus and taking orders on-site or booking appointments by phone have become slow and inefficient in the last decade. This is especially noticeable during peak business periods such as holidays, weekends, or promotional events. Small to medium-sized businesses in restaurant and beauty industries often cannot afford specialized web applications due to limited budgets and technical expertise.
% This creates multiple problems: employees face increased workload and inefficient information exchange, while customers increasingly expect online services like remote ordering and appointment booking. Without accessible digital platforms, businesses risk missing potential customers who rely on the internet, leading to reduced customer reach, lower operational efficiency, and decreased customer satisfaction. The lack of automation results in higher recurring costs and makes customers more reluctant to engage with businesses.

% \textbf{--end of updated text--}



Owners of small to medium-sized businesses in the restaurant and beauty
industries often cannot afford the resources to develop and maintain a
specialized web application for managing their business operations and customer
interactions. This is due to many of these businesses having limited budgets and
technical expertise, which makes custom software solutions expensive.

The people affected by this problem include customers, who increasingly expect
online services such as remote ordering or appointment booking. This challenge
is common in both urban and suburban areas where internet use is high and the
contest for the attention and loyalty of customers has become a defining force.
Without an accessible digital platform, businesses risk missing potential
customers who rely on the internet to discover and engage with services,
leading to reduced customer reach. 

Furthermore, business owners also face operational inefficiencies and lower
profits due to the lack of automation, requiring increased employee workload,
leading to increased recurring costs. The problem occurs continuously throughout
the year, but becomes critical during peak business periods such as holidays,
weekends, or promotional events.

\section{Stakeholder analysis}

\subsection{Identifying stakeholders}


% we have identified our primary stakeholders as employees, secondary as customers, indirect as business owners, competitors as null and development and maintance team as sysadmins and tech support. create a table bellow

\begin{table}[h]
  \centering
  \begin{tabular}{|c|c|}
    \hline
    Stakeholder Type    & Stakeholders \\ \hline
    Primary             & Client Business Owners, Employees \\ \hline
    Secondary           & Customers \\ \hline
    Indirect (Tertiary) & App Business Owners \\ \hline
    External/Competitors & Square, Toast POS users \\ \hline
    Technical Support   & Sysadmins, Tech Support \\ \hline
  \end{tabular}
  \caption{Stakeholders Identification}
  \label{tab:stakeholders}
\end{table}

% \textbf{Primary Stakeholders} directly and frequently use the system. Client
% business owners (restaurant/salon owners) operate the system daily. Employees
% process orders and manage appointments through the application.

% \textbf{Secondary Stakeholders} are occasionally involved or interact via
% others. Customers use the system when placing orders or booking appointments.
% Managers oversee operations and may need access to system statistics.

% \textbf{Indirect (Tertiary) Stakeholders} influence purchase decisions and
% benefit from the system. App business owners (our company's owners) profit from
% the system's success. Business managers make decisions about adopting the
% system.

% \textbf{Competitors} represent existing market solutions. Users currently
% employing Square, Toast POS, or similar platforms are potential adopters of our
% system.

% \textbf{Development and Maintenance Team} creates and maintains the system.
% Sysadmins handle infrastructure and servers. Tech support staff resolve user
% issues and provide assistance.

\subsection{Stakeholder Expectations}

% Nielsen‘s principles
% 1. Learnability
% 2. Efficiency of use
% 3. Memorability
% 4. Few and non-
% catastrophic errors
% 5. Satisfaction

\textbf{Primary Stakeholders:}
\begin{itemize}
    \item \textbf{Client Business Owners} want tools to manage their business,
    track sales, watch over staff, manage stock, and get reports to run their
    business better and reduce costs. They expect and error-free system that is
    reliable and efficient.
    \item \textbf{Employees} want an efficient, error-free, system that helps
    them process orders and book appointments with quick learning and
    memorization of the system.
\end{itemize}

\textbf{Secondary Stakeholders:}
\begin{itemize}
    \item \textbf{Customers} expect a effective, reliable booking and ordering
    experience with quick response times, and accurate information and
    satisfying interactions.
\end{itemize}

\textbf{Indirect (Tertiary) Stakeholders:}
\begin{itemize}
    \item \textbf{App Business Owners} want steady income growth, to increase
    market share, and to compete well with other business software companies.
\end{itemize}

\textbf{Competitors:}
\begin{itemize}
    \item \textbf{Square/Toast POS users} want the same features or better ones,
    fair prices, and easy switching from their current systems (quick learning and memorization).
\end{itemize}

\textbf{Development and Maintenance Team:}
\begin{itemize}
    \item \textbf{Sysadmins} expect reliable system architecture, scalable
    infrastructure, and clear deployment procedures.
    \item \textbf{Tech Support} expect comprehensive documentation, effective
    troubleshooting tools, and manageable user issue resolution processes.
\end{itemize}






\section{User needs analysis}


Now that we know what are our primary and secondary stakeholders expectations,
we can take a look into what are their needs in more detail. Specifically into
how their needs will influence the design of our system.

\subsection{Primary Stakeholders Needs}



\subsubsection{Client Business Owners Needs}

The main needs of the client business owners are to have a system where they 
can manage their business operations daily, track sales, manage inventory,
generate reports (taxes, profits, losses, etc.), and hire or fire employees.
This need is driven by the fact that business owners need to have a clear overview of their
business operations to make informed decisions and ensure the success of their
business.



% The main needs of the client business owners are to have a system that is
% \textbf{error-free}, \textbf{reliable} and \textbf{efficient}. Business owners will need to use the system daily to manage their employees,
% track sales, manage inventory, and generate reports (taxes, profits, losses,
% etc.). Therefore the system must be error-free to ensure that critical data like
% report generation is not compromised. The system must also be reliable to ensure
% that business operations are not disrupted and that business owners can trust
% the system to perform as expected. Finally, the system must be efficient to
% ensure that business owners can quickly and easily perform their tasks without
% wasting time or effort for looking for features or navigating through complex
% menus.

\subsubsection{Employees Needs}

The main need of the employees is to have a system that helps them process orders,
book appointments, and manage customer interactions. This need is driven by the
fact that employees need to be able to perform their tasks quickly and accurately
to ensure customer satisfaction and the smooth operation of the business.

% The main needs of the employees are to have a system that is \textbf{easy to
% learn}, \textbf{efficient} and \textbf{error-free}. Employees will need to use
% the system daily to process orders, book appointments, and manage customer
% interactions. Therefore the system must be easy to learn to ensure that
% employees can quickly adapt to the system and become proficient in its use. The
% system must also be efficient to ensure that employees can complete their tasks
% in a timely manner without unnecessary delays. Finally, the system must be
% error-free to ensure that employees can perform their tasks without costly mistakes
% or disruptions.

\subsection{Secondary Stakeholders Needs}

\subsubsection{Customers Needs}

The main needs of the customers are to have a system that allows them to
easily place orders or book appointments, view menus or services, and receive
accurate information about the business. This need is driven by the fact that
customers expect a seamless and convenient experience when interacting with
businesses, and a well-designed system can help meet these expectations.


% The main needs of the customers are to have a system that is \textbf{effective}, \textbf{efficient},
% \textbf{reliable} and \textbf{satisfying}. Customers will need to use the system
% occasionally to place orders or book appointments. Therefore the system must be
% effective to ensure that customers can easily find and use the features they
% need. The system must also be efficient to ensure that customers can complete
% their tasks quickly and without unnecessary delays. The system must be
% reliable to ensure that customers can trust the system to perform as expected
% and that their interactions with the business are not disrupted. Finally, the
% system must be satisfying to ensure that customers have a positive experience
% and are more likely to return to the business in the future.








% reikia sistemizuoti kokioje situacijoje ko reikia, labiau detaliau issiaiskinti stakeholder needs.
% reikia atlikti naudojimo konteksto analize, kaip tos funkcijos isilies i vartotojo esama gyvenima. Mes galvojame apie naudojima ir konteksta vis dar.
% Naudojimo kontekstas - kaip veiklos itakoja technologiju kaita ir technologiju kaita veiklas, po to kalbesime apie pagrindines zmoniu charakteristikas, kas, kokiose veiklose ir problemas kurias norime spresti, kokiame zingsnyje kas ten stringa, kokioje aplinkoje ten vyksta, kokios tech naudojamos ir ka galime pasiulyti.
% Naudojimo kontekstas - kombinacija vartotoju, uzduociu, resursu, tikslu. Fizines, socialines, technines, kulturines, organizicines aplinkos. Jas riekes issaiskinti ir aprasyti kaip dabar veikia tas vartotojas, ka tie pasakojimai turi atspindeti: turi buti aiskus visi aspektai, kas, kokie tisklai, aplinkos, kokiais prietaisais, programomis naudojasi ir t.t.
% PS inzinerijos standartas - naudojimo kontekstas pagrindinis informacijos reikalavimu saltinis.
% Vystimasis yra begalinis ciklas, yra zmones betkokio amziaus, kutluros, jie veikia kazkokiam kontekste (requirements), tame kontekste kyla nauji reikalavimai naujoms technologijosm, naujos technologijos sudaro galimybe keistis veiklos, vel atsiras nauji reikalavimai ir t.t. gaunasi begalinis ratas.
% naujos interaktyvios tech - prisidedame dabartines zinias, sukaupta patirti ir kuriame naujus reikalavimus. Suvokdami kazkokius nepatogumus, koks yra stovis, kokias galimybes mes galime isnaudoti su technologijomis mes pasiulome sprendimus, naujas veiklas.

% People, context, activities, technologies

% ____________________________

% People - kokiomis salygomis veikia, ju fizines galimybes, darbo vieta, veikimo aplinka kur bus naudojama technologija, zmoniu ugis, aukstis, antropronetiniai dalykai? ivestis isvestis, klaviaturos ir t.t. Nera vidutinio vartotojo, universali technologija reiskia kad visiem patogu naudotis, zinoma yra tam tikri kompromisai, reikalavimai turi tilpti i budget. Pvz.: ekrano padetis, ryskumas, kontrastas, color blindness, motion sensitivity, hearing etc. etc. Musu varottoju tarpe jeigu yra vyresni zmones tai turime i tai atsizvelgti, mygtukai didesni ir t.t. Kokia turi buti darbo aplinka kad zmogus nepavargtu, nebutu broko, darbo vieta turi buti tokia kuri leistu islaikyti darbinguma.

% psihologinis erdviu suvokimas, kai kurie pasiklysta, kai kurie gerai orientuojasi erdvese. kalbos skirtumai, kai kam gali buti izeidzianti kalba, kai kam ne...

% isiminimas, trumpalaikis isiminimas, sprendimu priemimas, komunikacija, paieska - mes turime sias veiklas kurioms kuriame technologijas.

% socialiniai skirtumai - itakoja motyva pirkti nauja technologija, gali atsirasti stiprus motyvas, pvz bendrauti nuotoliniu budu, todel zmones ismoks tai. skirstosi tipai, begginer, inermediate, expert zmoniu. pvz bileteliu popieriniu nenupirksi, tai beggineriai turi tai naudoti... vidutiniskai patyre kai reikia tia naudoja o taip tai nenaudoja. profesionalai - kiekviena diena naudoja. beggineriu ir expertu poreikia yra labai skirtingi.

% kad pradedantysis galetu naudotis technologija tai ji turi vesti uz rankos, pvz vedlys, jis vedamas uz rankos.

% ekspertai - jiem reikia greicio, juos vedlys erzins.

% vidutiniskai patyre - reikia prisiminti greitai.

% jie visi turi skirtingus poreikius, reikia suprasti su kuo turime reikala.


% Mental model - mintinis modelis, koki vaizda suformavo vartotojas savo galvoje, mes bandome tai suprasti, tai suvokimas kas ir kaip gali naudoti technologijas. Mintinis modelis nepilnas, nestabilus, mes negalime visko prisiminti, kazkas isilaiko ilgalaikeje atmintyje, kazkas dingsta. Analizuojame o ka dabar naudoja vartotojai, analizuojame ju mintines zinias. Mintinis modelis kuriamas saveikaujant su sistemomis, su kokiomis sistemomis saveikauja toki ir turesime mintini modeli.

% Nustatymas vartotojo igudzius yra musu tikslas. kokio amziaus, lytis, fizines, issilavinimas, kulturinis, motivacines galimybes, tikslai, asmenybe.
% Projektavimo tikslai turi susieti su tais skillais. musu tikslas kad vartotojai pilnai isnaudotu ka suprojektavome, norime kad is begginer greitai pereitu i intermediate.

% Pavyzdiziui kai pirma karta paleidziame programa, gali issokti gidas, padeti beginneriams, o kam nereikia tai gali uzdaryti, uzdaro vidutiniskai patyre. Ekspertam reikia kuo greiciau dirbti, kas stabdo (rankos kelimas nuo peles iki klaviaturos, galime sakyti kad viska darytu ant klaviaturos).

% kiek tu sluoksniu design reikia? tai papildomos islaidos ir t.t.


% begineriam programa turi pasakyti ka ji gali daryti, pagrindiniai dalykai ka ji daro, begineriu nera daug bet jie yra ir pirmi naudotojai todel jiems reikia viska paaiskinti. po keliu kartu dauguma begineriu tampa intermediate.
% intermediate svarbu matyti, isnaudoti visas funkcijas kurias jie zino kad yra, gebeti viska surasti, pazengusias funkcijas. GUI visas pagrindines ir pazengusias funkcijas rodo ir galima greitai surasti
% Ekspertu nera daug, kazkam galbut greiciau reikia.

% kuri viekla daznesne, ta turi buti arciau ekrano.


% Universal usability - prieinamas visiems, ir neigaliems. Panaudojamumas + prieinamumas = universalumas. kas aktualu, kam kuriame sistema, i sita turime atsakyti.



% Reikia pamineti demographics, age, occupation, gender (jei reikia), disabilities. Motyvacija ismokti technologijas, jomis naudotis. Naudojamos technologijos ir prietaisai, IT lygmuo, skillai.

% ____________________________

% Veiklos - reikia ne specifines mineti, o laiko apsketu daznas ar retas, bendradariavimo aspektus, individualus ar organizaciniai, tos veiklos sudetingumas, pasekmes, ar kritines, kokio tipo turinys toje veikloje.

% Daznis jos pirma charakteristika, jos trukme antra, laiko spaudimas, ar veikia
% ramybes busenoje ar skubos. Ar tai viena atomine veikla ar zingsniai yra, ar
% testine is zingsniu tai kazkuriame zingsnyje vartotojas gali sustoti. atsako
% laikas. Bendradarbiavimas, vienas ar grupeje, kaip zino kas ka padare, kaip
% koordinuoja, komunikuoja. Sudetingumas veiklos, ar veikla apibrezta ar ne,
% reikia suteikti vartotojui galimybe narsyti ivairias veiklas, kur vartotojas
% nori narsyti, kaip vartotojas supras ar nutrauke zingsni ir t.t. tai
% sudetingumas. kaip supras kad klaida padare, ar tai aktualu. kokie duomenys
% vaiksto, kiek ivesti ten reikia, kokia ten isvestis dabar ir kas siuloma, turi
% buti realiu laiku atnaujinamas turinys, kad nebutu blogai pavaizduotas, blogai
% atnaujintas.

% Ne pacios veiklos idomios bet ju charakteristikos.

% ____________________________

% Aplinka - fizine, kur vyksta ta veikla, patalpoje, uz patalpos, jeigu fizine aplinka lauke, tai skirtinga temperatura, kaip vartotojas su pirstinemis ar per lietu gali naudotis, apsvietimas, triuksmas ir t.t.

% Socialine aplinka - individuali, o gal grupine, ar privatumas aktualus, ar visi vienodu teisiu, ar yra super adminas.

%  Organizacinis kontekstas - teises ir t.t.

% ____________________________

% Technologijos - ka dabar naudoja zmones, kokias technologijas, ka jie dabar naudoja, tia itakoja ju mintini modeli, kaip jie dabar iveda informacija?

% Kokias technologijas naudosime - kompiuteris, programele telefone, ar t.t.

% Turime susivokti kokiu techniniu ir sistemu reikia. Kas yra geras turinys ir kokios charakteristikos galetu tobulinti tai, balsu ivedimas, AI ar whatever.

% kokia informacija ir funkcijos reikalingos sistemai, kas tures buti zinoma norinciam naudotis sitema.


% Interaktyvus produktas turi atitikti tai ko nori zmones, jis negali sugriauti kontekto, jis turi tai pagerinti.

% musu uzdavinys visus siuos dalykus issiaiskinti, parasyti kiekvienai pirminei ir antrinei vartotoju grupei aprasyti siuos dalykus ir po to dar kitai savaitei galime pradeti rasyti poreikius kas isto isplaukia ir kokiu funckiju riekia

% \sectionnonum{Results and conclusions}
% For details of what needs to be written in this section, please refer to the methodology requirements of the respective programme.


\section{PACT analysis}

\begin{center}
    \textcolor{red}{\textbf{(rough draft, dont look too deep into it yet)}}
\end{center}


We will use the \textbf{PACT} framework to analyze the needs of our system
users. That means we will look into the people and their physical and mental
capabilities, the characteristics of the activities they perform with current
systems, context in which they perform these activities, and the technologies
they use and which can improve their experience.


\subsection{Primary Stakeholders PACT analysis}

\subsubsection{Client Business Owners PACT analysis}

Lets start with analyzing the \textbf{people} aspect of PACT. 

\begin{itemize}
    \item \textbf{Physical capabilities:} Client business owners can have
    different physical capabilities, some of them might have visual or hearing
    impairments. This means that our system must have different contrast modes,
    font sizes, and support for screen readers.
    \item \textbf{Mental capabilities:} Client business owners can have
    different levels of technical expertise, mostly ranging from intermediate to
    expert. We expect that they will be able to use sales tracking, inventory
    management, and report generation features without much difficulty.
    \item \textbf{Psychological capabilities:} Client business owners can be
    from various cultural backgrounds and have different language preferences.
    This means that our system must support multiple languages and be culturally
    neutral.
\end{itemize}

Now lets analyze the \textbf{activities} aspect of PACT.

\begin{itemize}
    \item \textbf{Temporal aspects:} Client business owners will use the system
    daily, this means that the tasks like generating reports, tracking sales
    will be easy to memorize. Furthermore these tasks should be efficient to
    perform to avoid wasting time. Tasks like inventory management must be
    error-free as they can place time pressure on the user during busy periods.
    Furthermore this activity must be seperated into clear steps to avoid
    confusion. It is also necessary to have real time update of inventory data
    to avoid mistakes.
    \item \textbf{Cooperation:} Client business owners will mostly use the
    system individually, but they might need to share information with employees.
    There should be a clear way to share information.
    
    \item[] 
    \begin{center}
        \textcolor{red}{\textbf{(think deeper here, there is quite a lot of details to add)}}
    \end{center}


\end{itemize}

Now lets analyze the \textbf{context} aspect of PACT.

\begin{itemize}
    \item \textbf{Physical context:} Client business owners will use the system
    in a variety of physical contexts, including in their office, at home, or
    on the go. This means that our system should work well on different screen
    sizes and be responsive to different input methods (mouse, touch, keyboard).
    \item \textbf{Social context:} Client business owners will need to have administrative
    rights to manage employees and business operations. There should be a clear
    way to manage user roles and permissions.
\end{itemize}

Now lets analyze the \textbf{technologies} aspect of PACT.

\begin{itemize}
    \item \textbf{Technical capabilities:} Client business owners will have
    access to various devices (desktops, laptops, tablets, smartphones) and
    operating systems (Windows, macOS, iOS, Android). Our system must be
    compatible with these platforms and provide a consistent user experience.
    \item \textbf{Software capabilities:} Client business owners may use
    different software applications (CRM, ERP, Office Suite) alongside our
    system. We should consider integration options and data exchange formats
    (APIs, CSV, etc.) to facilitate smooth workflows.
    \item \textbf{Network capabilities:} Client business owners will rely on
    internet connectivity to access our system. We must ensure that our system
    performs well under different network conditions (Wi-Fi, mobile data).
\end{itemize}

\printbibliography[title = {References and sources}]

% \appendix
% \renewcommand{\thesection}{Appendix \arabic{section}. }

% \section{\phantom{Appendix} Examples of citations}
% In the document \textit{bibliography.bib}, you need to add all the cited sources and after using the function \textit{\{\textbackslash cite\{name of the cited object\}\}} the corresponding source will be added to the list of literature sources.


% \textit{bibliography.bib} provides examples of some of the most commonly cited types of sources:
% \begin{itemize}
%     \item web pages (\textit{@online}) \cite{PvzInternetinisPuslapis},
%     \item datasets (\textit{@dataset}) \cite{dataset}
%     \item articles (\textit{@article}) \cite{PvzStraipsnLt, PvzStraipsnEn}, 
%     \item articles from conferences (\textit{@inproceedings}) \cite{PvzKonfLt, PvzKonfEn}, 
%     \item books (\textit{@book}) \cite{PvzKnygLt, PvzKnygEn}, 
%     \item theses (\textit{@thesis or mastersthesis/phdthesis} \cite{PvzMagistrLt, PvzPhdEn})
%     \item electronic publications (\textit{@misc}) \cite{PvzElPubLt, PvzElPubEn}
% \end{itemize}

% Examples are also provided for ChatGPT citation, both in general \cite{chatgpt_bendrai} and for a specific conversation \cite{chatgpt_pokalbis}.

\end{document}
