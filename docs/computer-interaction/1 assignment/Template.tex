%%%%%
%%%%%  Use LUALATEX, not LATEX.
%%%%%
%%%%
\documentclass[]{VUMIFTemplateClass}

\usepackage{indentfirst}
\usepackage{amsmath, amsthm, amssymb, amsfonts}
\usepackage{mathtools}
\usepackage{physics}
\usepackage{graphicx}
\usepackage{verbatim}
\usepackage[hidelinks]{hyperref}
\usepackage{color,algorithm,algorithmic}
\usepackage[nottoc]{tocbibind}
\usepackage{tocloft}

\usepackage{amssymb}

\usepackage{titlesec}
\newcommand{\sectionbreak}{\clearpage}

\usepackage{titlesec}

\setcounter{secnumdepth}{4}
\setcounter{tocdepth}{3}

\titleformat{\paragraph}
{\normalfont\normalsize\bfseries}{\theparagraph}{1em}{}
\titlespacing*{\paragraph}
{0pt}{3.25ex plus 1ex minus .2ex}{1.5ex plus .2ex}

% Create the custom command
\newcommand{\subsubsubsection}[1]{\paragraph{#1}}

\makeatletter
\renewcommand{\fnum@algorithm}{\thealgorithm}
\makeatother
\renewcommand\thealgorithm{\arabic{algorithm} algorithm}

\usepackage{biblatex}
\bibliography{bibliografija}
%% to change the numbering (numeric or alphabetic) of bibliographic sources, make the change in VUMIFTemplateClass.cls, line 139

% Author's MACROS
\newcommand{\EE}{\mathbb{E}\,} % Mean
\newcommand{\ee}{{\mathrm e}}  % nice exponent
\newcommand{\RR}{\mathbb{R}}




\studyprogramme{Software engineering} %Write your study 


% pilnas pavadinimas, aiskina ka visa tai daro
% prasmingas pavadinimas cia
\worktitle{\textit{UniServe}: A Universal Service Management Platform for Reservation and Order-Based Businesses} 
\workauthor{Tadas Riksas, Darius Spruogis, Gustas Mickus, Kajus Bicka}

\supervisor{Kristina Lapin}

\begin{document}
\selectlanguage{english}

\onehalfspacing
\begin{titlepage}
\vskip 20pt
\begin{center}
\includegraphics[scale=0.55]{images/MIF.png}
\end{center}

\makeatletter

\vskip 20pt
\centerline{\bf \large \textbf{VILNIUS UNIVERSITY}}
\vskip 10pt
\centerline{\large \textbf{FACULTY OF MATHEMATICS AND INFORMATICS}}
\vskip 10pt
\centerline{\large \textbf{\MakeUppercase{\@studyprogramme \space study programme}}}

\vskip 80pt
\centerline{\Large \@worktype}
\vskip 20pt
\begin{center}
    {\bf \LARGE \@worktitle}
\end{center}
\begin{center}
    {\bf \Large \@secondworktitle}
\end{center}
\vskip 80pt

\centering{\Large \@workauthor}
\@ifundefined{@secondauthor}{}
{
\vskip 10pt
\centering{\Large \@secondauthor}
}
\vskip 20pt

\centering{
    \begin{tabular}{rcp{.7\textwidth}}
        {\Large Supervisor} & {\Large :} & {\Large \@supervisor}\\[10pt]
        \@ifundefined{@scientificadvisor}{}
            {
                {\Large Scientific advisor} & {\Large :} & {\Large \@scientificadvisor}\\[10pt]
            }
        \@ifundefined{@reviewer}{}
            {
                {\Large Reviewer} & {\Large :} & {\Large \@reviewer}\\[10pt]
            }
    \end{tabular}}


\vskip 110pt

\centerline{\large \textbf{Vilnius}}
\centerline{\large \textbf{\the\year{}}}

\makeatother

\newpage
\end{titlepage}
%\newgeometry{top=2cm,bottom=2cm,right=2cm,left=3cm}
\setcounter{page}{2}


\singlespacing
\selectlanguage{english}
% list of figures, delete if not needed
% \listoffigures 

%list of tables, delete if not needed
% \listoftables

\tableofcontents
\onehalfspacing


\section*{Check List \checkmark}

\section*{Assignment Requirements}

\subsection*{Documentation Requirements}

The assignment consists of the following parts: title page, table of contents,
and the main part.

\textbf{Section numbering.} Headings of the sections must be numbered strictly
hierarchically. The sections have assigned numbers, such as 1., subsections –
1.1., 1.2, subsubsections – 1.1.1., 1.1.2. and so on. \checkmark

\textbf{File naming requirements.} A file name contains a short project title,
an assignment number and an assignment title. Example: CarBuddy 1 User needs.pdf

\subsection*{An outline of the main part}

\begin{enumerate}
    \item Introduction \checkmark
    \begin{enumerate}
        \item[1.1.] Project titles \checkmark
        \item[1.2.] Problem statement \checkmark
    \end{enumerate}
    
    \item User needs analysis
    \begin{enumerate}
        \item[2.1.] Expectations of the stakeholders \checkmark
        \item[2.2.] $\langle$The name of the first user group$\rangle$ research \{for each user group in both primary and secondary stakeholders\} \checkmark
        \begin{enumerate}
            \item[2.2.1.] Current users' activities
            \begin{enumerate}
                \item[2.2.1.1.] $\langle$Title of the 1st user story$\rangle$
                \begin{itemize}
                    \item Description of current activity (unnumbered subsection)
                    \item Problems and opportunities (unnumbered subsection)
                \end{itemize}
                \item[2.2.1.2.] $\langle$Title of the 2nd user story$\rangle$
                \begin{itemize}
                    \item \ldots
                \end{itemize}
            \end{enumerate}
            \item[2.2.2.] Characteristics of the people, activities, context, and technologies
            \item[2.2.3.] Needs
            \item[2.2.4.] Usability objectives
        \end{enumerate}
        \item[2.3.] $\langle$The name of the next user group$\rangle$ needs analysis \{if needed\}
        \item[] \ldots
        \item[2.$\langle$n+1$\rangle$.] Inspiring user interface designs
    \end{enumerate}
\end{enumerate}

\subsection*{Explanations}

\textbf{The title page} contains:
\begin{itemize}
    \item university name, faculty, and study program, \checkmark
    \item project full title, \checkmark
    \item contributors' names and surnames, \checkmark
    \item year. \checkmark
\end{itemize}

\textbf{The table of contents} includes the sections and subsections until the 3rd level and page numbers. \checkmark

\textbf{Project titles.} The project must have two titles: \checkmark
\begin{itemize} 
    \item full (e.g. ``Vilnius University Library App'') and \checkmark
    \item short (e.g. ``Library''). \checkmark
\end{itemize}

The full title is used on the title page and in this section, a short title serves as a project reference in the document text and assignment file names. \checkmark

\textbf{Problem statement} explains the high-level project goals shortly and
specifically (about half a page). The problem is described in terms of user
activities and situations where the problem occurs, and what aspects might be
improved with a technical solution. Avoid describing or suggesting a solution at
this stage that will hamper your design thinking when you start solving the
problem. \checkmark

To gather the relevant facts for your problem statement, you can use a simple
technique called the 6 Ws, which involves answering the questions below: \checkmark
\begin{itemize}
    \item Who is affected by the problem? \checkmark
    \item What is the problem? \checkmark
    \item Where does this problem occur? \checkmark
    \item When does the problem occur? \checkmark
    \item Why does the problem occur? \checkmark
    \item Why is the problem important? \checkmark
\end{itemize}

\textbf{Expectations of the stakeholders} \checkmark

Specify the specific groups of people (stakeholders) involved in activities that
will be supported by your project (see Lecture 1. User-centered design).
Identify relevant stakeholders' intentions that are supposed to be fulfilled. In
the competitors subsection, you should list your closest competitors. Then you
should list your main competitive advantages over them. \checkmark

\textbf{Current user activities}

Watching how people do things is a great way to learn their goals and values, and come up with design insight. This is called user research. This assignment helps you train your eyes and ears to develop design ideas. Your goal is to uncover user needs, breakdowns, clever hacks, and opportunities for improvement. Begin by selecting a specific activity to observe.

The description contains the user story that tells about the current user activities. Note that your service or product does not exist yet. Therefore, you must describe the existing activities, such as how your users achieve their goals with existing competitive products, and highlight the pain points and troubles. Then, indicate the problems and improvement opportunities (see the assignment outline).

Describe at least 3 observations of current user activities for primary users, for the secondary – there can be fewer. Examples of activity descriptions:

\begin{itemize}
    \item \textbf{Story 1:} Ann is a clerk in her twenties, comes for a quick lunch. She wants to see dishes that are served within 15 min. She waits 10 min. for a waiter. The waiter says that only soup can be served in 15 min. Therefore Ann leaves the restaurant and goes for a fast food nearby.
    
    \item \textbf{Story 2:} Andrew, a thirty-two-year-old restaurant client, who suffers from allergies, comes to a restaurant. He examines the menu. The menu is long, ingredients are mentioned in every dish. So, it takes about 5 minutes to choose a dish. Waiting for the dish takes 20 min. Andrew must hurry and eats quickly because he has to return to work on time.
    
    \item \textbf{Story 3:} Thirty-five-years-old Peter goes to a restaurant with his family. He noticed on Google Maps that a new restaurant was opening not far away. They spend time reading the menu. The café has a children's menu, but children's favorite dishes are missing. Children are bored because of missing the game area. The family chooses quick dishes from the menu.
    
    \item \textbf{Story 4:} Fourty-year-old Sara comes with her friends to a restaurant on Sunday. They are looking for a big table but see that the restaurant is almost full. They started to explore the TripAdvisor app on their phone for other restaurants phone numbers to ask whether they have a free big table for a group. \ldots
\end{itemize}

\textbf{How to check whether a story is correct?} A story should cover characteristics of peoples, activities, environment and technologies that are related to computerized activities. If any characteristic is missing, you should augment the story.

\textbf{Needs statements and usability objectives}

After describing user stories, go over your findings. Brainstorm about both user needs and usability objectives. Observe the opportunities for design innovation that would improve the activity. User needs are formulated for primary and secondary users, only. Indirect stakeholders formulate business goals that are abstract and relate to business strategy. Business goals should be met with the needs of primary and secondary users. The impact of business goals on user needs is explained in Lecture 2, slides 52–53. Examples of usability goals and needs are provided in slides 54--57, and usability objectives are tackled in slides 58--60.

The project must contain at least 15 user needs. 3–5 usability objectives are defined for every user group. You are not looking for solutions yet: focus on user needs only. User needs and usability objectives must have unique identifiers. Below the needs and the usability objectives provide traceability matrices that link the need or objective identifier with the number of paragraph of the user story from which it is derived.

\textbf{Inspiring user interface designs}

Provide here examples of successful user interface solutions of existing software that can be useful for specified user needs and usability objectives. Provide a screenshot of the design, the number of user needs, or the usability objective for which this solution would be beneficial. Images should have captions with concise explanations and references to the source, from which this image is taken. The text related to the image should contain cross-reference to the image. Provide at least 5 examples.

\section*{Use of AI}

Just a little note: dont just copy the assignment text and expect good output,
look into slides also and try to apply lecture material. 


\section{Introduction}
\subsection{Project title}

\textbf{Full title:} \textit{UniServe: A Universal Service Management Platform for Reservation and Order-Based Businesses}

\textbf{Short title:} \textit{UniServe}


\textbf{Explanation:} "Uni" represents \emph{universal}, indicating the platform's flexibility across a variety of business models 
(reservation based businesses such as massage services or order based businesses such as restaurants). 
"Serve" refers to \emph{service}, emphasizing the platform’s role in providing core functionalities for these types of businesses.

\subsection{Problem statement}

Traditional methods of managing reservations and orders—such as handwritten
appointment books, paper menus, phone-based booking, and manual order
taking—create significant operational bottlenecks that directly impact business
performance and stakeholder satisfaction. Reliance on outdated manual processes
create several critical problems:

\begin{itemize}
    \item \textbf{For employees:} Manual order-taking and appointment booking
    requires constant \textbf{memorization} of availability, pricing, and menu details.
    Staff must physically check paper schedules or call to verify bookings,
    leading to \textbf{errors}, double-bookings, and frustrated customers. During peak
    periods (weekends, holidays, promotional events), employees become
    \textbf{overwhelmed} managing multiple phone lines while \textbf{simultaneously serving}
    in-person customers, resulting in poor service quality and increased stress.

    \item \textbf{For customers:} Modern consumers expect the
    \textbf{convenience} of online ordering and booking available 24/7, similar
    to services like Square or Toast. When forced to call during business hours
    or wait in line to place orders, customers often \textbf{abandon their
    purchase intent}. The \textbf{inability} to view real-time availability,
    modify orders easily, or receive confirmation notifications leads to
    customer \textbf{dissatisfaction} and \textbf{frustration} as customers
    cannot make informed decisions or plan their time effectively.

    \item \textbf{For business owners:} Manual systems prevent access to crucial
    business data—owners \textbf{cannot easily} track popular items, peak hours, or
    customer patterns. Inventory management relies on \textbf{guesswork}, leading to
    waste or stockouts. Without automated processes, businesses require more
    staff hours for administrative tasks that could be automated, increasing
    labor costs while \textbf{limiting growth potential}.
\end{itemize}

\textbf{Why this problem matters now:}

The COVID-19 pandemic accelerated consumer adoption of digital ordering and
contactless services. Customers now view online booking and ordering as standard
expectations rather than luxury features. Businesses without these capabilities
lose customers to competitors who offer digital convenience, even if their core
service quality is superior.

\textbf{Technical barriers:}

Limited budgets and lack of technical expertise prevent these businesses from
developing in-house solutions as these solutions are often expensive and complex.

\textbf{Consequences of inaction:}

Without accessible digital platforms, businesses experience reduced customer
reach (losing tech‑savvy customers who prefer online interaction), decreased
operational efficiency (staff spending time on manual tasks instead of customer
service), higher recurring labor costs (requiring more staff for administrative
work), and diminished competitive position (falling behind digitally-enabled
competitors).


\section{Stakeholder analysis}

\subsection{Identifying stakeholders}

\begin{table}[h]
  \centering
  \caption{Stakeholders identification}
  \begin{tabular}{|c|c|}
    \hline
    Stakeholder type    & Stakeholders \\ \hline
    Primary             & Employees, customers \\ \hline
    Secondary           & Client business owners \\ \hline
    Indirect (tertiary) & App business owners \\ \hline
    External/competitors & Square, Toast POS, Restaurant365 \\ \hline
    Technical support   & Sysadmins, tech support \\ \hline
  \end{tabular}
  \label{tab:stakeholders}
\end{table}

% \textbf{Primary Stakeholders} directly and frequently use the system. Client
% business owners (restaurant/salon owners) operate the system daily. Employees
% process orders and manage appointments through the application. Customers use
% the system when placing orders or booking appointments.

% \textbf{Secondary Stakeholders} are occasionally involved or interact via
% others. 
% Managers oversee operations and may need access to system statistics.

% \textbf{Indirect (Tertiary) Stakeholders} influence purchase decisions and
% benefit from the system. App business owners (our company's owners) profit from
% the system's success. Business managers make decisions about adopting the
% system.

% \textbf{Competitors} represent existing market solutions. Users currently
% employing Square, Toast POS, or similar platforms are potential adopters of our
% system.

% \textbf{Development and Maintenance Team} creates and maintains the system.
% Sysadmins handle infrastructure and servers. Tech support staff resolve user
% issues and provide assistance.

% Nielsen‘s principles
% 1. Learnability
% 2. Efficiency of use
% 3. Memorability
% 4. Few and non-
% catastrophic errors
% 5. Satisfaction

\textbf{Primary stakeholders:}
\begin{itemize}
    \item \textbf{Employees} expect a system that eliminates the need to
    memorize constantly changing information (availability, pricing, specials),
    reduces the risk of errors in bookings and order processing, and allows for
    a more satisfying and efficient customer serving experience.
    \item \textbf{Customers} expect 24/7 convenience for booking and ordering,
    real-time availability information, effective order modifications, and
    reliable confirmation notifications that enable informed and satisfying
    decision-making. Furthermore customers expect to have similar convenience as
    they have with other digital services.
\end{itemize}

\textbf{Secondary stakeholders:}
\begin{itemize}
    \item \textbf{Client business owners} expect automated access to business
    data (sales patterns, inventory levels, customer analytics) that are
    error-free, reduced administrative overhead. They expect efficient systems
    that will reduce labor and allow them to focus on strategic decision-making.
\end{itemize}

\textbf{Indirect (tertiary) stakeholders:}
\begin{itemize}
    \item \textbf{App business owners} recognize a market opportunity to serve
    underserved small and medium businesses that cannot afford or implement
    complex enterprise solutions. They want steady income growth by capturing
    market share from businesses currently using manual processes or considering
    expensive alternatives.
\end{itemize}

\textbf{Competitors:}
\begin{itemize}
    \item \textbf{Square, Toast POS, Restaurant365} serve similar markets but
    often with complex, expensive solutions targeted at larger businesses. Small
    business owners currently using these platforms (or those considering them)
    expect affordable pricing and guides allowing quick onboarding
    without extensive technical training, and features that directly address the
    problems outlined above. Our competitive advantage lies in providing
    simplified, cost-effective solutions specifically designed for small to
    medium businesses that need core functionality without enterprise
    complexity.
\end{itemize}

\textbf{Development and maintenance team:}
\begin{itemize}
    \item \textbf{Sysadmins} expect reliable system architecture that can handle
    varying loads (especially during peak business periods), scalable
    infrastructure to accommodate business growth, and clear deployment
    procedures to maintain system stability.
    \item \textbf{Tech support} expect comprehensive documentation and effective
    troubleshooting tools to efficiently resolve the types of issues small
    business users commonly encounter, along with manageable processes for user
    support that don't overwhelm limited support resources.
\end{itemize}


% <The name of the first user group> research {for each user group in both primary and secondary stakeholders}

% should include people characters, activities, enviroment and technologies or
% in other words it should give info to PACT analysis.          
% Character - who are the users, what are their physical and cognitive abilities,
% Activities - what tasks do they perform, what are their goals, describe the characteristics of tasks
% Context - where and when do they perform these tasks, what are the physical, social, cultural, organizational aspects of the environment
% Technologies - what tools, devices, software do they use to perform their tasks.
\subsection{Employee research}

\subsubsection{Current users' activities}

\subsubsubsection{Café Barista Managing Orders}

% Character - info needs more details, 21 years old. but what else? height, weight, experience, is he a fast learner? good memorization skills?
% Activities - takes orders at the counter. what are the characteristics of these tasks? (dont write the analysis of tasks, just describe the tasks, is it a complex task? does it require high concentration? is it repetitive? etc)
% Context - busy cafe, morning rush, sticky notes, smudged notes
% Technologies - just sticky notes
% /\ potential improvments if the current stuff is enough.
Marius, 21, works as a barista in a busy café. He takes orders at the counter
and writes them on sticky notes for the kitchen. During the morning rush, notes
sometimes fall to the floor or get smudged by spilled drinks. This leads to
wrong orders being prepared, which frustrates customers and wastes ingredients.
Marius tries to keep the counter tidy, but the lack of a structured system makes
mistakes inevitable and thus Marius spends extra time apologizing and remaking
orders, which slows down the queue.

\subsubsubsection{Salon Receptionist Controlling Bookings}

Lucille, in her twenties, is the receptionist at a busy, well‑known beauty
salon. Throughout the day she fields a constant stream of phone calls to arrange
appointments. Each booking is first written by hand in a paper diary, then later
re‑entered into a spreadsheet. On busy days, at least one record is often missed
or entered incorrectly, resulting in double bookings or missed appointments.
When a client calls to reschedule, Lucille leafs through pages of the diary to
locate the original booking. This takes time, breaks her concentration, and can
leave callers waiting on the line. The salon offers no online booking facility,
so all scheduling relies entirely on Lucille being available to answer the
phone.

\subsubsubsection{Restaurant Manager Handling Discounts}


Youmna, in her early twenties, manages a vibrant Lebanese restaurant on a busy
high street. To attract customers during quieter weekday afternoons, the
restaurant runs a “Taste of Lebanon” promotion, offering discounts on selected
platters and traditional dishes. When diners ask about the offer, Youmna or the
waiting staff consult a printed list of eligible dishes and manually work out
the reduced price at the till. At busy times, this slows service. Occasionally,
the discount is forgotten, prompting awkward conversations and amended bills; at
other times, it is applied in error, reducing profit margins. To monitor the
promotion, Youmna records discounted orders in a small ledger. However, during
the rush, staff often forget to make entries, leaving the records incomplete.
This makes it difficult to assess the effectiveness of the offer or refine it
for future campaigns.

\subsubsection{PACT analysis of employees}

\subsubsection{Needs of employees}

\subsubsection{Usability objectives for employees}

\subsection{Customer research}

\subsubsection{Current users' activities}

\subsubsubsection{Looking for a quick lunch}


Annie, a 20‑year‑old university student, has a short break between lectures and
only 30 minutes to eat. She visits a small café she has not tried before, hoping
to find something quick. The menu gives no indication of preparation times, so
she chooses a dish at random and waits nearly 10 minutes for a waiter to take
her order. She is then told it will take 25 minutes to prepare. With little time
left, Annie leaves and buys a takeaway wrap from a nearby chain instead.

\newpage

\subsubsubsection{Booking a haircut}


Felix, in his late forties, needs a haircut before a family gathering. He calls
his usual barber, but the line is engaged. After several attempts, he finally
gets through, only to learn that his preferred time slot has already been taken.
The receptionist offers alternative appointments, but Felix is at work and
cannot check his calendar straight away. Frustrated, he decides to book with
another salon that can fit him in immediately.

\subsubsubsection{Redeeming a coffee promotion}


Sarah‑Louise, 27, spots a “Buy One, Get One Free” coffee offer on a café’s
social media page. She visits the café to take advantage of the deal, but the
barista is unaware of the promotion and has to confirm it with the manager. This
causes a delay at the counter and creates an awkward queue behind  her. The
discount is eventually applied, but Sarah‑Louise leaves feeling the process was
poorly organised.

\subsubsection{PACT analysis of customers}

\subsubsection{Needs of customers}

\subsubsection{Usability objectives for customers}

\subsection{Client Business Owners Research (Secondary User Group)}

\subsubsection{Current users' activities}

\subsubsubsection{Restaurant Owner Handling Customer Feedback}

Apolline, 39, owns a family restaurant in a popular tourist area. Customers
occasionally leave reviews on Google Maps or TripAdvisor, but she only checks
them sporadically. Complaints (such as “long wait for a table” or “incorrect
order”) can go unnoticed for several days. Apolline also keeps a suggestion box
in the restaurant, though very few customers make use of it. She would prefer to
gather feedback in a more systematic way and respond swiftly in order to improve
the service. Without structured data, it is difficult for her to identify
recurring issues (for example, slow service during specific time frames) and to
train staff appropriately.

\subsubsubsection{Salon Owner Tracking Staff Performance}

Nils, in his forties, runs a busy salon with six stylists. He tries to track
performance and service quality, but still relies on a paper diary and casual
feedback. Calculating wages and commissions means hours spent combing through
notes and receipts. When figures are challenged, he has no firm record to refer
to, which not only heightens administrative burden but also sows discontent
amongst the team.

\subsubsection{PACT analysis of Client Business Owners}

\subsubsection{Needs of Client Business Owners}

\subsubsection{Usability objectives for Client Business Owners}


% ------------------------
% PACT stuff below
% ------------------------



% reikia sistemizuoti kokioje situacijoje ko reikia, labiau detaliau issiaiskinti stakeholder needs.
% reikia atlikti naudojimo konteksto analize, kaip tos funkcijos isilies i vartotojo esama gyvenima. Mes galvojame apie naudojima ir konteksta vis dar.
% Naudojimo kontekstas - kaip veiklos itakoja technologiju kaita ir technologiju kaita veiklas, po to kalbesime apie pagrindines zmoniu charakteristikas, kas, kokiose veiklose ir problemas kurias norime spresti, kokiame zingsnyje kas ten stringa, kokioje aplinkoje ten vyksta, kokios tech naudojamos ir ka galime pasiulyti.
% Naudojimo kontekstas - kombinacija vartotoju, uzduociu, resursu, tikslu. Fizines, socialines, technines, kulturines, organizicines aplinkos. Jas riekes issaiskinti ir aprasyti kaip dabar veikia tas vartotojas, ka tie pasakojimai turi atspindeti: turi buti aiskus visi aspektai, kas, kokie tisklai, aplinkos, kokiais prietaisais, programomis naudojasi ir t.t.
% PS inzinerijos standartas - naudojimo kontekstas pagrindinis informacijos reikalavimu saltinis.
% Vystimasis yra begalinis ciklas, yra zmones betkokio amziaus, kutluros, jie veikia kazkokiam kontekste (requirements), tame kontekste kyla nauji reikalavimai naujoms technologijosm, naujos technologijos sudaro galimybe keistis veiklos, vel atsiras nauji reikalavimai ir t.t. gaunasi begalinis ratas.
% naujos interaktyvios tech - prisidedame dabartines zinias, sukaupta patirti ir kuriame naujus reikalavimus. Suvokdami kazkokius nepatogumus, koks yra stovis, kokias galimybes mes galime isnaudoti su technologijomis mes pasiulome sprendimus, naujas veiklas.

% People, context, activities, technologies

% ____________________________

% People - kokiomis salygomis veikia, ju fizines galimybes, darbo vieta, veikimo aplinka kur bus naudojama technologija, zmoniu ugis, aukstis, antropronetiniai dalykai? ivestis isvestis, klaviaturos ir t.t. Nera vidutinio vartotojo, universali technologija reiskia kad visiem patogu naudotis, zinoma yra tam tikri kompromisai, reikalavimai turi tilpti i budget. Pvz.: ekrano padetis, ryskumas, kontrastas, color blindness, motion sensitivity, hearing etc. etc. Musu varottoju tarpe jeigu yra vyresni zmones tai turime i tai atsizvelgti, mygtukai didesni ir t.t. Kokia turi buti darbo aplinka kad zmogus nepavargtu, nebutu broko, darbo vieta turi buti tokia kuri leistu islaikyti darbinguma.

% psihologinis erdviu suvokimas, kai kurie pasiklysta, kai kurie gerai orientuojasi erdvese. kalbos skirtumai, kai kam gali buti izeidzianti kalba, kai kam ne...

% isiminimas, trumpalaikis isiminimas, sprendimu priemimas, komunikacija, paieska - mes turime sias veiklas kurioms kuriame technologijas.

% socialiniai skirtumai - itakoja motyva pirkti nauja technologija, gali atsirasti stiprus motyvas, pvz bendrauti nuotoliniu budu, todel zmones ismoks tai. skirstosi tipai, begginer, inermediate, expert zmoniu. pvz bileteliu popieriniu nenupirksi, tai beggineriai turi tai naudoti... vidutiniskai patyre kai reikia tia naudoja o taip tai nenaudoja. profesionalai - kiekviena diena naudoja. beggineriu ir expertu poreikia yra labai skirtingi.

% kad pradedantysis galetu naudotis technologija tai ji turi vesti uz rankos, pvz vedlys, jis vedamas uz rankos.

% ekspertai - jiem reikia greicio, juos vedlys erzins.

% vidutiniskai patyre - reikia prisiminti greitai.

% jie visi turi skirtingus poreikius, reikia suprasti su kuo turime reikala.


% Mental model - mintinis modelis, koki vaizda suformavo vartotojas savo galvoje, mes bandome tai sufprasti, tai suvokimas kas ir kaip gali naudoti technologijas. Mintinis modelis nepilnas, nestabilus, mes negalime visko prisiminti, kazkas isilaiko ilgalaikeje atmintyje, kazkas dingsta. Analizuojame o ka dabar naudoja vartotojai, analizuojame ju mintines zinias. Mintinis modelis kuriamas saveikaujant su sistemomis, su kokiomis sistemomis saveikauja toki ir turesime mintini modeli.

% Nustatymas vartotojo igudzius yra musu tikslas. kokio amziaus, lytis, fizines, issilavinimas, kulturinis, motivacines galimybes, tikslai, asmenybe.
% Projektavimo tikslai turi susieti su tais skillais. musu tikslas kad vartotojai pilnai isnaudotu ka suprojektavome, norime kad is begginer greitai pereitu i intermediate.

% Pavyzdiziui kai pirma karta paleidziame programa, gali issokti gidas, padeti beginneriams, o kam nereikia tai gali uzdaryti, uzdaro vidutiniskai patyre. Ekspertam reikia kuo greiciau dirbti, kas stabdo (rankos kelimas nuo peles iki klaviaturos, galime sakyti kad viska darytu ant klaviaturos).

% kiek tu sluoksniu design reikia? tai papildomos islaidos ir t.t.


% begineriam programa turi pasakyti ka ji gali daryti, pagrindiniai dalykai ka ji daro, begineriu nera daug bet jie yra ir pirmi naudotojai todel jiems reikia viska paaiskinti. po keliu kartu dauguma begineriu tampa intermediate.
% intermediate svarbu matyti, isnaudoti visas funkcijas kurias jie zino kad yra, gebeti viska surasti, pazengusias funkcijas. GUI visas pagrindines ir pazengusias funkcijas rodo ir galima greitai surasti
% Ekspertu nera daug, kazkam galbut greiciau reikia.

% kuri viekla daznesne, ta turi buti arciau ekrano.


% Universal usability - prieinamas visiems, ir neigaliems. Panaudojamumas + prieinamumas = universalumas. kas aktualu, kam kuriame sistema, i sita turime atsakyti.



% Reikia pamineti demographics, age, occupation, gender (jei reikia), disabilities. Motyvacija ismokti technologijas, jomis naudotis. Naudojamos technologijos ir prietaisai, IT lygmuo, skillai.

% ____________________________

% Veiklos - reikia ne specifines mineti, o laiko apsketu daznas ar retas, bendradariavimo aspektus, individualus ar organizaciniai, tos veiklos sudetingumas, pasekmes, ar kritines, kokio tipo turinys toje veikloje.

% Daznis jos pirma charakteristika, jos trukme antra, laiko spaudimas, ar veikia
% ramybes busenoje ar skubos. Ar tai viena atomine veikla ar zingsniai yra, ar
% testine is zingsniu tai kazkuriame zingsnyje vartotojas gali sustoti. atsako
% laikas. Bendradarbiavimas, vienas ar grupeje, kaip zino kas ka padare, kaip
% koordinuoja, komunikuoja. Sudetingumas veiklos, ar veikla apibrezta ar ne,
% reikia suteikti vartotojui galimybe narsyti ivairias veiklas, kur vartotojas
% nori narsyti, kaip vartotojas supras ar nutrauke zingsni ir t.t. tai
% sudetingumas. kaip supras kad klaida padare, ar tai aktualu. kokie duomenys
% vaiksto, kiek ivesti ten reikia, kokia ten isvestis dabar ir kas siuloma, turi
% buti realiu laiku atnaujinamas turinys, kad nebutu blogai pavaizduotas, blogai
% atnaujintas.

% Ne pacios veiklos idomios bet ju charakteristikos.

% ____________________________

% Aplinka - fizine, kur vyksta ta veikla, patalpoje, uz patalpos, jeigu fizine aplinka lauke, tai skirtinga temperatura, kaip vartotojas su pirstinemis ar per lietu gali naudotis, apsvietimas, triuksmas ir t.t.

% Socialine aplinka - individuali, o gal grupine, ar privatumas aktualus, ar visi vienodu teisiu, ar yra super adminas.

%  Organizacinis kontekstas - teises ir t.t.

% ____________________________

% Technologijos - ka dabar naudoja zmones, kokias technologijas, ka jie dabar naudoja, tia itakoja ju mintini modeli, kaip jie dabar iveda informacija?

% Kokias technologijas naudosime - kompiuteris, programele telefone, ar t.t.

% Turime susivokti kokiu techniniu ir sistemu reikia. Kas yra geras turinys ir kokios charakteristikos galetu tobulinti tai, balsu ivedimas, AI ar whatever.

% kokia informacija ir funkcijos reikalingos sistemai, kas tures buti zinoma norinciam naudotis sitema.


% Interaktyvus produktas turi atitikti tai ko nori zmones, jis negali sugriauti kontekto, jis turi tai pagerinti.

% musu uzdavinys visus siuos dalykus issiaiskinti, parasyti kiekvienai pirminei ir antrinei vartotoju grupei aprasyti siuos dalykus ir po to dar kitai savaitei galime pradeti rasyti poreikius kas isto isplaukia ir kokiu funckiju riekia

% \sectionnonum{Results and conclusions}
% For details of what needs to be written in this section, please refer to the methodology requirements of the respective programme.


% \section{PACT analysis}

% \begin{center}
%     \textcolor{red}{\textbf{(rough draft, dont look too deep into it yet)}}
% \end{center}


% We will use the \textbf{PACT} framework to analyze the needs of our system
% users. That means we will look into the people and their physical and mental
% capabilities, the characteristics of the activities they perform with current
% systems, context in which they perform these activities, and the technologies
% they use and which can improve their experience.


% \subsection{Primary Stakeholders PACT analysis}

% \subsubsection{Client Business Owners PACT analysis}

% Lets start with analyzing the \textbf{people} aspect of PACT. 

% \begin{itemize}
%     \item \textbf{Physical capabilities:} Client business owners can have
%     different physical capabilities, some of them might have visual or hearing
%     impairments. This means that our system must have different contrast modes,
%     font sizes, and support for screen readers.
%     \item \textbf{Mental capabilities:} Client business owners can have
%     different levels of technical expertise, mostly ranging from intermediate to
%     expert. We expect that they will be able to use sales tracking, inventory
%     management, and report generation features without much difficulty.
%     \item \textbf{Psychological capabilities:} Client business owners can be
%     from various cultural backgrounds and have different language preferences.
%     This means that our system must support multiple languages and be culturally
%     neutral.
% \end{itemize}

% Now lets analyze the \textbf{activities} aspect of PACT.

% \begin{itemize}
%     \item \textbf{Temporal aspects:} Client business owners will use the system
%     daily, this means that the tasks like generating reports, tracking sales
%     will be easy to memorize. Furthermore these tasks should be efficient to
%     perform to avoid wasting time. Tasks like inventory management must be
%     error-free as they can place time pressure on the user during busy periods.
%     Furthermore this activity must be seperated into clear steps to avoid
%     confusion. It is also necessary to have real time update of inventory data
%     to avoid mistakes.
%     \item \textbf{Cooperation:} Client business owners will mostly use the
%     system individually, but they might need to share information with employees.
%     There should be a clear way to share information.
    
%     \item[] 
%     \begin{center}
%         \textcolor{red}{\textbf{(think deeper here, there is quite a lot of details to add)}}
%     \end{center}


% \end{itemize}

% Now lets analyze the \textbf{context} aspect of PACT.

% \begin{itemize}
%     \item \textbf{Physical context:} Client business owners will use the system
%     in a variety of physical contexts, including in their office, at home, or
%     on the go. This means that our system should work well on different screen
%     sizes and be responsive to different input methods (mouse, touch, keyboard).
%     \item \textbf{Social context:} Client business owners will need to have administrative
%     rights to manage employees and business operations. There should be a clear
%     way to manage user roles and permissions.
% \end{itemize}

% Now lets analyze the \textbf{technologies} aspect of PACT.

% \begin{itemize}
%     \item \textbf{Technical capabilities:} Client business owners will have
%     access to various devices (desktops, laptops, tablets, smartphones) and
%     operating systems (Windows, macOS, iOS, Android). Our system must be
%     compatible with these platforms and provide a consistent user experience.
%     \item \textbf{Software capabilities:} Client business owners may use
%     different software applications (CRM, ERP, Office Suite) alongside our
%     system. We should consider integration options and data exchange formats
%     (APIs, CSV, etc.) to facilitate smooth workflows.
%     \item \textbf{Network capabilities:} Client business owners will rely on
%     internet connectivity to access our system. We must ensure that our system
%     performs well under different network conditions (Wi-Fi, mobile data).
% \end{itemize}

% \subsubsection{Employees PACT analysis}

% Lets start with analyzing the \textbf{people} aspect of PACT. 

% Now lets analyze the \textbf{activities} aspect of PACT.

% Now lets analyze the \textbf{context} aspect of PACT.

% Now lets analyze the \textbf{technologies} aspect of PACT.



% \subsection{Secondary Stakeholders PACT analysis}

% \subsubsection{Customers PACT analysis}

% Lets start with analyzing the \textbf{people} aspect of PACT. 

% Now lets analyze the \textbf{activities} aspect of PACT.

% Now lets analyze the \textbf{context} aspect of PACT.

% Now lets analyze the \textbf{technologies} aspect of PACT.




% \subsection{Primary Stakeholders Needs}


% Now that we know what are our primary and secondary stakeholders expectations,
% we can take a look into what are their needs in more detail. Specifically into
% how their needs will influence the design of our system.


% \subsubsection{Client Business Owners Needs}

% The main needs of the client business owners are to have a system where they 
% can manage their business operations daily, track sales, manage inventory,
% generate reports (taxes, profits, losses, etc.), and hire or fire employees.
% This need is driven by the fact that business owners need to have a clear overview of their
% business operations to make informed decisions and ensure the success of their
% business.



% The main needs of the client business owners are to have a system that is
% \textbf{error-free}, \textbf{reliable} and \textbf{efficient}. Business owners will need to use the system daily to manage their employees,
% track sales, manage inventory, and generate reports (taxes, profits, losses,
% etc.). Therefore the system must be error-free to ensure that critical data like
% report generation is not compromised. The system must also be reliable to ensure
% that business operations are not disrupted and that business owners can trust
% the system to perform as expected. Finally, the system must be efficient to
% ensure that business owners can quickly and easily perform their tasks without
% wasting time or effort for looking for features or navigating through complex
% menus.

% \subsubsection{Employees Needs}

% The main need of the employees is to have a system that helps them process orders,
% book appointments, and manage customer interactions. This need is driven by the
% fact that employees need to be able to perform their tasks quickly and accurately
% to ensure customer satisfaction and the smooth operation of the business.

% The main needs of the employees are to have a system that is \textbf{easy to
% learn}, \textbf{efficient} and \textbf{error-free}. Employees will need to use
% the system daily to process orders, book appointments, and manage customer
% interactions. Therefore the system must be easy to learn to ensure that
% employees can quickly adapt to the system and become proficient in its use. The
% system must also be efficient to ensure that employees can complete their tasks
% in a timely manner without unnecessary delays. Finally, the system must be
% error-free to ensure that employees can perform their tasks without costly mistakes
% or disruptions.

% \subsection{Secondary Stakeholders Needs}

% \subsubsection{Customers Needs}

% The main needs of the customers are to have a system that allows them to
% easily place orders or book appointments, view menus or services, and receive
% accurate information about the business. Moreover customers need easy to pay options. This need is driven by the fact that 
% customers expect a seamless and convenient experience when interacting with
% businesses, and a well-designed system can help meet these expectations.


% The main needs of the customers are to have a system that is \textbf{effective}, \textbf{efficient},
% \textbf{reliable} and \textbf{satisfying}. Customers will need to use the system
% occasionally to place orders or book appointments. Therefore the system must be
% effective to ensure that customers can easily find and use the features they
% need. The system must also be efficient to ensure that customers can complete
% their tasks quickly and without unnecessary delays. The system must be
% reliable to ensure that customers can trust the system to perform as expected
% and that their interactions with the business are not disrupted. Finally, the
% system must be satisfying to ensure that customers have a positive experience
% and are more likely to return to the business in the future.






\printbibliography[title = {References and sources}]

\end{document}

