%%%%%
%%%%%  Use LUALATEX, not LATEX.
%%%%%
%%%%
\documentclass[]{VUMIFTemplateClass}

\usepackage{indentfirst}
\usepackage{amsmath, amsthm, amssymb, amsfonts}
\usepackage{mathtools}
\usepackage{physics}
\usepackage{graphicx}
\usepackage{verbatim}
\usepackage[hidelinks]{hyperref}
\usepackage{color,algorithm,algorithmic}
\usepackage[nottoc]{tocbibind}
\usepackage{tocloft}

\usepackage{titlesec}
\newcommand{\sectionbreak}{\clearpage}

\makeatletter
\renewcommand{\fnum@algorithm}{\thealgorithm}
\makeatother
\renewcommand\thealgorithm{\arabic{algorithm} algorithm}

\usepackage{biblatex}
\bibliography{bibliografija}
%% to change the numbering (numeric or alphabetic) of bibliographic sources, make the change in VUMIFTemplateClass.cls, line 139

% Author's MACROS
\newcommand{\EE}{\mathbb{E}\,} % Mean
\newcommand{\ee}{{\mathrm e}}  % nice exponent
\newcommand{\RR}{\mathbb{R}}


\studyprogramme{Software engineering} %Write your study programme (example – Software engineering, Financial and Actuarial Mathematics, etc.)
% \worktype{\{Work type\}} % Bachelor's thesis or Master's thesis
\worktitle{Human computer interaction}
\secondworktitle{Žmogaus kompiuterio sąveika}
\workauthor{Tadas Riksas, Darius Spruogis, Gustas Mickus, Kajus Bicka}

%There may be more than one author, in which case each author is written from a new line which is added in Titlepage.tex or LongerTitlePage.tex
%\secondauthor{Name Surname} %If present, otherwise delete

\supervisor{Kristina Lapin}
\reviewer{Kristina Lapin} %If present, otherwise delete
% \scientificadvisor{pedagogical/scientific title Name Surname} %If present, otherwise delete

\begin{document}
\selectlanguage{english}

\onehalfspacing
\begin{titlepage}
\vskip 20pt
\begin{center}
\includegraphics[scale=0.55]{images/MIF.png}
\end{center}

\makeatletter

\vskip 20pt
\centerline{\bf \large \textbf{VILNIUS UNIVERSITY}}
\vskip 10pt
\centerline{\large \textbf{FACULTY OF MATHEMATICS AND INFORMATICS}}
\vskip 10pt
\centerline{\large \textbf{\MakeUppercase{\@studyprogramme \space study programme}}}

\vskip 80pt
\centerline{\Large \@worktype}
\vskip 20pt
\begin{center}
    {\bf \LARGE \@worktitle}
\end{center}
\begin{center}
    {\bf \Large \@secondworktitle}
\end{center}
\vskip 80pt

\centering{\Large \@workauthor}
\@ifundefined{@secondauthor}{}
{
\vskip 10pt
\centering{\Large \@secondauthor}
}
\vskip 20pt

\centering{
    \begin{tabular}{rcp{.7\textwidth}}
        {\Large Supervisor} & {\Large :} & {\Large \@supervisor}\\[10pt]
        \@ifundefined{@scientificadvisor}{}
            {
                {\Large Scientific advisor} & {\Large :} & {\Large \@scientificadvisor}\\[10pt]
            }
        \@ifundefined{@reviewer}{}
            {
                {\Large Reviewer} & {\Large :} & {\Large \@reviewer}\\[10pt]
            }
    \end{tabular}}


\vskip 110pt

\centerline{\large \textbf{Vilnius}}
\centerline{\large \textbf{\the\year{}}}

\makeatother

\newpage
\end{titlepage}
%\newgeometry{top=2cm,bottom=2cm,right=2cm,left=3cm}
\setcounter{page}{2}


\singlespacing
\selectlanguage{english}
% list of figures, delete if not needed
% \listoffigures 

%list of tables, delete if not needed
% \listoftables

%Turinys
\tableofcontents
\onehalfspacing


% %Section for abbreviations
% \sectionnonum{List of abbreviations} %Leave if necessary
% This section is for when abbreviations are used. For example:

% \begin{tabular}{rcp{.7\textwidth}}
%     {u.d.i.r.v.} & {} & {uniformly distributed independent random variables}
% \end{tabular}



\section{Introduction}
\subsection{Project Title}

\textbf{Title:} UniServe

\textbf{Explanation:} "Uni" represents \emph{universal} inidicating the platform's flexibility across a variety of business models 
(reservation based businesses such as massage services or order based businesses such as restaurants). 
"Serve" refers to \emph{service}, emphasizing the platform’s role in providing core functionalities for these types of businesses.

\subsection{Problem statement}

\subsubsection{The background}

% Decades ago, it was still common to go to restaurants and expect menus to be
% handed out on-site, or to call by phone to register an appointment. However, as
% we move into an age of ever-accelerating communication and information exchange,
% these methods have become stubborn and slow. More customers and businesses recognize a need for
% faster communication and more accurate information exchange.

% Reliance on traditional methods slows down information exchange between
% employees, introduces human error, and creates friction in customer
% interactions. This leads to employee dissatisfaction, increased workload, as
% well as a decrease in operational efficiency.

% Slower communications and unstandardized registrations or menus introduce friction
% in customer interactions, makes customers more reluctant to engage with the
% business and leads to overall decrease in customer satisfaction.


\subsubsection{Impact on the business}

Owners of small to medium-sized businesses in the restaurant and beauty
industries often cannot afford the resources to develop and maintain a
specialized web application for managing their business operations and customer
interactions. This is due to many of these businesses having limited budgets and
technical expertise, which makes custom software solutions expensive.

The people affected by this problem include customers, who increasingly expect
online services such as remote ordering or appointment booking. This challenge
is common both in urban and suburban areas where internet use is high and the
contest for the attention and loyalty of customers has become a defining force.
Without an accessible digital platform, businesses risk missing potential
customers who rely on the internet for discovering and engaging with services,
leading to reduced customer reach. 

Furthermore, business owners also face operaqtional inefficiencies and lower
profits due to the lack of automation, requiring increased employee workload,
leading to increased recurring costs. The problem occurs continuously throughout
the year, but becomes critical during peak business periods such as holidays,
weekends, or promotional events.

\section{User needs analysis}


\subsection{Identifying stakeholders}


\subsection{Expectations of the stakeholders}










% \sectionnonum{Results and conclusions}
% For details of what needs to be written in this section, please refer to the methodology requirements of the respective programme. 

\printbibliography[title = {References and sources}]

\appendix
\renewcommand{\thesection}{Appendix \arabic{section}. }

\section{\phantom{Appendix} Examples of citations}
In the document \textit{bibliography.bib}, you need to add all the cited sources and after using the function \textit{\{\textbackslash cite\{name of the cited object\}\}} the corresponding source will be added to the list of literature sources.


\textit{bibliography.bib} provides examples of some of the most commonly cited types of sources:
\begin{itemize}
    \item web pages (\textit{@online}) \cite{PvzInternetinisPuslapis},
    \item datasets (\textit{@dataset}) \cite{dataset}
    \item articles (\textit{@article}) \cite{PvzStraipsnLt, PvzStraipsnEn}, 
    \item articles from conferences (\textit{@inproceedings}) \cite{PvzKonfLt, PvzKonfEn}, 
    \item books (\textit{@book}) \cite{PvzKnygLt, PvzKnygEn}, 
    \item theses (\textit{@thesis or mastersthesis/phdthesis} \cite{PvzMagistrLt, PvzPhdEn})
    \item electronic publications (\textit{@misc}) \cite{PvzElPubLt, PvzElPubEn}
\end{itemize}

Examples are also provided for ChatGPT citation, both in general \cite{chatgpt_bendrai} and for a specific conversation \cite{chatgpt_pokalbis}.

\end{document}
