%%%%%
%%%%%  Use LUALATEX, not LATEX.
%%%%%
%%%%
\documentclass[]{VUMIFTemplateClass}

\usepackage{indentfirst}
\usepackage{amsmath, amsthm, amssymb, amsfonts}
\usepackage{mathtools}
\usepackage{physics}
\usepackage{graphicx}
\usepackage{verbatim}
\usepackage[hidelinks]{hyperref}
\usepackage{color,algorithm,algorithmic}

\usepackage{xcolor}
\usepackage{tcolorbox}

\newcommand{\yellowcomment}[1]{%
    \begin{tcolorbox}[colback=yellow!80, colframe=yellow!80, arc=0pt, outer arc=0pt, boxrule=0pt, left=3pt, right=3pt, top=3pt, bottom=3pt]
        \textbf{\textcolor{red}{COMMENT:}} #1
    \end{tcolorbox}
}

% Alternative using colorbox with parbox for inline comments
\newcommand{\inlineyellow}[1]{%
    \colorbox{yellow!80}{\parbox{\dimexpr\textwidth-2\fboxsep}{%
        \textbf{\textcolor{red}{COMMENT:}} #1%
    }}%
}

% Even more obnoxious version with red border
\newcommand{\warningcomment}[1]{%
    \begin{tcolorbox}[colback=yellow!90, colframe=red, arc=0pt, outer arc=0pt, boxrule=2pt, left=5pt, right=5pt, top=5pt, bottom=5pt]
        \Large\textbf{\textcolor{red}{FIX THIS: }} \normalsize #1
    \end{tcolorbox}
}

\newcommand{\goodcomment}[1]{%
    \begin{tcolorbox}[colback=green!20, colframe=green!60, arc=0pt, outer arc=0pt, boxrule=1pt, left=3pt, right=3pt, top=3pt, bottom=3pt]
        \textbf{\textcolor{green!70!black}{GOOD:}} #1
    \end{tcolorbox}
}

\newcommand{\noticecomment}[1]{%
    \begin{tcolorbox}[colback=blue!20, colframe=blue!60, arc=0pt, outer arc=0pt, boxrule=1pt, left=3pt, right=3pt, top=3pt, bottom=3pt]
        \textbf{\textcolor{blue!70!black}{NOTE:}} #1
    \end{tcolorbox}
}

% Inline version for positive comments
\newcommand{\inlinegreen}[1]{%
    \colorbox{green!20}{\textbf{\textcolor{green!70!black}{✓ GOOD:}} #1}%
}

\newcommand{\todocomment}[1]{%
    \begin{tcolorbox}[colback=red!20, colframe=red!60, arc=0pt, outer arc=0pt, boxrule=1pt, left=3pt, right=3pt, top=3pt, bottom=3pt]
        \textbf{\textcolor{orange!70!black}{TODO:}} #1
    \end{tcolorbox}
}

\newcommand{\suggestioncomment}[1]{%
    \definecolor{lime}{RGB}{50,205,50}%
    \begin{tcolorbox}[colback=lime!15, colframe=lime!60, arc=0pt, outer arc=0pt, boxrule=1pt, left=3pt, right=3pt, top=3pt, bottom=3pt]
        \textbf{\textcolor{lime!70!black}{SUGGESTION:}} #1
    \end{tcolorbox}%
}

\usepackage[nottoc]{tocbibind}
\usepackage{tocloft}

\usepackage{amssymb}

\usepackage{titlesec}
\newcommand{\sectionbreak}{\clearpage}

\usepackage{titlesec}

\setcounter{secnumdepth}{4}
\setcounter{tocdepth}{3}

\titleformat{\paragraph}
{\normalfont\normalsize\bfseries}{\theparagraph}{1em}{}
\titlespacing*{\paragraph}
{0pt}{3.25ex plus 1ex minus .2ex}{1.5ex plus .2ex}

% Create the custom command
\newcommand{\subsubsubsection}[1]{\paragraph{#1}}

\makeatletter
\renewcommand{\fnum@algorithm}{\thealgorithm}
\makeatother
\renewcommand\thealgorithm{\arabic{algorithm} algorithm}

\usepackage{biblatex}
\bibliography{bibliografija}
%% to change the numbering (numeric or alphabetic) of bibliographic sources, make the change in VUMIFTemplateClass.cls, line 139

% Author's MACROS
\newcommand{\EE}{\mathbb{E}\,} % Mean
\newcommand{\ee}{{\mathrm e}}  % nice exponent
\newcommand{\RR}{\mathbb{R}}




\studyprogramme{Software engineering} %Write your study 


% pilnas pavadinimas, aiskina ka visa tai daro
% prasmingas pavadinimas cia
\worktitle{\textit{UniServe}: A Universal Service Management Platform for Reservation and Order-Based Businesses} 
\workauthor{Tadas Riksas, Darius Spruogis, Gustas Mickus, Kajus Bicka}

\supervisor{Kristina Lapin}

\begin{document}
\selectlanguage{english}

\onehalfspacing
\begin{titlepage}
\vskip 20pt
\begin{center}
\includegraphics[scale=0.55]{images/MIF.png}
\end{center}

\makeatletter

\vskip 20pt
\centerline{\bf \large \textbf{VILNIUS UNIVERSITY}}
\vskip 10pt
\centerline{\large \textbf{FACULTY OF MATHEMATICS AND INFORMATICS}}
\vskip 10pt
\centerline{\large \textbf{\MakeUppercase{\@studyprogramme \space study programme}}}

\vskip 80pt
\centerline{\Large \@worktype}
\vskip 20pt
\begin{center}
    {\bf \LARGE \@worktitle}
\end{center}
\begin{center}
    {\bf \Large \@secondworktitle}
\end{center}
\vskip 80pt

\centering{\Large \@workauthor}
\@ifundefined{@secondauthor}{}
{
\vskip 10pt
\centering{\Large \@secondauthor}
}
\vskip 20pt

\centering{
    \begin{tabular}{rcp{.7\textwidth}}
        {\Large Supervisor} & {\Large :} & {\Large \@supervisor}\\[10pt]
        \@ifundefined{@scientificadvisor}{}
            {
                {\Large Scientific advisor} & {\Large :} & {\Large \@scientificadvisor}\\[10pt]
            }
        \@ifundefined{@reviewer}{}
            {
                {\Large Reviewer} & {\Large :} & {\Large \@reviewer}\\[10pt]
            }
    \end{tabular}}


\vskip 110pt

\centerline{\large \textbf{Vilnius}}
\centerline{\large \textbf{\the\year{}}}

\makeatother

\newpage
\end{titlepage}
%\newgeometry{top=2cm,bottom=2cm,right=2cm,left=3cm}
\setcounter{page}{2}


\singlespacing
\selectlanguage{english}
% list of figures, delete if not needed
% \listoffigures 

%list of tables, delete if not needed
% \listoftables

\tableofcontents
\onehalfspacing


\section*{Check List \checkmark}

\section*{Assignment Requirements}

\subsection*{Documentation Requirements}

The assignment consists of the following parts: title page, table of contents,
and the main part.

\textbf{Section numbering.} Headings of the sections must be numbered strictly
hierarchically. The sections have assigned numbers, such as 1., subsections –
1.1., 1.2, subsubsections – 1.1.1., 1.1.2. and so on. \checkmark

\textbf{File naming requirements.} A file name contains a short project title,
an assignment number and an assignment title. Example: CarBuddy 1 User needs.pdf

\subsection*{An outline of the main part}

\begin{enumerate}
    \item Introduction \checkmark
    \begin{enumerate}
        \item[1.1.] Project titles \checkmark
        \item[1.2.] Problem statement \checkmark
    \end{enumerate}
    
    \item User needs analysis
    \begin{enumerate}
        \item[2.1.] Expectations of the stakeholders \checkmark
        \item[2.2.] $\langle$The name of the first user group$\rangle$ research \{for each user group in both primary and secondary stakeholders\} \checkmark
        \begin{enumerate}
            \item[2.2.1.] Current users' activities \checkmark
            \begin{enumerate}
                \item[2.2.1.1.] $\langle$Title of the 1st user story$\rangle$ \checkmark
                \begin{itemize}
                    \item Description of current activity (unnumbered subsection) \checkmark
                    \item Problems and opportunities (unnumbered subsection) \checkmark
                \end{itemize}
                \item[2.2.1.2.] $\langle$Title of the 2nd user story$\rangle$ \checkmark
                \begin{itemize}
                    \item \ldots \checkmark
                \end{itemize}
            \end{enumerate}
            \item[2.2.2.] Characteristics of the people, activities, context, and technologies \checkmark
            \item[2.2.3.] Needs
            \item[2.2.4.] Usability objectives
        \end{enumerate}
        \item[2.3.] $\langle$The name of the next user group$\rangle$ needs analysis \{if needed\}
        \item[] \ldots
        \item[2.$\langle$n+1$\rangle$.] Inspiring user interface designs
    \end{enumerate}
\end{enumerate}

\subsection*{Explanations}

\textbf{The title page} contains:
\begin{itemize}
    \item university name, faculty, and study program, \checkmark
    \item project full title, \checkmark
    \item contributors' names and surnames, \checkmark
    \item year. \checkmark
\end{itemize}

\textbf{The table of contents} includes the sections and subsections until the 3rd level and page numbers. \checkmark

\textbf{Project titles.} The project must have two titles: \checkmark
\begin{itemize} 
    \item full (e.g. ``Vilnius University Library App'') and \checkmark
    \item short (e.g. ``Library''). \checkmark
\end{itemize}

The full title is used on the title page and in this section, a short title serves as a project reference in the document text and assignment file names. \checkmark

\textbf{Problem statement} explains the high-level project goals shortly and
specifically (about half a page). The problem is described in terms of user
activities and situations where the problem occurs, and what aspects might be
improved with a technical solution. Avoid describing or suggesting a solution at
this stage that will hamper your design thinking when you start solving the
problem. \checkmark

To gather the relevant facts for your problem statement, you can use a simple
technique called the 6 Ws, which involves answering the questions below: \checkmark
\begin{itemize}
    \item Who is affected by the problem? \checkmark
    \item What is the problem? \checkmark
    \item Where does this problem occur? \checkmark
    \item When does the problem occur? \checkmark
    \item Why does the problem occur? \checkmark
    \item Why is the problem important? \checkmark
\end{itemize}

\textbf{Expectations of the stakeholders} \checkmark

Specify the specific groups of people (stakeholders) involved in activities that
will be supported by your project (see Lecture 1. User-centered design).
Identify relevant stakeholders' intentions that are supposed to be fulfilled. In
the competitors subsection, you should list your closest competitors. Then you
should list your main competitive advantages over them. \checkmark

\textbf{Current user activities} \checkmark

Watching how people do things is a great way to learn their goals and values,
and come up with design insight. This is called user research. This assignment
helps you train your eyes and ears to develop design ideas. Your goal is to
uncover user needs, breakdowns, clever hacks, and opportunities for improvement.
Begin by selecting a specific activity to observe. \checkmark

The description contains the user story that tells about the current user
activities. Note that your service or product does not exist yet. Therefore, you
must describe the existing activities, such as how your users achieve their
goals with existing competitive products, and highlight the pain points and
troubles. Then, indicate the problems and improvement opportunities (see the
assignment outline). \checkmark

Describe at least 3 observations of current user activities for primary users,
for the secondary – there can be fewer. Examples of activity descriptions: \checkmark

\begin{itemize}
    \item \textbf{Story 1:} Ann is a clerk in her twenties, comes for a quick
    lunch. She wants to see dishes that are served within 15 min. She waits 10
    min. for a waiter. The waiter says that only soup can be served in 15 min.
    Therefore Ann leaves the restaurant and goes for a fast food nearby. \checkmark
    
    \item \textbf{Story 2:} Andrew, a thirty-two-year-old restaurant client, who
    suffers from allergies, comes to a restaurant. He examines the menu. The
    menu is long, ingredients are mentioned in every dish. So, it takes about 5
    minutes to choose a dish. Waiting for the dish takes 20 min. Andrew must
    hurry and eats quickly because he has to return to work on time. \checkmark
    
    \item \textbf{Story 3:} Thirty-five-years-old Peter goes to a restaurant
    with his family. He noticed on Google Maps that a new restaurant was opening
    not far away. They spend time reading the menu. The café has a children's
    menu, but children's favorite dishes are missing. Children are bored because
    of missing the game area. The family chooses quick dishes from the menu. \checkmark
    
    \item \textbf{Story 4:} Fourty-year-old Sara comes with her friends to a
    restaurant on Sunday. They are looking for a big table but see that the
    restaurant is almost full. They started to explore the TripAdvisor app on
    their phone for other restaurants phone numbers to ask whether they have a
    free big table for a group. \ldots \checkmark
\end{itemize}

\textbf{How to check whether a story is correct?} \checkmark

A story should cover
characteristics of peoples, activities, environment and technologies that are
related to computerized activities. If any characteristic is missing, you should
augment the story. \checkmark

\textbf{Needs statements and usability objectives} 
 
After describing user stories, go over your findings. Brainstorm about both user
needs and usability objectives. Observe the opportunities for design innovation
that would improve the activity. User needs are formulated for primary and
secondary users, only. Indirect stakeholders formulate business goals that are
abstract and relate to business strategy. Business goals should be met with the
needs of primary and secondary users. The impact of business goals on user needs
is explained in Lecture 2, slides 52–53. Examples of usability goals and needs
are provided in slides 54--57, and usability objectives are tackled in slides
58--60. 

The project must contain at least 15 user needs. 3–5 usability objectives are
defined for every user group. You are not looking for solutions yet: focus on
user needs only. User needs and usability objectives must have unique
identifiers. Below the needs and the usability objectives provide traceability
matrices that link the need or objective identifier with the number of paragraph
of the user story from which it is derived.

\textbf{Inspiring user interface designs}

Provide here examples of successful user interface solutions of existing
software that can be useful for specified user needs and usability objectives.
Provide a screenshot of the design, the number of user needs, or the usability
objective for which this solution would be beneficial. Images should have
captions with concise explanations and references to the source, from which this
image is taken. The text related to the image should contain cross-reference to
the image. Provide at least 5 examples.

\section*{Other}

\subsubsection*{Use of AI}

Just a little note: dont just copy the assignment text and expect good output,
look into slides also and try to apply lecture material. 

\subsubsection*{Comments}

\warningcomment{This is a very important comment}
\yellowcomment{This is a comment}
\todocomment{This is a todo comment}
\noticecomment{This is a notice comment}
\goodcomment{This is a good comment}
\suggestioncomment{This is a suggestion comment}

Usage of comments in text is recommended as it helps to highlight problems and
solutions. The Overleaf comments are often ignored, therefore, it is recommended
to use the comments defined in this template.

\section{Introduction}
\subsection{Project title}

\textbf{Full title:} \textit{UniServe: A Universal Service Management Platform for Reservation and Order-Based Businesses}

\textbf{Short title:} \textit{UniServe}


\textbf{Explanation:} "Uni" represents \emph{universal}, indicating the platform's flexibility across a variety of business models 
(reservation based businesses such as massage services or order based businesses such as restaurants). 
"Serve" refers to \emph{service}, emphasizing the platform’s role in providing core functionalities for these types of businesses.

\subsection{Problem statement}

Traditional methods of managing reservations and orders—such as handwritten
appointment books, paper-based inventory tracking, manual payment processing,
and physical order documentation—create significant operational bottlenecks that
directly impact business performance and stakeholder satisfaction. Reliance on
outdated manual processes for tracking orders, services, inventory, payments,
and taxes create several critical problems:

\begin{itemize}
    \item \textbf{For employees:} Manual order-taking and appointment booking
    requires constant \textbf{memorization} of availability, pricing, and menu
    details. Staff must physically check paper schedules or call to verify
    bookings, leading to \textbf{errors}, double-bookings, and frustrated
    customers. During peak periods (weekends, holidays, promotional events),
    employees become \textbf{overwhelmed} managing multiple phone lines while
    \textbf{simultaneously serving} in-person customers, \textbf{calculating
    totals}, resulting in poor service quality and increased stress.

    \item \textbf{For customers:} Manual paper-based systems create
    \textbf{longer wait times} as employees must physically write orders, search
    through paper schedules, and handle multiple forms or cards during each
    transaction. Customers experience \textbf{increased likelihood} of lost or
    incorrect orders due to handwriting errors, misplaced papers, or
    miscommunication between staff members. The \textbf{slower service} caused
    by employees manually searching orders, calculating totals, processing
    appointments leads to customer \textbf{frustration} and
    \textbf{dissatisfaction}, especially during busy periods when paper-based
    workflows create bottlenecks.
    % \todocomment{ask PS design lecturer if users should be able to view that information or if the interface remains the same and simply the employees have access to that information}

    \item \textbf{For business owners:} Manual systems prevent access to crucial
    business data—owners \textbf{cannot easily} track popular items, peak hours, or
    customer patterns. Inventory management relies on \textbf{guesswork}, leading to
    waste or stockouts. \textbf{Tax management} becomes complicated and
    error-prone when relying on scattered paper receipts and handwritten
    records. Without automated processes, businesses require more
    staff hours for administrative tasks that could be automated, increasing
    labor costs while \textbf{limiting growth potential}.
\end{itemize}

\textbf{Technical barriers:}

Limited budgets and lack of technical expertise prevent these businesses from
developing in-house solutions as these solutions are often expensive and complex.

\textbf{Why this problem matters and consequences of inaction:}

These manual processes directly limit how much revenue a business can generate.
When employees spend time on paperwork instead of serving customers, the
business serves fewer people and makes less money. The problem gets worse
during busy periods when potential customers leave due to long wait times,
representing direct lost sales. Over time, businesses using manual systems
fall further behind competitors who adopted digital tools—they cannot match
their speed, cannot reduce their costs, and cannot grow at the same rate.
Without change, the gap widens: competitors continue improving their
efficiency while manual-dependent businesses remain stuck with the same
limitations, eventually losing market share and facing reduced profitability
that threatens long-term viability.


\section{Stakeholder analysis}

\subsection{Identifying stakeholders}

\begin{table}[h]
  \centering
  \caption{Stakeholders identification}
  \begin{tabular}{|c|c|}
    \hline
    Stakeholder type    & Stakeholders \\ \hline
    Primary             & Employees, customers \\ \hline
    Secondary           & Client business owners \\ \hline
    Indirect (tertiary) & App business owners \\ \hline
    External/competitors & Square, Toast POS, Restaurant365 \\ \hline
    Technical support   & Sysadmins, tech support \\ \hline
  \end{tabular}
  \label{tab:stakeholders}
\end{table}

% Nielsen‘s principles
% 1. Learnability
% 2. Efficiency of use
% 3. Memorability
% 4. Few and non-
% catastrophic errors
% 5. Satisfaction

\textbf{Primary stakeholders:}
\begin{itemize}
    \item \textbf{Employees} expect a system that eliminates the need to
    memorize constantly changing information (availability, pricing, specials),
    allows calculating totals efficiently, reduces the risk of errors in
    bookings and order processing, and allows for a more satisfying and
    efficient customer serving experience.
    % \item \textcolor{gray}{\textbf{Employees} expect a system that removes the
    % need to memorize, provides quick access to changing information (item
    % availability, pricing, specials), automates calculations (discounts, split
    % bills, etc.), validates orders and bookings.}
    
    % \warningcomment{from lecture: "So, you don't need to think of self-service,
    % maybe like Burger King or McDonald's", "...you can assume that all business
    % operations are performed by business employees. So, you don't need to worry
    % that much about customer accounts..."}

\end{itemize}

\textbf{Secondary stakeholders:}
\begin{itemize}
    % o sitie primary? jie interactina su musu sistemos UI???? Jie interactina, bet pagal primary users apibrezima jie turetu intereactinti frequently, o owners ocasionally - tai secondary turetu buti manau
    \item \textbf{Client business owners} expect automated access to business
    data (sales patterns, inventory levels, customer analytics, tax information)
    that are error-free, reduced administrative overhead. They expect efficient
    systems that will reduce labor and allow them to focus on strategic
    decision-making and preserving historical records—to help identify trends,
    and boost overall profits.
    % not sure ar cia geriau, bet bandau kazkaip labiau specific kelis ir pateikti examples
    % \item \textcolor{gray}{\textbf{Client business owners} expect a system that
    % accelerates employee-managed orders and reservations, for example through
    % automatic validation or menu management (e.g., limiting options when items
    % are out of stock). They also expect actionable data—such as tracking sales,
    % payments, discounts, tips, and preserving historical records—to help
    % identify trends, improve inventory management, and boost overall profits.}
    
    \item \textbf{Customers} expect satisfying and efficient service with minimal wait
    times during ordering and booking processes. They expect their orders and
    appointments to be accurately recorded without errors or confusion.
    Customers want reliable service where their bookings won't be lost
    or double-booked due to manual record-keeping mistakes.
    % \item \textcolor{gray}{\textbf{Customers} expect quick ordering and booking
    % with accurate records, and no inconveniences like being told later that an
    % item is unavailable or an appointment was entered incorrectly.}

\end{itemize}

\textbf{Indirect (tertiary) stakeholders:}
\begin{itemize}
    \item \textbf{App business owners} recognize a market opportunity to serve
    underserved small and medium businesses that cannot afford or implement
    complex enterprise solutions. They want steady income growth by capturing
    market share from businesses currently using manual processes or considering
    expensive alternatives.
    % \item \textcolor{gray}{\textbf{App business owners} expect to capture market
    % share from small and medium businesses using manual tools (e.g., Excel) or
    % unable to use more advanced and expensive solutions (e.g., Toast POS). They
    % expect steady income growth by attracting these customers.}


\end{itemize}


\textbf{Competitors:}

\begin{itemize}
    \item Square
\begin{itemize}
        \item \textbf{COMPETITOR ANALYSIS:}
        
        Square offers a free basic POS system with no monthly fees, appealing to
        small businesses starting out. However, Square charges per-transaction
        fees that accumulate for high-volume businesses, and their cloud-based
        system can experience outages that disrupt operations when connectivity
        issues occur. Users report that the interface requires significant
        training time and that some UI updates have made workflows less
        intuitive for employees.

        \item {\textbf{OUR ADVANTAGES:} 
        
        Our flat-rate pricing model provides cost predictability for growing
        businesses without per-transaction fees. The system functions offline,
        ensuring employees can continue serving customers even during network
        issues. Our interface is designed around common employee workflows,
        reducing training time and cognitive load during busy service periods.}

    \end{itemize}
    

    \item {Toast POS}
    \begin{itemize}
        \item  \textbf{COMPETITOR ANALYSIS:}
        Toast is a restaurant-specific system with strong
        features for food service operations. However, Toast requires significant
        upfront investment in proprietary hardware (costing \$494-\$1,034) and monthly
        software fees starting at \$69 for advanced features. This pricing model
        puts it out of reach for many small restaurants operating on tight margins.
        Toast also requires technical staff to setup.

        \item  \textbf{OUR ADVANTAGES:} 
        We offer lower upfront costs by working with
        standard devices businesses may already own. Our simplified interface
        requires minimal training, allowing staff to start using the system
        immediately without dedicated technical support or lengthy onboarding
        processes.
    \end{itemize}
    
    \item Restaurant365
    \begin{itemize}
        \item \textbf{COMPETITOR ANALYSIS:}
        Restaurant365 provides comprehensive
        accounting and operations management with deep financial reporting
        capabilities. However, it starts at $90-$249 per location per month, making
        it expensive for single-location small businesses. The system is built for
        restaurant groups managing multiple locations and complex accounting needs,
        offering far more features than small businesses require. This complexity
        makes it difficult to learn and overkill for businesses that simply need
        basic order management and inventory tracking.

        \item \textbf{OUR ADVANTAGES:} 
        We focus on core operational needs—order taking,
        reservations, basic inventory—without expensive accounting modules that
        small businesses do not need. Our pricing is accessible for single-location
        businesses, and our simplified feature set means owners can understand and
        use the system without accounting expertise.
    \end{itemize}
    

    \item Excel: 
    
    \begin{itemize}
        \item \textbf{COMPETITOR ANALYSIS:}
        Many small businesses currently use Excel spreadsheets
        to manually track orders, inventory, and schedules. Excel is familiar and
        inexpensive, which explains its continued use. However, Excel requires
        businesses to build their own systems from scratch, offers no automation for
        repetitive tasks, and provides no real-time updates when multiple staff
        members need access. Data entry errors are common because everything must be
        typed manually, and there is no way to connect Excel directly to payment
        processing or customer-facing ordering systems.
        
        \item \textbf{OUR ADVANTAGES:} 
        We provide a ready-to-use system with built-in
        templates for common business operations, eliminating the need to build
        tracking systems from scratch. Our system automates repetitive tasks like
        inventory calculations and sales reporting, reduces data entry errors
        through dropdown menus and automated calculations, and allows multiple staff
        members to access updated information in real-time.
        \end{itemize}
        
\end{itemize}

\textbf{Development and maintenance team:}
\begin{itemize}
    \item \textbf{Sysadmins} expect reliable system architecture that can handle
    varying loads (especially during peak business periods), scalable
    infrastructure to accommodate business growth, and clear deployment
    procedures to maintain system stability.
    \item \textbf{Tech support} expect comprehensive documentation and effective
    troubleshooting tools to efficiently resolve the types of issues small
    business users commonly encounter, along with manageable processes for user
    support that don't overwhelm limited support resources.
\end{itemize}


% <The name of the first user group> research {for each user group in both primary and secondary stakeholders}

% should include people characters, activities, enviroment and technologies or
% in other words it should give info to PACT analysis.          
% Character - who are the users, what are their physical and cognitive abilities,
% Activities - what tasks do they perform, what are their goals, describe the characteristics of tasks
% Context - where and when do they perform these tasks, what are the physical, social, cultural, organizational aspects of the environment
% Technologies - what tools, devices, software do they use to perform their tasks.


\subsection{Employee research}
% some teams have 5 stories, maybe we need more?
\noticecomment{Mention current competitors in the stories (not all stories, just some)}

\subsubsection{Current users' activities}

\subsubsubsection{Café Barista Managing Orders}

Marius, 21, works as a barista in a busy café. He takes orders at the counter
and writes them on sticky notes for the kitchen. During the morning rush, notes
sometimes fall to the floor or get smudged by spilled drinks. This leads to
wrong orders being prepared, which frustrates customers and wastes ingredients.
Marius tries to keep the counter tidy, but the lack of a structured system makes
mistakes inevitable and thus Marius spends extra time apologizing and remaking
orders, which slows down the queue.

% \yellowcomment{identify problems and opportunities, identify the problem in the story, go immediately to the essence of the problem}

\textbf{PROBLEMS:}

\begin{itemize}
    \item Lost orders due to sticky notes falling or getting smudged
    \item Time and inventory assets wasted on remaking incorrect orders
\end{itemize}

\textbf{OPPORTUNITIES:}

\begin{itemize}
    \item Implement a digital order management system to reduce reliance on sticky notes and minimize lost orders
\end{itemize}

% add "Description of current activity: " to the story
% seperate problems and opportunities in seperate lines.
% problems and opportunities have to be closelly related to the story.
% problems and oppurtunities can be seperated in bullet points.

% if you write oppurtunities make sure to highlight what that oppurtunity gives
% to the user, what problem it solves.

\subsubsubsection{Salon Receptionist Controlling Bookings}

Lucille, in her twenties, is the receptionist at a busy, well‑known beauty
salon. Throughout the day she fields a constant stream of phone calls to arrange
appointments. Each booking is first written by hand in a paper diary, then later
re‑entered into a spreadsheet. On busy days, at least one record is often missed
or entered incorrectly, resulting in double bookings or missed appointments.
When a client calls to reschedule, Lucille leafs through pages of the diary to
locate the original booking. This takes time, breaks her concentration, and can
leave callers waiting on the line. The salon offers no online booking system for employees,
so all scheduling relies entirely on Lucille being able to manage the diary
accurately.

\textbf{PROBLEMS:}

\begin{itemize}
    \item High risk of double bookings or missed appointments due to manual entry errors.
    \item Time wasted searching for original bookings in the paper diary.
\end{itemize}

\textbf{OPPORTUNITIES:}
\begin{itemize}
    \item Implement a digital booking system to reduce manual entry errors and
    makes searching for bookings efficient.
\end{itemize}
% \todocomment{if we say that ppl write in excel we need to specify that as a competitor}


\subsubsubsection{Restaurant Manager Handling Discounts}


Youmna, in her early twenties, manages a vibrant Lebanese restaurant on a busy
high street. To attract customers during quieter weekday afternoons, the
restaurant runs a “Taste of Lebanon” promotion, offering discounts on selected
platters and traditional dishes. When diners ask about the offer, Youmna or the
waiting staff consult a printed list of eligible dishes and manually work out
the reduced price at the till. At busy times, this slows service. Occasionally,
the discount is forgotten, prompting awkward conversations and amended bills; at
other times, it is applied in error, reducing profit margins. To monitor the
promotion, Youmna records discounted orders in a small ledger. However, during
the rush, staff often forget to make entries, leaving the records incomplete.
This makes it difficult to assess the effectiveness of the offer or refine it
for future campaigns.

\textbf{PROBLEMS:}

\begin{itemize}
    \item High risk of incorrect discounts being applied due to manual calculations.
    \item Time wasted searching for eligible dishes on printed lists.
    \item Incomplete records of discounted orders make it hard to assess promotion effectiveness.
\end{itemize}
\textbf{OPPORTUNITIES:}
\begin{itemize}
    \item Use a digital system that tracks what discounts are available and
    automatically applies them at the till to reduce errors and speed up service.
\end{itemize}

\subsubsection{PACT analysis of employees}
    \textbf{People:} Employees are mostly young to mid-career workers (baristas, receptionists, restaurant managers). They multitask under pressure, balancing customer interaction with operational tasks. Some are comfortable with smartphones and PCs. while other rely heavily on paper-based systems, showing varied levels of technical familiarity. They are motivated by efficiency: fewer errors and faster service lead to smoother operations and higher customer satisfaction. There are three types of employees: frontline employees (baristas, receptionists) who are constantly interacting with customers can often get frustrated dealing with plenty of paper orders that get lost or destroyed. Supervisors/managers oversee staff, they need better technology to make calculated decisions regarding discounts, handle exceptions etc. Support staff (kitchen staff, stylists) depend on accurate and timely information passed from the frontline, errors directly affect they output.

    \noticecomment{Perhaps add people of different ages. At the moment all of the employees are in their twenties}
    

    \textbf{Activities:} Key activities are taking and managing orders or 
booking and rescheduling appointments, all to make business as profitable as possible. It is necessary to communicate information to customers quickly, otherwise, employees have
to resolve problems due to inadvertence. Regardless of their ranking, all employees try to do they job as efficiently as possible, using the tools they have. Activities require a high cognitive load due to memorizing availability, prices, etc. In this field, problem solving is also one of the key characteristics of employees as they encounter a plethora of different situations every day.

    
    \textbf{Context:} Employees work in a fast-paced, noisy and time-sensitive
    environments, where they have constant interactions with both in-person and
    remote customers. Errors cascade into customer
    dissatisfaction which consume extra time and increase stress.

    \todocomment{employees work in different environments, not live in them}
    \todocomment{rush hours are activity characteristics, not context}
    
    \textbf{Technologies:} Currently employees rely mostly on paper (sticky
    notes, diaries, printed lists). Point of Sale systems may exist, but are
    underutilized or lack integration of bookings/promotions.  
    \yellowcomment{POS systems do exist, make sure to include our competitors in some stories.}
    \todocomment{write scenarios of these POS systems}
\subsubsection{Needs of employees}
Employees in customer-facing service roles need a centralized system to manage information and transactions efficiently using android tablet. 
%because constantly memorizing and manually updating details leads to errors, slower service, and lower customer satisfaction.
\subsubsection{Usability objectives for employees}
\begin{itemize}
    \item[OBJ-01] Employees should complete tasks (e.g., entering a booking, processing an order) in less time compared to the old method.
    \item[OBJ-02] The system prevents employee errors (such as duplicate bookings or recording orders consisting of unavailable menu items) and provides clear messages.
    \item[OBJ-03] Orders and bookings are recorded and processed with greater than 95\% accuracy.
    \item[OBJ-04] Employees rate their experience with the system as more enjoyable and less frustrating compared to previous methods.
\end{itemize}

\subsection{Customer research (Secondary User Group)}
\subsubsection{Current users' activities}

\subsubsubsection{Spa Client Booking an Appointment}
Ann, a student in her twenties, calls to book a spa appointment. The employee
promises to send a confirmation by SMS, but since it’s done manually, the
message does not arrive. Ann, unsure if the booking is valid, decides to go to
the spa anyway. Upon arrival, she finds the spa busy and has to wait. The
receptionist, unable to quickly verify her booking due to the manual system,
asks Ann to wait while they check. After a long wait, Ann is told that she is
late and must reschedule. Frustrated by the confusion and wasted trip, Ann
decides to try a different spa next time.

\textbf{PROBLEMS:}
\begin{itemize}
    \item Manual booking confirmation leads to missed appointments.
    \item Inability to quickly verify bookings causes customer frustration.
    \item Lack of efficient communication results in wasted time for both customers and staff.
\end{itemize}
\textbf{OPPORTUNITIES:}
\begin{itemize}
    \item Implement an automated SMS confirmation system to ensure customers receive timely booking confirmations.
    \item Provide staff with a digital booking system for quick verification of appointments.
\end{itemize}

\subsubsubsection{Looking for a quick lunch}

Annie, a 20‑year‑old university student, has a short break between lectures and
only 30 minutes to eat. She visits a small café she has not tried before, hoping
to find something quick. When she asks the server which dishes can be prepared
quickly, the server is unsure and has to go to the kitchen to check with the
chef. After waiting nearly 10 minutes, the server returns and tells her that
only soup can be served within her timeframe, as the server had no quick way to
access preparation time information. With little time left and disappointed by
the limited options, Annie leaves and buys a takeaway wrap from a nearby chain
instead.

\textbf{PROBLEMS:}
\begin{itemize}
    \item Servers lack quick access to preparation time information, forcing them to leave customers waiting while checking with kitchen staff.
    \item Lost business opportunity as customers leave due to long wait times and limited options.
    \item Poor customer experience resulting in loss of potential repeat customers.
\end{itemize}
\textbf{OPPORTUNITIES:}
\begin{itemize}
    \item Provide servers with a digital system showing estimated preparation times for each menu item, enabling immediate responses to customer inquiries.
\end{itemize}

\newpage

\subsubsubsection{Waiting for a haircut}

Felix, in his late forties, needs a haircut before a family gathering. He calls
his usual barber, but the line is engaged because the receptionist is handling
another customer while simultaneously trying to answer phones. After several
attempts, he finally gets through to the receptionist, who must manually flip
through a paper diary to check availability. While the receptionist searches
through multiple pages, Felix waits on the line for over three minutes. When
the receptionist finally finds his preferred time slot, she realizes there is
conflicting handwriting from another staff member and has to verify the booking
by checking a second notebook. The confusion continues as she reads out
alternative appointments, but Felix grows frustrated by the long wait and
uncertainty caused by the disorganized paper system. He decides to book with
another salon that can fit him in immediately.

\textbf{PROBLEMS:}
\begin{itemize}
    \item Manual paper diary system causes long wait times for customers trying to book appointments.
    \item Conflicting handwritten entries lead to confusion and require verification from multiple sources.
    \item Receptionist cannot multitask effectively, resulting in missed calls and lost business.
    \item Customer frustration leads to choosing competitors with more efficient booking systems.
\end{itemize}
\textbf{OPPORTUNITIES:}
\begin{itemize}
    \item Implement a digital booking system that allows receptionists to instantly check and confirm availability.
    \item Enable staff to view and update bookings in real-time, eliminating conflicting entries.
    \item Reduce phone handling time, by using quick access to information, allowing receptionists to serve more customers efficiently.
\end{itemize}

\subsubsubsection{Order with multiple variations}

Lucas, a 26-year-old customer, places an order for a smoothie with multiple
variations at a café. The barista manually writes down the order without any
system validation for the variations. Later, the staff member preparing the
drinks notices conflicts in the requested options and asks Lucas to choose
different variations. This error reduces his trust in the staff’s ability to
process customized orders accurately and reliably.
\todocomment{the employees UI should somehow influence the customers experience}

\textbf{PROBLEMS:}
\begin{itemize}
    \item Manual order entry leads to errors in processing customized orders.
    \item Lack of system validation for order variations causes confusion and delays.
    \item Customer trust is diminished due to systematic problems in handling complex orders.
\end{itemize}
\textbf{OPPORTUNITIES:}
\begin{itemize}
    \item Implement a digital order system that validates customization options in real-time.
    \item Provide staff with clear prompts and confirmations to ensure accurate order processing.
    \item Enhance customer trust by demonstrating reliable handling of complex orders.
\end{itemize}


\subsubsubsection{Redeeming a coffee promotion}


Sarah‑Louise, 27, spots a “Buy One, Get One Free” coffee offer on a café’s
social media page. She visits the café to take advantage of the deal, but when
she mentions the promotion at the counter, the barista is unaware of it and has
to leave the register to find the manager for confirmation. Sarah‑Louise waits
awkwardly as a queue forms behind her. After several minutes, the manager
confirms the promotion and the barista manually recalculates the bill. The
discount is eventually applied, but Sarah‑Louise leaves feeling embarrassed by
the confusion and delay her order caused.

\textbf{PROBLEMS:}
\begin{itemize}
    \item Staff unaware of current promotions leads to delays and customer embarrassment.
    \item Manual recalculation of bills causes longer wait times and disrupts service flow.
    \item Lack of real-time promotion updates results in missed opportunities for customers to take advantage of deals.
\end{itemize}
\textbf{OPPORTUNITIES:}
\begin{itemize}
    \item Implement a digital system that automatically updates staff on current promotions.
    \item Enable automatic application of discounts at the point of sale to streamline billing.
    \item Provide staff with quick access to promotion details to improve customer service.
\end{itemize}

\subsubsection{PACT analysis of customers}
\todocomment{PACT is not very important for customers, they are not using the system as much as employees, so we can be more general here}
\textbf{People:} There is a variety of different-aged customers with diverse
digital literacy: younger users expect seamless mobile-first interactions, while
older customers prefer phone bookings or real-life interactions. 
\yellowcomment{not enough characteristics of people described here, specify how fast they make decisions, do they have any disabilities, etc. }

\textbf{Activities:} Customers order food and drinks or book appointments, often
trying to fit them to their daily schedule. Customers choose their options by
checking the promotions and offers, comparing availability, prices, and service
quality. They are the ones interacting with the staff when digital systems are
unavailable or unclear.

\yellowcomment{no characteristics of activities described here}

\textbf{Context:} Customers are in on-the-go environments, frequently
interacting with the service during lunch breaks or between classes. They often
have to stand in queues, call during office hours and wait for their
orders/services. Customers' expectations are shaped by competitors and broader
digital experiences (food delivery apps, online booking platforms).

\noticecomment{on-the-go environments - specify it bit more? office environment, is it noisy, bright?}

\textbf{Technologies:} There is a heavy reliance on smartphones and internet
access for convenience, though currently limited to phone calls, in-person
ordering, or static media posts. There are still no centralized, real-time
booking/order management systems for many small businesses. Customers expect
advanced digital systems with real-time availability updates, seamless discount
application and other features missing in manual systems.

\subsubsection{Needs of customers}
Customers need automated and reliable communication and service processes, when interacting with employees, so they can avoid delays, repeated confirmations, and confusion.
%Customers need fast and reliable ordering and booking systems when placing orders or making reservations, so that they can avoid delays, mistakes, and lost records.

\subsubsection{Usability objectives for customers}
\begin{itemize}
    \item[OBJ-05] Customers receive SMS confirmations without delays.
    \item[OBJ-06] Customers are not delayed or inconvenienced by manual staff processes.
    \item[OBJ-07] Customers can trust employees to perform their tasks accurately and reliably.
\end{itemize}

\subsection{Client Business Owners Research (Secondary User Group)}

\subsubsection{Current users' activities}
\suggestioncomment{add more stories? it feels like we cannot have a deep analysis with just a few ones}

\subsubsubsection{Restaurant Owner Handling Customer Feedback}

Apolline, 39, owns a family restaurant in a popular tourist area. Customers
occasionally leave reviews on Google Maps or TripAdvisor, but she only checks
them sporadically. Complaints (such as “long wait for a table” or “incorrect
order”) can go unnoticed for several days. Apolline also keeps a suggestion box
in the restaurant, though very few customers make use of it. She would prefer to
gather feedback in a more systematic way and respond swiftly in order to improve
the service. Without structured data, it is difficult for her to identify
recurring issues (for example, slow service during specific time frames) and to
train staff appropriately.

\todocomment{specify that the owner is looking at the reviews after the first sentence}

\subsubsubsection{Salon Owner Tracking Staff Performance}

Nils, in his forties, runs a busy salon with six stylists. He tries to track
performance and service quality, but still relies on a paper diary and casual
feedback. Calculating wages and commissions means hours spent combing through
notes and receipts. When figures are challenged, he has no firm record to refer
to, which not only heightens administrative burden but also sows discontent
amongst the team.
\todocomment{specify how he inputs the data}

\subsubsection{PACT analysis of Client Business Owners}
    \textbf{People:} These are small to medium business owners, typically 30-50
    years old, balancing managerial, operational, and customer service roles.
    Although they have limited time and often technical expertise, business
    owners are motivated by improving service quality, staying competitive, and
    reducing costs.

    \todocomment{as an owner he can be more older}

    \textbf{Activities:} Main activities consist of tracking staff performance
    and calculating wages/commissions. Business owners also have to monitor and
    respond to customer feedback, and manage business performance indicators
    (sales, promotions, peak periods). The biggest hardships are making
    strategic decisions based on incomplete/fragmented information and handling
    administrative tasks manually without advanced technology.
    \todocomment{which ones are common, which ones are rare, atomic, seperated into steps, etc.}
    \noticecomment{Examples of characteristics of activities: repetitive, concentration required, error-prone, separated into steps, etc.????}

    \textbf{Context:} Business owners work in competitive environments, where
    work often blends strategic planning with hands-on problem solving.
    Information is spread across different informal channels, and the lack of
    structured data makes it difficult to spot trends or train staff
    effectively. This work also comes with social pressure from customers'
    expectations of professionalism and constant improvement.
    \todocomment{physical environment, sitting near PC, where does he work, he uses in the technologies spreadsheets, so he must be sitting somewhere.}
    \todocomment{social context - does he work alone, with employees, with customers?}
    \todocomment{organizational context - is he the only owner, does he have partners, does he have employees?}
    \noticecomment{organizational context - different permissions or same? Doesnt apply here, i think.}

    \textbf{Technologies:} Currently business owners rely on paper diaries,
    manual notes and spreadsheets. They are using external platforms such as
    Google Maps, and TripAdvisor for customer feedback. However, due to budget
    or complexity barriers, there is a limited exposure to advanced digital
    solutions therefore, there are no dedicated analytics dashboards or
    automated tracking systems.

    \todocomment{connectivity of internet not very important}
    \todocomment{POS concurents could be mentioned here, what is he using currently now?}

\subsubsection{Needs of Client Business Owners}

\subsubsection{Usability objectives for Client Business Owners}

\subsection{Inspiring user interface designs}


Taken from:
\begin{itemize}
     \item \href{https://www.restaurant365.com/resource-category/videos/}{Restaurant365 — Videos}
     \item \href{https://squareup.com/help/us/en}{Square — Help}
     \item \href{https://support.toasttab.com/search/#f-100=Account%20Settings%20%26%20Billing,Marketing,Menu%20%26%20Items,POS%20%26%20Location%20Operations,Takeout%20%26%20Catering&f-@commonsource=Videos&f-@language=English}{Toast — Support search}
\end{itemize}

\todocomment{choose ones that are the best, not all of them}

\begin{figure}[h]
    \centering
    \begin{minipage}{0.48\textwidth}
        \centering
        \includegraphics[width=\textwidth]{images/examples/applicants_r365.png}
        \caption{Applicants management interface (Restaurant365)}
    \end{minipage}
    \hfill
    \begin{minipage}{0.48\textwidth}
        \centering
        \includegraphics[width=\textwidth]{images/examples/hiring_status_r365.png}
        \caption{Hiring status overview (Restaurant365)}
    \end{minipage}
\end{figure}

\begin{figure}[h]
    \centering
    \begin{minipage}{0.48\textwidth}
        \centering
        \includegraphics[width=\textwidth]{images/examples/appointment_creating_square.png}
        \caption{Appointment creation interface (Square)}
    \end{minipage}
    \hfill
    \begin{minipage}{0.48\textwidth}
        \centering
        \includegraphics[width=\textwidth]{images/examples/homapage_toast.png}
        \caption{Homepage overview (Toast)}
    \end{minipage}
\end{figure}

\begin{figure}[h]
    \centering
    \begin{minipage}{0.48\textwidth}
        \centering
        \includegraphics[width=\textwidth]{images/examples/appointments_square.png}
        \caption{Appointments list (Square)}
    \end{minipage}
    \hfill
    \begin{minipage}{0.48\textwidth}
        \centering
        \includegraphics[width=\textwidth]{images/examples/homepage_r365.png}
        \caption{Homepage dashboard (Restaurant365)}
    \end{minipage}
\end{figure}

\begin{figure}[h]
    \centering
    \begin{minipage}{0.48\textwidth}
        \centering
        \includegraphics[width=\textwidth]{images/examples/forms_r365.png}
        \caption{Forms management (Restaurant365)}
    \end{minipage}
    \hfill
    \begin{minipage}{0.48\textwidth}
        \centering
        \includegraphics[width=\textwidth]{images/examples/image.png}
        \caption{General interface example}
    \end{minipage}
\end{figure}

\begin{figure}[h]
    \centering
    \begin{minipage}{0.48\textwidth}
        \centering
        \includegraphics[width=\textwidth]{images/examples/hiring_r365.png}
        \caption{Hiring process overview (Restaurant365)}
    \end{minipage}
    \hfill
    \begin{minipage}{0.48\textwidth}
        \centering
        \includegraphics[width=\textwidth]{images/examples/inventory_r365.png}
        \caption{Inventory management (Restaurant365)}
    \end{minipage}
\end{figure}

\begin{figure}[h]
    \centering
    \begin{minipage}{0.48\textwidth}
        \centering
        \includegraphics[width=\textwidth]{images/examples/item_creating_square.png}
        \caption{Item creation interface (Square)}
    \end{minipage}
    \hfill
    \begin{minipage}{0.48\textwidth}
        \centering
        \includegraphics[width=\textwidth]{images/examples/item_transactions_r365.png}
        \caption{Item transactions (Restaurant365)}
    \end{minipage}
\end{figure}

\begin{figure}[h]
    \centering
    \begin{minipage}{0.48\textwidth}
        \centering
        \includegraphics[width=\textwidth]{images/examples/item_inventory_square.png}
        \caption{Item inventory view (Square)}
    \end{minipage}
    \hfill
    \begin{minipage}{0.48\textwidth}
        \centering
        \includegraphics[width=\textwidth]{images/examples/item_variations_square.png}
        \caption{Item variations (Square)}
    \end{minipage}
\end{figure}

\begin{figure}[h]
    \centering
    \begin{minipage}{0.48\textwidth}
        \centering
        \includegraphics[width=\textwidth]{images/examples/item_menu_square.png}
        \caption{Menu item configuration (Square)}
    \end{minipage}
    \hfill
    \begin{minipage}{0.48\textwidth}
        \centering
        \includegraphics[width=\textwidth]{images/examples/item_vendor_items_r365.png}
        \caption{Vendor items management (Restaurant365)}
    \end{minipage}
\end{figure}

\begin{figure}[h]
    \centering
    \begin{minipage}{0.48\textwidth}
        \centering
        \includegraphics[width=\textwidth]{images/examples/item_recipes_r365.png}
        \caption{Recipe management (Restaurant365)}
    \end{minipage}
    \hfill
    \begin{minipage}{0.48\textwidth}
        \centering
        \includegraphics[width=\textwidth]{images/examples/jobs_r365.png}
        \caption{Jobs overview (Restaurant365)}
    \end{minipage}
\end{figure}

\begin{figure}[h]
    \centering
    \begin{minipage}{0.48\textwidth}
        \centering
        \includegraphics[width=\textwidth]{images/examples/items_cost_r365.png}
        \caption{Items cost analysis (Restaurant365)}
    \end{minipage}
    \hfill
    \begin{minipage}{0.48\textwidth}
        \centering
        \includegraphics[width=\textwidth]{images/examples/more_tasks_r365.png}
        \caption{Task list (Restaurant365)}
    \end{minipage}
\end{figure}

\begin{figure}[h]
    \centering
    \begin{minipage}{0.48\textwidth}
        \centering
        \includegraphics[width=\textwidth]{images/examples/payment_center_r365.png}
        \caption{Payment center (Restaurant365)}
    \end{minipage}
    \hfill
    \begin{minipage}{0.48\textwidth}
        \centering
        \includegraphics[width=\textwidth]{images/examples/task_management_r365.png}
        \caption{Task management (Restaurant365)}
    \end{minipage}
\end{figure}

\begin{figure}[h]
    \centering
    \begin{minipage}{0.48\textwidth}
        \centering
        \includegraphics[width=\textwidth]{images/examples/task_review_r365.png}
        \caption{Task review interface (Restaurant365)}
    \end{minipage}
    \hfill
    \begin{minipage}{0.48\textwidth}
        \centering
        \includegraphics[width=\textwidth]{images/examples/trainings_r365.png}
        \caption{Training modules (Restaurant365)}
    \end{minipage}
\end{figure}

% sestadieni reikes ikelti.

% didinti verslinguma nera poreikis, nebus tokio mygtuko

% tarpininkas turi gauti kazkokius notifications
% poreikiai bus padavejo GUI.

% tam tikros funkcijos ateis is pasakotojo pasakojimo, atsekamumo matrica
% customer nera usability objectives, nes jis naudoja per tarpininka, sita
% reikia pamineti.

% ook for existing design examples that could inspire future design solutions
% (at least 5). - geri is kitu sistemu, screenshot, kuriai funckijai cia galetu
% tikti, kodel manom kad otks dizainas tiktu.


% paziureti kaip needs rasomi pagal standarta, skaidrese, angliskai ir
% lietuviskai kazkur parasytas paaiskinimas

% atsekamumo matrica -  viena galima vienam dalykui, kita kitam. atsekamumas
% turi parodyti kad poreikis is kazkur isplaukia, is kurio scenarijaus, is kurio
% pasakojimo, is kurio user story.

% jeigu rasome kaip sekmes kriteriju kad turi vartotojas gauti kazka per viena
% min tai reiskia kazkur parasyta kad adabar jis ta gauna ilgiau nei viena min.


% tekste paveiksliuka pacituoti, pasakyti kodel jis yra ten, koks poreikis.

% ------------------------

% primari ir secondary stakeholder kursime sasaja
% poreikis - ka veiks vartotojas ir ka gaus ekrane




% ------------------------



% ------------------------
% PACT stuff below
% ------------------------



% reikia sistemizuoti kokioje situacijoje ko reikia, labiau detaliau issiaiskinti stakeholder needs.
% reikia atlikti naudojimo konteksto analize, kaip tos funkcijos isilies i vartotojo esama gyvenima. Mes galvojame apie naudojima ir konteksta vis dar.
% Naudojimo kontekstas - kaip veiklos itakoja technologiju kaita ir technologiju kaita veiklas, po to kalbesime apie pagrindines zmoniu charakteristikas, kas, kokiose veiklose ir problemas kurias norime spresti, kokiame zingsnyje kas ten stringa, kokioje aplinkoje ten vyksta, kokios tech naudojamos ir ka galime pasiulyti.
% Naudojimo kontekstas - kombinacija vartotoju, uzduociu, resursu, tikslu. Fizines, socialines, technines, kulturines, organizicines aplinkos. Jas riekes issaiskinti ir aprasyti kaip dabar veikia tas vartotojas, ka tie pasakojimai turi atspindeti: turi buti aiskus visi aspektai, kas, kokie tisklai, aplinkos, kokiais prietaisais, programomis naudojasi ir t.t.
% PS inzinerijos standartas - naudojimo kontekstas pagrindinis informacijos reikalavimu saltinis.
% Vystimasis yra begalinis ciklas, yra zmones betkokio amziaus, kutluros, jie veikia kazkokiam kontekste (requirements), tame kontekste kyla nauji reikalavimai naujoms technologijosm, naujos technologijos sudaro galimybe keistis veiklos, vel atsiras nauji reikalavimai ir t.t. gaunasi begalinis ratas.
% naujos interaktyvios tech - prisidedame dabartines zinias, sukaupta patirti ir kuriame naujus reikalavimus. Suvokdami kazkokius nepatogumus, koks yra stovis, kokias galimybes mes galime isnaudoti su technologijomis mes pasiulome sprendimus, naujas veiklas.

% People, context, activities, technologies

% ____________________________

% People - kokiomis salygomis veikia, ju fizines galimybes, darbo vieta, veikimo
% aplinka kur bus naudojama technologija, zmoniu ugis, aukstis, antropronetiniai
% dalykai? ivestis isvestis, klaviaturos ir t.t. Nera vidutinio vartotojo,
% universali technologija reiskia kad visiem patogu naudotis, zinoma yra tam
% tikri kompromisai, reikalavimai turi tilpti i budget. Pvz.: ekrano padetis,
% ryskumas, kontrastas, color blindness, motion sensitivity, hearing etc. etc.
% Musu varottoju tarpe jeigu yra vyresni zmones tai turime i tai atsizvelgti,
% mygtukai didesni ir t.t. Kokia turi buti darbo aplinka kad zmogus nepavargtu,
% nebutu broko, darbo vieta turi buti tokia kuri leistu islaikyti darbinguma.

% psihologinis erdviu suvokimas, kai kurie pasiklysta, kai kurie gerai
% orientuojasi erdvese. kalbos skirtumai, kai kam gali buti izeidzianti kalba,
% kai kam ne...

% isiminimas, trumpalaikis isiminimas, sprendimu priemimas, komunikacija,
% paieska - mes turime sias veiklas kurioms kuriame technologijas.

% socialiniai skirtumai - itakoja motyva pirkti nauja technologija, gali
% atsirasti stiprus motyvas, pvz bendrauti nuotoliniu budu, todel zmones ismoks
% tai. skirstosi tipai, begginer, inermediate, expert zmoniu. pvz bileteliu
% popieriniu nenupirksi, tai beggineriai turi tai naudoti... vidutiniskai patyre
% kai reikia tia naudoja o taip tai nenaudoja. profesionalai - kiekviena diena
% naudoja. beggineriu ir expertu poreikia yra labai skirtingi.

% kad pradedantysis galetu naudotis technologija tai ji turi vesti uz rankos,
% pvz vedlys, jis vedamas uz rankos.

% ekspertai - jiem reikia greicio, juos vedlys erzins.

% vidutiniskai patyre - reikia prisiminti greitai.

% jie visi turi skirtingus poreikius, reikia suprasti su kuo turime reikala.


% Mental model - mintinis modelis, koki vaizda suformavo vartotojas savo
% galvoje, mes bandome tai sufprasti, tai suvokimas kas ir kaip gali naudoti
% technologijas. Mintinis modelis nepilnas, nestabilus, mes negalime visko
% prisiminti, kazkas isilaiko ilgalaikeje atmintyje, kazkas dingsta.
% Analizuojame o ka dabar naudoja vartotojai, analizuojame ju mintines zinias.
% Mintinis modelis kuriamas saveikaujant su sistemomis, su kokiomis sistemomis
% saveikauja toki ir turesime mintini modeli.

% Nustatymas vartotojo igudzius yra musu tikslas. kokio amziaus, lytis, fizines,
% issilavinimas, kulturinis, motivacines galimybes, tikslai, asmenybe.
% Projektavimo tikslai turi susieti su tais skillais. musu tikslas kad
% vartotojai pilnai isnaudotu ka suprojektavome, norime kad is begginer greitai
% pereitu i intermediate.

% Pavyzdiziui kai pirma karta paleidziame programa, gali issokti gidas, padeti
% beginneriams, o kam nereikia tai gali uzdaryti, uzdaro vidutiniskai patyre.
% Ekspertam reikia kuo greiciau dirbti, kas stabdo (rankos kelimas nuo peles iki
% klaviaturos, galime sakyti kad viska darytu ant klaviaturos).

% kiek tu sluoksniu design reikia? tai papildomos islaidos ir t.t.


% begineriam programa turi pasakyti ka ji gali daryti, pagrindiniai dalykai ka
% ji daro, begineriu nera daug bet jie yra ir pirmi naudotojai todel jiems
% reikia viska paaiskinti. po keliu kartu dauguma begineriu tampa intermediate.
% intermediate svarbu matyti, isnaudoti visas funkcijas kurias jie zino kad yra,
% gebeti viska surasti, pazengusias funkcijas. GUI visas pagrindines ir
% pazengusias funkcijas rodo ir galima greitai surasti Ekspertu nera daug,
% kazkam galbut greiciau reikia.

% kuri viekla daznesne, ta turi buti arciau ekrano.


% Universal usability - prieinamas visiems, ir neigaliems. Panaudojamumas +
% prieinamumas = universalumas. kas aktualu, kam kuriame sistema, i sita turime
% atsakyti.



% Reikia pamineti demographics, age, occupation, gender (jei reikia),
% disabilities. Motyvacija ismokti technologijas, jomis naudotis. Naudojamos
% technologijos ir prietaisai, IT lygmuo, skillai.

% ____________________________

% Veiklos - reikia ne specifines mineti, o laiko aspektu daznas ar retas,
% bendradariavimo aspektus, individualus ar organizaciniai, tos veiklos
% sudetingumas, pasekmes, ar kritines, kokio tipo turinys toje veikloje.

% Daznis jos pirma charakteristika, jos trukme antra, laiko spaudimas, ar veikia
% ramybes busenoje ar skubos. Ar tai viena atomine veikla ar zingsniai yra, ar
% testine is zingsniu tai kazkuriame zingsnyje vartotojas gali sustoti. atsako
% laikas. Bendradarbiavimas, vienas ar grupeje, kaip zino kas ka padare, kaip
% koordinuoja, komunikuoja. Sudetingumas veiklos, ar veikla apibrezta ar ne,
% reikia suteikti vartotojui galimybe narsyti ivairias veiklas, kur vartotojas
% nori narsyti, kaip vartotojas supras ar nutrauke zingsni ir t.t. tai
% sudetingumas. kaip supras kad klaida padare, ar tai aktualu. kokie duomenys
% vaiksto, kiek ivesti ten reikia, kokia ten isvestis dabar ir kas siuloma, turi
% buti realiu laiku atnaujinamas turinys, kad nebutu blogai pavaizduotas, blogai
% atnaujintas.

% Ne pacios veiklos idomios bet ju charakteristikos.

% ____________________________

% Aplinka - fizine, kur vyksta ta veikla, patalpoje, uz patalpos, jeigu fizine
% aplinka lauke, tai skirtinga temperatura, kaip vartotojas su pirstinemis ar
% per lietu gali naudotis, apsvietimas, triuksmas ir t.t.

% Socialine aplinka - individuali, o gal grupine, ar privatumas aktualus, ar
% visi vienodu teisiu, ar yra super adminas.

%  Organizacinis kontekstas - teises ir t.t.

% ____________________________

% Technologijos - ka dabar naudoja zmones, kokias technologijas, ka jie dabar naudoja, tia itakoja ju mintini modeli, kaip jie dabar iveda informacija?

% Kokias technologijas naudosime - kompiuteris, programele telefone, ar t.t.

% Turime susivokti kokiu techniniu ir sistemu reikia. Kas yra geras turinys ir kokios charakteristikos galetu tobulinti tai, balsu ivedimas, AI ar whatever.

% kokia informacija ir funkcijos reikalingos sistemai, kas tures buti zinoma norinciam naudotis sitema.


% Interaktyvus produktas turi atitikti tai ko nori zmones, jis negali sugriauti kontekto, jis turi tai pagerinti.

% musu uzdavinys visus siuos dalykus issiaiskinti, parasyti kiekvienai pirminei ir antrinei vartotoju grupei aprasyti siuos dalykus ir po to dar kitai savaitei galime pradeti rasyti poreikius kas isto isplaukia ir kokiu funckiju riekia

% \sectionnonum{Results and conclusions}
% For details of what needs to be written in this section, please refer to the methodology requirements of the respective programme.


% ------------------------
% User Studies
% ------------------------

% Suvokimo procesas - nelabai gali issivaizduoti, projekte vyksta kuryba.
% Jeigu viskas patogu tai niekas nepastebi, jeigu kazkas nepatogu tai visi pastebi..

% Yra PACT sistema kuri apibendrina viska. Patys bendriausi reikalavimai yra
% vartotoju poreikiai, tai nevisai kaip reikalavimai. Poreikiai - ka vartotojas
% nuveiks, ka matys sistemoje.

% Vartotojau nori kad processas butu sklandus, intuityvus <- negeras aprasymas,
% ar cia bus mygtukas ar kazkas ekrane, cia yra labiau verslo lukesciai.



% Scenarijai turetu pateikti zingsnius kaip vartotojai dabar veikia, 
% what
% how
% any problems /\<- visa sita jau padarem

% ------------------------
% Koki tyrima darom
% yra kokybiniai ir kiekybiniai tyrimai
% kiekybiniai - apklausos, anketos, statistika, daug zmoniu, galime apibendrinti visai populiacijai, bet gauname mazai izvalgu
% kokybiniai - interviu, stebejimas, mazai zmoniu, gauname daug izvalgu, bet galime atsidurti klaidingoje situacijoje, negalime apibendrinti visai populiacijai


% kiekybiniams reikia didelio biudzeto.

% kiekybiniame tyrime suzinome nuostatas, problemas su laiku pradeda kartotis info.


% human centered design - suzinome prie ko zmones priprate ir atitinkamai
% kuriame technologijas. kokios veiklos daznos, retos, danzesnes veiklos
% pradiniame ekrane, retesnes kitame, etc.


% PERSONAS - isgrynina visas iskalbas kalbant su tam tikra vartotoju grupe.
% persona gaunama pakalbant bent su keliais atstovais is tos vartotoju grupes.
% persona yra konkretus atstovas skirtingu vartotoju grupiu.
% tikslai, prasmingos veiklos isskiriamos.


% zmones kurie daug zino ir kurie nenaudoja mainstream technologijas yra geras
% info saltinis. nereikia kalbeti su mainstream, uzkart galime suzinoti kas
% blogai su tuo mainstream. Gausime gera izvalga, kokie trukumai su ta
% mainstream technologija.



% 1 setting goals - atlikti PACT analize 

% 2 identifikuoti dalyvius, nuspresti is
% ko rinkti info (suinteresuotu analize jau atlikome), probability sampling
% (paimame atsitiktiniu budu zmones) arba non probability sampling (krastutiniu 
% grupiu). Saturation sampling - su visais kalbame kol nebelieka naujos info.

% 3 Relationship with participants - susitarti su vadovybe kad galime kalbinti
% zmones, virsininkas turi pristatyti kad bus toks tyrimas, etc. etc. Informed
% consent - pasiraso kad sutinka dalyvauti.

% 4 Triangulation - vienas info saltinis mazai, turi buti bent keli scenarijai,
% kad zinotume skirtingus poziurius, juos derintume.

% 5 Pilot studies - pabandome su vienu ar keliais zmonemis, kad pamatytume ar
% viskas ok, ar klausimai aiskus, ar suprantami.


% Interviews

% Pirma buna nestrukturizuoti, kuo toliau tuo labiau strukturuoti, tikslinames
% funkcionalumus, pozymius.

% semi strukturuoti - yra pagrindiniai klausimai, bet galima klausti ir papildomai

% group inteviesws - diskusijos, su keliais kalbam.

% vengti ilgu sakiniu, ar, ir, zargonu, vengti stumti i viena ar kita puse.

% fokuso grupe - atstovai is skirtingu grupiu, kai isriskeja skirtingi
% poziuriai. reikia kompromisiniu sprendimu.



\printbibliography[title = {References and sources}]

\end{document}

