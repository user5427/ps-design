%%%%%
%%%%%  Use LUALATEX, not LATEX.
%%%%%
%%%%
\documentclass[]{VUMIFTemplateClass}

\usepackage{indentfirst}
\usepackage{amsmath, amsthm, amssymb, amsfonts}
\usepackage{mathtools}
\usepackage{physics}
\usepackage{graphicx}
\usepackage{verbatim}
\usepackage[hidelinks]{hyperref}
\usepackage{xcolor,algorithm,algorithmic}
\definecolor{gray}{gray}{0.6}


\usepackage{tcolorbox}

\newcommand{\yellowcomment}[1]{%
    \begin{tcolorbox}[colback=yellow!80, colframe=yellow!80, arc=0pt, outer arc=0pt, boxrule=0pt, left=3pt, right=3pt, top=3pt, bottom=3pt]
        \textbf{\textcolor{red}{COMMENT:}} #1
    \end{tcolorbox}
}

\newcommand{\warningcomment}[1]{%
    \begin{tcolorbox}[colback=yellow!90, colframe=red, arc=0pt, outer arc=0pt, boxrule=2pt, left=5pt, right=5pt, top=5pt, bottom=5pt]
        \Large\textbf{\textcolor{red}{FIX THIS: }} \normalsize #1
    \end{tcolorbox}
}

\newcommand{\goodcomment}[1]{%
    \begin{tcolorbox}[colback=green!20, colframe=green!60, arc=0pt, outer arc=0pt, boxrule=1pt, left=3pt, right=3pt, top=3pt, bottom=3pt]
        \textbf{\textcolor{green!70!black}{GOOD:}} #1
    \end{tcolorbox}
}

\newcommand{\noticecomment}[1]{%
    \begin{tcolorbox}[colback=blue!20, colframe=blue!60, arc=0pt, outer arc=0pt, boxrule=1pt, left=3pt, right=3pt, top=3pt, bottom=3pt]
        \textbf{\textcolor{blue!70!black}{NOTE:}} #1
    \end{tcolorbox}
}

\newcommand{\todocomment}[1]{%
    \begin{tcolorbox}[colback=red!20, colframe=red!60, arc=0pt, outer arc=0pt, boxrule=1pt, left=3pt, right=3pt, top=3pt, bottom=3pt]
        \textbf{\textcolor{orange!70!black}{TODO:}} #1
    \end{tcolorbox}
}

\newcommand{\suggestioncomment}[1]{%
    \definecolor{lime}{RGB}{50,205,50}%
    \begin{tcolorbox}[colback=lime!15, colframe=lime!60, arc=0pt, outer arc=0pt, boxrule=1pt, left=3pt, right=3pt, top=3pt, bottom=3pt]
        \textbf{\textcolor{lime!70!black}{SUGGESTION:}} #1
    \end{tcolorbox}%
}



\newenvironment{mpitemlist}[1][\linewidth]{%
    \begin{minipage}[t]{#1}%
        \setlength{\leftmargini}{12pt}%
        \begin{itemize}%
            \setlength{\itemsep}{1pt}%
            \setlength{\parskip}{0pt}%
            \setlength{\parsep}{0pt}%
}{%
        \end{itemize}%
    \end{minipage}\newline
}

\usepackage[nottoc]{tocbibind}
\usepackage{tocloft}
\usepackage{longtable}
\usepackage{amssymb}

\usepackage{titlesec}
\newcommand{\sectionbreak}{\clearpage}

\makeatletter
\renewcommand{\fnum@algorithm}{\thealgorithm}
\makeatother
\renewcommand\thealgorithm{\arabic{algorithm} algorithm}

\usepackage{biblatex}
\bibliography{bibliografija}
%% to change the numbering (numeric or alphabetic) of bibliographic sources, make the change in VUMIFTemplateClass.cls, line 139

% Author's MACROS
\newcommand{\EE}{\mathbb{E}\,} % Mean
\newcommand{\ee}{{\mathrm e}}  % nice exponent
\newcommand{\RR}{\mathbb{R}}


\usepackage{titlesec}
\usepackage{hyperref}

\titleclass{\subsubsubsection}{straight}[\subsection]

\newcounter{subsubsubsection}[subsubsection]
\renewcommand\thesubsubsubsection{\thesubsubsection.\arabic{subsubsubsection}}
\renewcommand\theparagraph{\thesubsubsubsection.\arabic{paragraph}} % optional; useful if paragraphs are to be numbered

\titleformat{\subsubsubsection}
  {\normalfont\normalsize\bfseries}{\thesubsubsubsection}{1em}{}
\titlespacing*{\subsubsubsection}
{0pt}{3.25ex plus 1ex minus .2ex}{1.5ex plus .2ex}

\makeatletter
\renewcommand\paragraph{\@startsection{paragraph}{5}{\z@}%
  {3.25ex \@plus1ex \@minus.2ex}%
  {-1em}%
  {\normalfont\normalsize\bfseries}}
\renewcommand\subparagraph{\@startsection{subparagraph}{6}{\parindent}%
  {3.25ex \@plus1ex \@minus .2ex}%
  {-1em}%
  {\normalfont\normalsize\bfseries}}
\def\toclevel@subsubsubsection{4}
\def\toclevel@paragraph{5}
%\def\toclevel@paragraph{6}
\def\toclevel@subparagraph{6}
\def\l@subsubsubsection{\@dottedtocline{4}{7em}{4em}}
\def\l@paragraph{\@dottedtocline{5}{10em}{5em}}
\def\l@subparagraph{\@dottedtocline{6}{14em}{6em}}
\makeatother

\setcounter{secnumdepth}{4}
\setcounter{tocdepth}{4}
   
\studyprogramme{Software Engineering} 
\worktype{PS Software Design} 
\worktitle{PS Sofware Design documentation}
\workauthor{Tadas Riksas, Darius Spruogis, Gustas Mickus, Julius Jauga}
\supervisor{Vasilij Savin}

\begin{document}
\selectlanguage{english}

\onehalfspacing
\begin{titlepage}
\vskip 20pt
\begin{center}
\includegraphics[scale=0.55]{images/MIF.png}
\end{center}

\makeatletter

\vskip 20pt
\centerline{\bf \large \textbf{VILNIUS UNIVERSITY}}
\vskip 10pt
\centerline{\large \textbf{FACULTY OF MATHEMATICS AND INFORMATICS}}
\vskip 10pt
\centerline{\large \textbf{\MakeUppercase{\@studyprogramme \space study programme}}}

\vskip 80pt
\centerline{\Large \@worktype}
\vskip 20pt
\begin{center}
    {\bf \LARGE \@worktitle}
\end{center}
\begin{center}
    {\bf \Large \@secondworktitle}
\end{center}
\vskip 80pt

\centering{\Large \@workauthor}
\@ifundefined{@secondauthor}{}
{
\vskip 10pt
\centering{\Large \@secondauthor}
}
\vskip 20pt

\centering{
    \begin{tabular}{rcp{.7\textwidth}}
        {\Large Supervisor} & {\Large :} & {\Large \@supervisor}\\[10pt]
        \@ifundefined{@scientificadvisor}{}
            {
                {\Large Scientific advisor} & {\Large :} & {\Large \@scientificadvisor}\\[10pt]
            }
        \@ifundefined{@reviewer}{}
            {
                {\Large Reviewer} & {\Large :} & {\Large \@reviewer}\\[10pt]
            }
    \end{tabular}}


\vskip 110pt

\centerline{\large \textbf{Vilnius}}
\centerline{\large \textbf{\the\year{}}}

\makeatother

\newpage
\end{titlepage}
%\newgeometry{top=2cm,bottom=2cm,right=2cm,left=3cm}
\setcounter{page}{2}


\tableofcontents
\onehalfspacing



\section{Introduction}
The structure of the document is:
\begin{itemize}
    \item Introduciton
    \item Functionallity 
    \begin{itemize}
        \item Functionality domain 1
        Use cases
        \begin{itemize}
            \item GUIs (We will not include all GUIs, you will need to check out the Figma link to find all GUIs). Notice that GUIs may not always show all the functionality shown by business flow and that we are only showing the fundamental GUIs.
            \item Business flows
        \end{itemize}
        \item Functionality domain 2...
    \end{itemize}
    \item Package diagram
    \item Data model    
    \begin{itemize}
        \item The diagrams
        \item Discussion
    \end{itemize}
    \item API Contracts
    \item Conclusion
\end{itemize}

Note: after making 80+ Figma GUI wire-frames we realized that there is really no limit to the domain size of this application. While we did cover all the essential functionality, there is still a lot of questions left to solve. Here are some of them that we encountered:
\begin{itemize}
    \item How product supply chains work? How our system integrates that? (For inventory management)
    \item What to do in case payment system (Stripe) goes down? How will the system manage that?
    \item etc.
\end{itemize}
And there are way way more things to take in consideration. But writing all these things out would make this document huge and unreadable (hopefully, it is readable at the moment ). We are aware that this document lacks some features/details (or to put simply sucks in some areas) but that simply goes way beyond the scope of what we can do in this course. 

So why so many GUIs wire-frames? After trying to write this document for the first time using only text for describing how users interact with the system we kind of failed. Text is not enough to understand entire workflow. So, we decided that it would be ideal to simply use Figma and quickly prototype as much GUIs as possible for all interesting features. This would allow each team member to quickly understand how their colleagues understand the business problem and if there are any problems with that understanding. It also served as a great way to ensure consistency between system components. We hope that the reader will appreciate our huge effort put into these GUIs and that it will make their job easier.

\subsection{Important links}
\label{links}
Before you start it would be ideal if you familiarized your self with GUI.


Use these links to learn what are our UI prototypes, business flows, data-model and package diagrams, API contract.

\begin{itemize}
    \item \href{https://www.figma.com/design/IpiyOU6rDdMWJAuKfbXnuQ/HCI-2025-mockups?node-id=0-1&t=jng9VzugzoqFH0nr-1}{Figma (LINK)}
    
    All mock-ups are labeled as follows: \begin{itemize}
        \item Messages 1-2
        \item Account 1
        \item Menu 1
        \item Homepage 1
        \item Register 1-4
        \item Dashboard 1
        \item Business 1 - 21 (10 \& 18 removed)
        \item Discount 1-5
        \item Location 1-2
        \item Inventory 1-10
        \item Services 1-6
        \item Work 1-4
        \item Cashier 1-9
        \item Cashier(Ser) 1-6 (for services)
        \item Manager 1-6
        \item Kitchen 1-3
        \item Waiter 1-3
        \item Service 1-3
        \item Multi-role 1
    \end{itemize}
    \item \href{https://drive.google.com/drive/folders/1TnLndEsfCymHmstJi26E5D_D4CpJfU4y?usp=drive_link}{Drawio diagrams for business flows and packages (LINK)}
    \item \href{https://github.com/user5427/ps-design/blob/main/openapi.yaml}{API contract (LINK)}
\end{itemize}



\section{Functionality}


% ------------------------------------------------------------------------
% ORDER STUFF
% ------------------------------------------------------------------------



\subsection{Order Management (Bars/Restaurants)}


\begin{longtable}{p{0.25\linewidth} p{0.70\linewidth}}
\caption{Use cases for the Order Management domain} \\
\textbf{Use case} & \textbf{Description} \\
\hline
\endfirsthead

\multicolumn{2}{c}{{\tablename\ \thetable{} -- Continued from previous page}} \\
\textbf{Use case} & \textbf{Description} \\
\hline
\endhead

\multicolumn{2}{c}{{Continued on next page}} \\
\endfoot

\endlastfoot

Create order &
\begin{minipage}[t]{\linewidth}
Reserve table like A2, B1, etc., choose menu items and send order details to kitchen staff.
\end{minipage} \\[6pt]

\multicolumn{2}{@{}c@{}}{\color{gray}\rule{0.95\linewidth}{0.4pt}} \\[6pt]
Modify order &
\begin{minipage}[t]{\linewidth}
Change table reservation, choose menu items and send order details to kitchen staff.
\end{minipage} \\[6pt]

\multicolumn{2}{@{}c@{}}{\color{gray}\rule{0.95\linewidth}{0.4pt}} \\[6pt]
Pay for order &
\begin{minipage}[t]{\linewidth}
Customer chooses to pay for the items currently added to the list, employee initiates payment flow.
\end{minipage} \\[6pt]


\multicolumn{2}{@{}c@{}}{\color{gray}\rule{0.95\linewidth}{0.4pt}} \\[6pt]
Finalize order &
\begin{minipage}[t]{\linewidth}
Beginning is similar to \textbf{Pay for order} flow, at the end employee provides receipt.
\end{minipage} \\[6pt]

\end{longtable}

\subsubsection{GUIs}

The order creating GUI (\ref{fig:order-creating}) allows for employee to choose the table where to serve the customer (employee can also open table view (\ref{fig:table-view}). It also shows menu on  for quickly choosing and searching for items. On the right side there is order information order information. Employee can send order details to the kitchen staff, initiate customer payment process for current items and finalize the order by printing receipt. 


\begin{figure}[H]
    \centering
    \includegraphics[width=0.8\textwidth]{docs/ps-design/design-document/images/GUI/Work create order.png}
    \caption{Order creating GUI (Cashier - 5 in Figma)}
    \label{fig:order-creating}
\end{figure}

This table view GUI (\ref{fig:table-view}) allows for employee to quickly identify table numeration and to select what orders are currently active for table. It also shows how many seats each table has. 

\begin{figure}[H]
    \centering
    \includegraphics[width=0.8\textwidth]{docs/ps-design/design-document/images/GUI/Work floor plan.png}
    \caption{Table view GUI (Cashier - 4 in Figma)}
    \label{fig:table-view}
\end{figure}

This order editing GUI (\ref{fig:order-editing}) is very similar to the order creating  GUI (\ref{fig:order-creating}). It is shown after the first payment is made (by redirect). It shows what items are sent and allows to    add new ones. It also allows to edit table number. At the bottom there is a refund button in case user asks for refund.

\begin{figure}[H]
    \centering
    \includegraphics[width=0.8\textwidth]{docs/ps-design/design-document/images/GUI/Work view order.png}
    \caption{Order editing GUI (Cashier - 6 in Figma)}
    \label{fig:order-editing}
\end{figure}





\subsubsection{Business flows}

\subsubsubsection{Create order}
The creating order flow (\ref{fig:creating-order-flow}) uses order creating (\ref{fig:order-creating}) when employee chooses menu items and table view (\ref{fig:table-view}) when employee chooses table number.
\begin{figure}[H]
    \centering
    \includegraphics[width=0.8\textwidth]{docs/ps-design/design-document/images/flows/create_order.png}
    \caption{Creating order flow}
    \label{fig:creating-order-flow}
\end{figure}
\begin{center}
\setlength{\tabcolsep}{8pt}
\begin{tabular}{|p{0.48\linewidth}|p{0.48\linewidth}|}
\hline
\textbf{Tables/Entities} \newline
\begin{mpitemlist}
\item Employee
\item Menu item
\item Order
\item Order item
\item Location seating
\end{mpitemlist}
&
\textbf{Components} \newline
\begin{mpitemlist}
\item Employee comp.
\item Location comp.
\item Order comp.
\item Menu comp.
\end{mpitemlist}
\\ \hline
\textbf{Actions} \newline
\begin{mpitemlist}
\item Assign table
\item Add items to order
\item Send order details to kitchen staff
\end{mpitemlist}
&

\\ \hline
\end{tabular}
\end{center}


\subsubsubsection{Modify order}

The modifying order flow (\ref{fig:modifying-order-flow}) uses order modifying (\ref{fig:order-editing}) when employee adds new items to order and table view {\ref{fig:table-view}} if changing customers chosen table.

\begin{figure}[H]
    \centering
    \includegraphics[width=0.5\textwidth]{docs/ps-design/design-document/images/flows/modify_order.png}
    \caption{Modifying order flow}
    \label{fig:modifying-order-flow}
\end{figure}

\begin{center}
\setlength{\tabcolsep}{8pt}
\begin{tabular}{|p{0.48\linewidth}|p{0.48\linewidth}|}
\hline
\textbf{Tables/Entities} \newline
\begin{mpitemlist}
\item Same as in order creating
\end{mpitemlist}
&
\textbf{Components} \newline
\begin{mpitemlist}
\item Same as in order creating
\end{mpitemlist}
\\ \hline
\textbf{Actions} \newline
\begin{mpitemlist}
\item Get table number
\item Edit order details
\item Send order to kitchen staff
\end{mpitemlist}
&

\\ \hline
\end{tabular}
\end{center}

\subsubsubsection{Pay for order}

The pay for order flow (\ref{fig:pay-order-flow}) shows either creating order (\ref{fig:order-creating}) or editing order GUI based on whenever the employee is creating the order for first time or is editing it. Employee then can press the pay button to initiate the payment flow.

\begin{figure}[H]
    \centering
    \includegraphics[width=0.4\textwidth]{docs/ps-design/design-document/images/flows/pay_for_order.png}
    \caption{Pay for order flow}
    \label{fig:pay-order-flow}
\end{figure}

\begin{center}
\setlength{\tabcolsep}{8pt}
\begin{tabular}{|p{0.48\linewidth}|p{0.48\linewidth}|}
\hline
\textbf{Tables/Entities} \newline
\begin{mpitemlist}
\item Order
\item Order bill
\end{mpitemlist}
&
\textbf{Components} \newline
\begin{mpitemlist}
\item Order comp.
\item Payment comp.
\end{mpitemlist}
\\ \hline
\textbf{Actions} \newline
\begin{mpitemlist}
\item Initiate payment process
\end{mpitemlist}
&

\\ \hline
\end{tabular}
\end{center}

\subsubsubsection{Finalize order}

The finalize order (\ref{fig:finalize-order-flow}) is very similar to pay for order flow (\ref{fig:pay-order-flow}), the only difference is that after the end of pay for order flow employee clicks the print button to print out the receipt.

\begin{figure}[H]
    \centering
    \includegraphics[width=0.3\textwidth]{docs/ps-design/design-document/images/flows/finalize_order.png}
    \caption{Finalize order flow}
    \label{fig:finalize-order-flow}
\end{figure}

\begin{center}
\setlength{\tabcolsep}{8pt}
\begin{tabular}{|p{0.48\linewidth}|p{0.48\linewidth}|}
\hline
\textbf{Tables/Entities} \newline
\begin{mpitemlist}
\item Same as in pay for order flow
\end{mpitemlist}
&
\textbf{Components} \newline
\begin{mpitemlist}
\item Same as in pay for order flow
\end{mpitemlist}
\\ \hline
\textbf{Actions} \newline
\begin{mpitemlist}
\item Initiate payment process
\item Print receipt
\end{mpitemlist}
&

\\ \hline
\end{tabular}
\end{center}



% ------------------------------------------------------------------------
% SERVICE STUFF
% ------------------------------------------------------------------------

\subsection{Services Management (Spas/Salons)}



\vspace{1cm}
\begin{longtable}{p{0.25\linewidth} p{0.70\linewidth}}
\caption{Use cases for the Services Management domain} \\
\textbf{Use case} & \textbf{Description} \\
\hline
\endfirsthead

\multicolumn{2}{c}{{\tablename\ \thetable{} -- Continued from previous page}} \\
\textbf{Use case} & \textbf{Description} \\
\hline
\endhead

\multicolumn{2}{c}{{Continued on next page}} \\
\endfoot

\endlastfoot

Create appointment &
\begin{minipage}[t]{\linewidth}
Customer calls business, asks for service, employee provides details, reserves the slot, customer gets SMS message.
\end{minipage} \\[6pt]
\multicolumn{2}{@{}c@{}}{\color{gray}\rule{0.95\linewidth}{0.4pt}} \\[6pt]

Modify appointment &
\begin{minipage}[t]{\linewidth}
Change appointment details (time, service, date, assigned employee) and customer gets SMS message.
\end{minipage} \\[6pt]
\multicolumn{2}{@{}c@{}}{\color{gray}\rule{0.95\linewidth}{0.4pt}} \\[6pt]

Cancel appointment &
\begin{minipage}[t]{\linewidth}
Customer calls to cancels an appointment and receives SMS message for cancellation. 
\end{minipage} \\[6pt]
\multicolumn{2}{@{}c@{}}{\color{gray}\rule{0.95\linewidth}{0.4pt}} \\[6pt]

Service Customer &
\begin{minipage}[t]{\linewidth}
    The customer gets the service done by the assigned employee; employee marks the service as completed in the system, which updates records and triggers payment flow. Customer may also ask refund if they do not like how the appointment was done. After appointment is done customer receives SMS message.
\end{minipage} \\
\end{longtable}

\vspace{1cm}

\subsubsection{GUIs}

In the appointment creating GUI (\ref{fig:appointment-creating} employee can create an appointment by inputting customer details such as name, surname, email and phone. There employee can also choose time, services, search for services served by specific employees and lastly add break time.

\begin{figure}[H]
    \centering
    \includegraphics[width=0.8\textwidth]{docs/ps-design/design-document/images/GUI/Work appointment creating.png}
    \caption{Appointment creating GUI (Cashier (Ser) - 3 in Figma)}
    \label{fig:appointment-creating}
\end{figure}

In the employee checking GUI (\ref{fig:employee-checking}) employee can check what employees are busy and free.

\begin{figure}[H]
    \centering
    \includegraphics[width=0.8\textwidth]{docs/ps-design/design-document/images/GUI/Work employees.png}
    \caption{Employee checking GUI (Cashier (Ser) - 5 in Figma)}
    \label{fig:employee-checking}
\end{figure}


In the appointment details GUI (\ref{fig:appointment details}) employee can check the details of an appointment, start refund process, cancel appointment or initiate pay process.

\begin{figure}[H]
    \centering
    \includegraphics[width=0.8\textwidth]{docs/ps-design/design-document/images/GUI/Work appointments.png}
    \caption{Appointment details GUI (Cashier (Ser) - 6 in Figma)}
    \label{fig:appointment details}
\end{figure}



\subsubsection{Business flows}

Notice: all message sending is done automatically by system based on appointment status.

\subsubsubsection{Create appointment}

The appointment creating flow (\ref{fig:create-appointment-flow}) uses (\ref{fig:appointment-creating}) GUI. The employee enters customer information, chooses time (available from) and service and enters break time if needed. (Break time shows how much time to leave blocked after appointment so employee can get ready for next appointment)

\begin{figure}[H]
    \centering
    \includegraphics[width=0.5\textwidth]{docs/ps-design/design-document/images/flows/create appointment.png}
    \caption{Create appointment flow}
    \label{fig:create-appointment-flow}
\end{figure}

\begin{center}
\setlength{\tabcolsep}{8pt}
\begin{tabular}{|p{0.48\linewidth}|p{0.48\linewidth}|}
\hline
\textbf{Tables/Entities} \newline
\begin{mpitemlist}
\item Employee
\item Service
\item Location service
\item Employee service

\end{mpitemlist}
&
\textbf{Components} \newline
\begin{mpitemlist}
\item Employee comp.
\item Service comp.
\item Appointment comp.
\item Message comp.
\end{mpitemlist}
\\ \hline
\textbf{Actions} \newline
\begin{mpitemlist}
\item Search employees
\item Search services
\item Get customer details
\item Create appointment
\end{mpitemlist}
&

\\ \hline
\end{tabular}
\end{center}


\subsubsubsection{Modify appointment}

The appointment modifying flow (\ref{fig:modify-appointment-flow}) uses GUI similar to (\ref{fig:appointment-creating}) except all fields are filed in already. Employee can change all the information like customer details, service type, time (available from), etc.

\begin{figure}[H]
    \centering
    \includegraphics[width=0.5\textwidth]{docs/ps-design/design-document/images/flows/modify appointment.png}
    \caption{Modify appointment flow}
    \label{fig:modify-appointment-flow}
\end{figure}


\begin{center}
\setlength{\tabcolsep}{8pt}
\begin{tabular}{|p{0.48\linewidth}|p{0.48\linewidth}|}
\hline
\textbf{Tables/Entities} \newline
\begin{mpitemlist}
\item Same to creating appointment flow.
\end{mpitemlist}
&
\textbf{Components} \newline
\begin{mpitemlist}
\item Same to creating appointment flow.
\end{mpitemlist}
\\ \hline
\textbf{Actions} \newline
\begin{mpitemlist}
\item Get customer details
\item Check if appointment exists
\item Search employees
\item Search services
\item Complete appointment modification
\end{mpitemlist}
&

\\ \hline
\end{tabular}
\end{center}



\subsubsubsection{Cancel appointment}

The cancel appointment flow (\ref{fig:cancel-appointment-flow}) uses (\ref{fig:appointment details}) GUI. There employee can click cancel button to cancel the appointment.

\begin{figure}[H]
    \centering
    \includegraphics[width=0.5\textwidth]{docs/ps-design/design-document/images/flows/cancel appointment.png}
    \caption{Cancel appointment flow}
    \label{fig:cancel-appointment-flow}
\end{figure}


\begin{center}
\setlength{\tabcolsep}{8pt}
\begin{tabular}{|p{0.48\linewidth}|p{0.48\linewidth}|}
\hline
\textbf{Tables/Entities} \newline
\begin{mpitemlist}
\item Employee
\item Appointment
\end{mpitemlist}
&
\textbf{Components} \newline
\begin{mpitemlist}
\item Employee comp.
\item Appointment comp.
\end{mpitemlist}
\\ \hline
\textbf{Actions} \newline
\begin{mpitemlist}
\item Get customer details
\item Check if appointment exists
\item Cancel appointment
\end{mpitemlist}
&

\\ \hline
\end{tabular}
\end{center}




\subsubsubsection{Service customer}

The service customer flow (\ref{fig:service-customer-flow}) uses GUI (\ref{fig:employee-checking}). There cashier can check what appointments are coming up with customer names and surnames. Individual employees that will service the customer can see GUI similar to (\ref{fig:appointment details}) except there is no pay or cancel buttons but only one button and that is "mark serviced" button. After customer is serviced, customer walks to cashier and cashier uses  (\ref{fig:appointment details}) GUI to start payment flow.

\begin{figure}[H]
    \centering
    \includegraphics[width=0.5\textwidth]{docs/ps-design/design-document/images/flows/service customer.png}
    \caption{Service customer flow}
    \label{fig:service-customer-flow}
\end{figure}

\begin{center}
\setlength{\tabcolsep}{8pt}
\begin{tabular}{|p{0.48\linewidth}|p{0.48\linewidth}|}
\hline
\textbf{Tables/Entities} \newline
\begin{mpitemlist}
\item Employee
\item Employee service
\item Appointment
\item Appointment bill
\end{mpitemlist}
&
\textbf{Components} \newline
\begin{mpitemlist}
\item Employee comp.
\item Service comp.
\item Appointment comp.
\item Payment comp.
\end{mpitemlist}
\\ \hline
\textbf{Actions} \newline

\begin{mpitemlist}
\item Get customer details
\item Check if appointment exists
\item Service customer
\item Begin payment process
Refund customer (in case customer dislikes sth)
\end{mpitemlist}
&

\\ \hline
\end{tabular}
\end{center}





% ------------------------------------------------------------------------
% PAYMENT STUFF
% ------------------------------------------------------------------------


\subsection{Payment}

\textbf{Supported Payment Methods:}

\begin{itemize}
\item Cash
\item Gift card
\item Credit/debit card (via Stripe)
\end{itemize}

\textbf{Key Business Rules:}

\begin{itemize}
\item Split payments are supported: an order total can be paid by multiple parties and/or multiple payment methods.
\item Employees can add tips and apply discounts during payment.
\item Final receipts must include applicable taxes, tips, and discounts for each payment.
\item Closed/paid orders are preserved for audit purposes and can be refunded (partial or total).
\item Refund for card payments is processed through Stripe. For other methods, the status is updated in the system.
\end{itemize}

\subsubsection{GUIs}

The refund GUI (\ref{fig:refund-GUI}) allows for employees to initiate refund process by inputting the reason for refund.

The Stripe payment GUI (\ref{fig:stripe-payment}) shows how our payments will be handled.

\begin{figure}[H]
    \centering
    \begin{minipage}{0.48\textwidth}
        \centering
        \includegraphics[width=\textwidth]{docs/ps-design/design-document/images/GUI/Frame 5 (1).png}
        \caption{Refund GUI}
        \label{fig:refund-GUI}
    
    \end{minipage}
    \hfill
    \begin{minipage}{0.48\textwidth}
        \centering
        \includegraphics[width=\textwidth]{docs/ps-design/design-document/images/GUI/Frame 4.png}
        \caption{Stripe payment GUI}
        \label{fig:stripe-payment}
    \end{minipage}
\end{figure}


\subsubsection{Business flows}

\subsubsubsection{Payment flow}
The following sequence diagram (\ref{fig:payment_flow}) illustrates the payment process, including split payments, tips, and refunds.  (\ref{fig:stripe-payment}) and (\ref{fig:refund-GUI}) GUIs are used.

\begin{figure}[H]
    \centering
    \includegraphics[width=0.95\textwidth]{images/diagrams/payment/payment_flow.png}
    \caption{Payment Flow Sequence Diagram}
    \label{fig:payment_flow}
\end{figure}



% ------------------------------------------------------------------------
% BUSINESS STUFF
% ------------------------------------------------------------------------



\subsection{Business Management}

\vspace{1cm}
\begin{longtable}{p{0.25\linewidth} p{0.70\linewidth}}
\caption{Use cases for the Business Management domain} \\
\textbf{Use case} & \textbf{Description} \\
\hline
\endfirsthead

\multicolumn{2}{c}{{\tablename\ \thetable{} -- Continued from previous page}} \\
\textbf{Use case} & \textbf{Description} \\
\hline
\endhead

\multicolumn{2}{c}{{Continued on next page}} \\
\endfoot

\endlastfoot

User Registration \& Invitation Flow &
\begin{minipage}[t]{\linewidth}
The system allows a new user to sign up as an employee or business owner. It defines the entry point to the system and role-based access setup.
\end{minipage} \\[6pt]
\multicolumn{2}{@{}c@{}}{\color{gray}\rule{0.95\linewidth}{0.4pt}} \\[6pt]
User Management &
\begin{minipage}[t]{\linewidth}
The business owner can invite employees or co-owners to help run their business or manage existing users in their business.
\end{minipage} \\[6pt]
\multicolumn{2}{@{}c@{}}{\color{gray}\rule{0.95\linewidth}{0.4pt}} \\[6pt]
Business Creation \& Setup &
\begin{minipage}[t]{\linewidth}
The business owner creates and configures a new business entity by providing necessary details and setting its operational type (service or order).
\end{minipage} \\[6pt]
\multicolumn{2}{@{}c@{}}{\color{gray}\rule{0.95\linewidth}{0.4pt}} \\[6pt]
Manage Employee Information &
\begin{minipage}[t]{\linewidth}
The Owner or Co-Owner views and modifies an employee’s information, performance, or feedback data.
\end{minipage} \\[6pt]
\multicolumn{2}{@{}c@{}}{\color{gray}\rule{0.95\linewidth}{0.4pt}} \\[6pt]
Manage Co-Owner Information &
\begin{minipage}[t]{\linewidth}
The Owner manages Co-Owners within the business. They can modify their permissions (e.g., access to employee management or financial data) or remove them entirely from the business.
\end{minipage} \\[6pt]
\multicolumn{2}{@{}c@{}}{\color{gray}\rule{0.95\linewidth}{0.4pt}} \\[6pt]
Manage Jobs &
\begin{minipage}[t]{\linewidth}
The owner or co-owner creates or modifies job roles for employees within a business. A job includes a name, description, assigned role, and an activity timetable specifying minimum and maximum durations for tasks.
\end{minipage} \\[6pt]
\multicolumn{2}{@{}c@{}}{\color{gray}\rule{0.95\linewidth}{0.4pt}} \\[6pt]
Manage Locations &
\begin{minipage}[t]{\linewidth}
The owner or co-owner manages business locations. A location includes address, open hours, and a seating plan represented as a grid with labeled tables (e.g., 2A, 5A, 1B). Locations can be created, modified, or deleted.
\end{minipage} \\[6pt]
\multicolumn{2}{@{}c@{}}{\color{gray}\rule{0.95\linewidth}{0.4pt}} \\[6pt]
Task Assignment &
\begin{minipage}[t]{\linewidth}
Managers create tasks for employees, assigning responsibilities with deadlines. Employees can view and do their tasks.
\end{minipage} \\[6pt]
\multicolumn{2}{@{}c@{}}{\color{gray}\rule{0.95\linewidth}{0.4pt}} \\[6pt]
Peer Feedback \& Ratings &
\begin{minipage}[t]{\linewidth}
Employees can rate and review colleagues based on friendliness, teamwork, or behavior. When submitting feedback, the employee can choose to remain anonymous. Feedback is stored in the system and optionally aggregated for management insight.
\end{minipage} \\[6pt]
\multicolumn{2}{@{}c@{}}{\color{gray}\rule{0.95\linewidth}{0.4pt}} \\[6pt]

\end{longtable}


We did not include a super admin since the operations would be the same, the only one thing changes is that the super admin could control all the businesses. In addition super admin could change global application variables such as country currency, VAT tax rate, etc. (see \ref{fig:application_data} data model for more information)


\subsubsection{GUIs}

In this GUI (\ref{fig:sign-up-page}) user can choose to sign up with his email and also enter password. We will not implement the third party authentication for now.
\begin{figure}[H]
    \centering
    \includegraphics[width=0.5\linewidth]{Sign up page.png}
    \caption{Sign up page}
    \label{fig:sign-up-page}
\end{figure}

In this GUI (\ref{fig:choose-roles}) user can choose the account type. This is used for GUI to decide if business section should be shown or not.
\begin{figure}[H]
    \centering
    \includegraphics[width=0.5\linewidth]{Role chooser.png}
    \caption{Role chooser}
    \label{fig:choose-roles}
\end{figure}
In this GUI (\ref{fig:enter-info}) user can enter additional information about himself and his contact information.
\begin{figure}[H]
    \centering
    \includegraphics[width=0.5\linewidth]{Role chooser (1).png}
    \caption{User information}
    \label{fig:enter-info}
\end{figure}

In this GUI (\ref{fig:66}) business owner can invite a new employee to business, specify job, working hours, and location and the expiration of the invite.
% 6
\begin{figure}[H]
    \centering
    \includegraphics[width=\textwidth]{docs/ps-design/design-document/images/GUI/business/Specific business employee code invite.png}
    \caption{employee invite (Business - 11 Figma)}
    \label{fig:66}
\end{figure}




In this GUI (\ref{fig:44}) business owner can invite a coowner, specify expiration of the invite and permissions.
% 4
\begin{figure}[H]
    \centering
    \includegraphics[width=\textwidth]{docs/ps-design/design-document/images/GUI/business/Specific business co-owner code invite.png}
    \caption{business co-owner invite (Business - 19 Figma)}
    \label{fig:44}
\end{figure}





In this GUI (\ref{fig:11}) user can create a new business, specify name, description, type, country and business contact information.
% 1
\begin{figure}[H]
    \centering
    \includegraphics[width=\textwidth]{docs/ps-design/design-document/images/GUI/business/Business creating.png}
    \caption{Business creating (Business - 2 Figma)}
    \label{fig:11}
\end{figure}





In this GUI (\ref{fig:88}) business owner can manage employee data, change working hours, salary, location and status.

% 8
\begin{figure}[H]
    \centering
    \includegraphics[width=\textwidth]{docs/ps-design/design-document/images/GUI/business/Specific Busines Manage Employee modify.png}
    \caption{Business Manage Employee (Business - 15 Figma)}
    \label{fig:88}
\end{figure}



In this GUI (\ref{fig:77}) business owner can see how well their employees are doing, if they are getting good ratings from other employees or bad. They can also see who left the feedback and what are their jobs.
% 7
\begin{figure}[H]
    \centering
    \includegraphics[width=\textwidth]{docs/ps-design/design-document/images/GUI/business/Specific Busines Manage Employee modify (1).png}
    \caption{Business Employee Ratings (Business - 17 Figma)}
    \label{fig:77}
\end{figure}


In this GUI (\ref{fig:33}) business owner can change co-owner permissions.
% 3
\begin{figure}[H]
    \centering
    \includegraphics[width=\textwidth]{docs/ps-design/design-document/images/GUI/business/Specific Busines owners.png}
    \caption{Business co-owner managing (Business - 21 Figma)}
    \label{fig:33}
\end{figure}
   

In this GUI (\ref{fig:55}) business owner can create a new job by naming it, giving a description, choosing a few system predefined roles such as cashier, manager, kitchen staff, waiter, cashier (service), service. There is also a way to define activities of the job and what proportions of employee work time it will take.
% 5
\begin{figure}[H]
    \centering
    \includegraphics[width=\textwidth]{docs/ps-design/design-document/images/GUI/business/Specific Busines jobs.png}
    \caption{Business job creating (Business - 13 Figma)}
    \label{fig:55}
\end{figure}


In this GUI (\ref{fig:22}) business owner can create a new work locaiton, specify working hours and seating layout.
% 2
\begin{figure}[H]
    \centering
    \includegraphics[width=\textwidth]{docs/ps-design/design-document/images/GUI/business/Business location creating.png}
    \caption{Business location creating (Business - 12 Figma)}
    \label{fig:22}
\end{figure}



In this GUI (\ref{fig:1010}) manager can give new tasks to employees such as "clean the tables", etc. Manager can assign multiple employees to one task. When employee completes the task they simply delete it from their menu (not shown).
% 10
\begin{figure}[H]
    \centering
    \includegraphics[width=\textwidth]{docs/ps-design/design-document/images/GUI/business/work tasks.png}
    \caption{Work tasks (Manager - 3 Figma)}
    \label{fig:1010}
\end{figure}


In this GUI (\ref{fig:99}) employee can give feedback to another employee from 0 to 5 rating, choosing to give the employee as anonymous or no. They can also make the feedback only available to business owners to see.
% 9
\begin{figure}[H]
    \centering
    \includegraphics[width=\textwidth]{docs/ps-design/design-document/images/GUI/business/Work schedule.png}
    \caption{Feedback (Work - 4 Figma)}
    \label{fig:99}
\end{figure}





\newpage
\subsubsection{Business flows}

\subsubsubsection{User Registration \& Invitation Business Flow}

The user registration and invitation flow (\ref{fig:user_registration_invitation_flow}) uses (\ref{fig:sign-up-page}), (\ref{fig:choose-roles}) and (\ref{fig:enter-info}) GUIs. (Note in the flow the enter personal info and select roles nodes are mixed up, first goes role selection, then additional info inputting actions)

\begin{figure}[H]
    \centering
    \includegraphics[width=0.7\textwidth]{docs/ps-design/design-document/images/diagrams/business/user_registration.png}
    \caption{User Registration and Invitation Business Flow Diagram}
    \label{fig:user_registration_invitation_flow}
\end{figure}


\begin{center}
\setlength{\tabcolsep}{8pt}
\renewcommand{\arraystretch}{1.3}
\begin{tabular}{|p{0.48\linewidth}|p{0.48\linewidth}|}
\hline
\textbf{Tables / Entities} \newline
\begin{mpitemlist}
\item \textbf{User} – stores registered users (employees or owners).
\item \textbf{Invitation} – holds user invitations to businesses as Employee/Co-owner
\item \textbf{Business} – associates users with existing companies.
\item \textbf{Auth Credentials} – stores encrypted passwords and sessions.
\end{mpitemlist}
&
\textbf{Components} \newline
\begin{mpitemlist}
\item \textbf{Registration UI} – sign-up and login forms.
\item \textbf{Invitation comp.} – validates user invitation.
\item \textbf{Auth comp.} – manages authentication and sessions.
\end{mpitemlist}
\\ \hline
\textbf{User Actions} \newline
\begin{mpitemlist}
\item Open homepage and choose “Sign Up” or “Log In”.
\item Enter details.
\item Submit registration form.
\end{mpitemlist}
&
\textbf{System Reactions} \newline
\begin{mpitemlist}
\item Validate inputs and user invitation.
\item Create user with assigned role and business link.
\item Generate auth session and redirect to dashboard.
\end{mpitemlist}
\\ \hline
\end{tabular}
\end{center}



\subsubsubsection{User Management Business Flow}

The user management flows (\ref{fig:user_manage_flow}), (\ref{fig:user_manage_flow}) and uses (\ref{fig:66}), (\ref{fig:44}), (\ref{fig:88}), (\ref{fig:33}) GUIs.

\begin{figure}[H]
    \centering
    \includegraphics[width=0.8\textwidth]{docs/ps-design/design-document/images/diagrams/business/bpmn_user_manage.png}
    \caption{Employee Management Business Flow Diagram}
    \label{fig:user_manage_flow}
\end{figure}
\begin{figure}[H]
    \centering
    \includegraphics[width=0.8\textwidth]{docs/ps-design/design-document/images/diagrams/business/bpmn_user_manage_coowner.png}
    \caption{Co-Owner Management Business Flow Diagram}
    \label{fig:user_manage_flow}
\end{figure}


\begin{center}
\setlength{\tabcolsep}{8pt}
\renewcommand{\arraystretch}{1.3}
\begin{tabular}{|p{0.48\linewidth}|p{0.48\linewidth}|}
\hline
\textbf{Tables / Entities} \newline
\begin{mpitemlist}
\item \textbf{Owner} – manages business users.
\item \textbf{User} – employees or co-owners linked to a business.
\item \textbf{Business} – connects users to the correct company.
\end{mpitemlist}
&
\textbf{Components} \newline
\begin{mpitemlist}
\item \textbf{User UI} – for inviting, searching, and managing users.
\item \textbf{Auth Service} – checks permissions.
\item \textbf{User comp.} – handles user actions (create, update, remove).
\end{mpitemlist}
\\ \hline
\textbf{User Actions} \newline
\begin{mpitemlist}
\item Log in and open a business.
\item Go to “Employees” section.
\item Invite or manage existing users.
\end{mpitemlist}
&
\textbf{System Reactions} \newline
\begin{mpitemlist}
\item Verify permissions.
\item Execute request via User Service.
\item Update database and show result.
\end{mpitemlist}
\\ \hline
\end{tabular}
\end{center}




\subsubsubsection{Business Creation Business Flow}

The business creation flow (\ref{fig:business_creation_flow}) uses (\ref{fig:11}) GUIs.

\begin{figure}[H]
    \centering
    \includegraphics[width=0.7\textwidth]{docs/ps-design/design-document/images/diagrams/business/business_creation.png}
    \caption{Business Creation Business Flow Diagram}
    \label{fig:business_creation_flow}
\end{figure}

\begin{center}
\setlength{\tabcolsep}{8pt}
\renewcommand{\arraystretch}{1.3}
\begin{tabular}{|p{0.48\linewidth}|p{0.48\linewidth}|}
\hline
\textbf{Tables / Entities} \newline
\begin{mpitemlist}
\item \textbf{Business} – stores name, type, and description.
\item \textbf{Owner} – user who creates the business.
\item \textbf{Contact Info} – holds support email and phone.
\item \textbf{Country} – links business to its location.
\end{mpitemlist}
&
\textbf{Components} \newline
\begin{mpitemlist}
\item \textbf{Business UI} – form for business creation.
\item \textbf{Business comp.} – validates and saves business data.
\item \textbf{Auth comp.} – verifies user identity.
\end{mpitemlist}
\\ \hline
\textbf{User Actions} \newline
\begin{mpitemlist}
\item Log in and go to “Create Business”.
\item Enter details: name, type, country, contact info.
\item Submit form.
\end{mpitemlist}
&
\textbf{System Reactions} \newline
\begin{mpitemlist}
\item Validate inputs and ownership.
\item Create business linked to owner.
\item Save contact info and metadata.
\item Show success or error message.
\end{mpitemlist}
\\ \hline
\end{tabular}
\end{center}



\subsubsubsection{Manage Employee Information Business Flow}

The manage employee information flow (\ref{fig:user_manage_flow}) uses (\ref{fig:88}) GUI.

\begin{figure}[H]
    \centering
    \includegraphics[width=1\textwidth]{docs/ps-design/design-document/images/diagrams/business/bpmn_manage_employee.png}
    \caption{Employee Information Management Business Flow Diagram}
    \label{fig:user_manage_flow}
\end{figure}

\begin{center}
\setlength{\tabcolsep}{8pt}
\renewcommand{\arraystretch}{1.3}
\begin{tabular}{|p{0.48\linewidth}|p{0.48\linewidth}|}
\hline
\textbf{Tables / Entities} \newline
\begin{mpitemlist}
\item \textbf{Owner / Co-Owner} – users managing employee data.
\item \textbf{Employee} – staff member whose data or feedback is managed.
\item \textbf{Business} – links employees to their business context.
\end{mpitemlist}
&
\textbf{Components} \newline
\begin{mpitemlist}
\item \textbf{Employee UI} – interface to view and modify employee details or feedback.
\item \textbf{Auth comp.} – checks management permissions.
\item \textbf{Employee comp.} – retrieves and updates employee data.
\item \textbf{Feedback comp.} – provides employee ratings and reviews.
\end{mpitemlist}
\\ \hline
\textbf{User Actions} \newline
\begin{mpitemlist}
\item Log in and go to \textit{Business → Employees}.
\item Select \textit{Manage Employee} and search for an employee.
\item Choose an action:
    \begin{itemize}
        \item Modify Information (salary, hours, status)
        \item View Feedback (ratings, reviews)
    \end{itemize}
\item Save any changes.
\end{mpitemlist}
&
\textbf{System Reactions} \newline
\begin{mpitemlist}
\item Verify access rights.
\item Display employee data.
\item Process requested updates or feedback retrieval.
\item Save changes to the database.
\item Show confirmation or error message.
\item Refresh profile with latest data.
\end{mpitemlist}
\\ \hline
\end{tabular}
\end{center}



\subsubsubsection{Manage Co-Owner Information Business Flow}

The manage co-owner information flow (\ref{fig:manage_co_owner_information_business_flow}) uses (\ref{fig:33}) GUI.
\begin{figure}[H]
    \centering
    \includegraphics[width=0.6\textwidth]{docs/ps-design/design-document/images/diagrams/business/bpmn_manage_coowner_information.png}
    \caption{Co-Owner Information Management Business Flow Diagram}
    \label{fig:manage_co_owner_information_business_flow}
\end{figure}

\begin{center}
\setlength{\tabcolsep}{8pt}
\renewcommand{\arraystretch}{1.3}
\begin{tabular}{|p{0.48\linewidth}|p{0.48\linewidth}|}
\hline
\textbf{Tables / Entities} \newline
\begin{mpitemlist}
\item \textbf{Owner} – owner with specific permissions.
\item \textbf{Business} –  specific business.
\item \textbf{Auth Session} – verifies Owner access and authentication.
\end{mpitemlist}
&
\textbf{Components} \newline
\begin{mpitemlist}
\item \textbf{Owner UI} – interface for editing or removing co-owners.
\item \textbf{Auth comp.} – checks Owner authority.
\item \textbf{Business owner comp.} – updates or removes co-owners.
\end{mpitemlist}
\\ \hline
\textbf{User Actions} \newline
\begin{mpitemlist}
\item Log in and go to \textit{Business → Owners}.
\item Select a Co-Owner.
\item Choose to modify permissions or remove access.
\item Save or confirm the action.
\end{mpitemlist}
&
\textbf{System Reactions} \newline
\begin{mpitemlist}
\item Verify Owner authorization.
\item Load Co-Owner details.
\item Apply requested updates or removal.
\item Save changes to the database.
\item Show success or error message.
\item Refresh the Co-Owner list.
\end{mpitemlist}
\\ \hline
\end{tabular}
\end{center}




\subsubsubsection{Manage Jobs Business Flow}

The manage jobs flow (\ref{fig:manage_jobs_business_flow}) uses (\ref{fig:55}) GUI.

\begin{figure}[H]
    \centering
    \includegraphics[width=0.6\textwidth]{docs/ps-design/design-document/images/diagrams/business/bpmn_manage_jobs.png}
    \caption{Job Management Business Flow Diagram}
    \label{fig:manage_jobs_business_flow}
\end{figure}

\begin{center}
\setlength{\tabcolsep}{8pt}
\renewcommand{\arraystretch}{1.3}
\begin{tabular}{|p{0.48\linewidth}|p{0.48\linewidth}|}
\hline
\textbf{Tables / Entities} \newline
\begin{mpitemlist}
\item \textbf{Business} – company that owns job definitions.
\item \textbf{Job} – stores job name, description, and role data.
\item \textbf{User} – owner or co-owner managing job entries.
\end{mpitemlist}
&
\textbf{Components} \newline
\begin{mpitemlist}
\item \textbf{Job UI} – create or edit job details and timetables.
\item \textbf{Job comp.} – validates and updates job data.
\item \textbf{Auth comp.} – checks user permissions.
\end{mpitemlist}
\\ \hline
\textbf{User Actions} \newline
\begin{mpitemlist}
\item Open the “Jobs” page.
\item Create or edit a job.
\item Enter details and activities.
\item Save changes.
\end{mpitemlist}
&
\textbf{System Reactions} \newline
\begin{mpitemlist}
\item Validate input data.
\item Save or update job records.
\item Show success or error message.
\item Refresh job list.
\end{mpitemlist}
\\ \hline
\end{tabular}
\end{center}



\subsubsubsection{Manage Locations Business Flow}

The manage locations flow (\ref{fig:manage_locations_business_flow}) uses (\ref{fig:22}) GUI.

\begin{figure}[H]
    \centering
    \includegraphics[width=0.6\textwidth]{docs/ps-design/design-document/images/diagrams/business/bpmn_manage_locations.png}
    \caption{Location Management Business Flow Diagram}
    \label{fig:manage_locations_business_flow}
\end{figure}

\begin{center}
\setlength{\tabcolsep}{8pt}
\renewcommand{\arraystretch}{1.3}
\begin{tabular}{|p{0.48\linewidth}|p{0.48\linewidth}|}
\hline
\textbf{Tables / Entities} \newline
\begin{mpitemlist}
\item \textbf{Business} – company managing branches or locations.
\item \textbf{Location} – stores name, address, and operating details.
\item \textbf{Seating Plan} – defines table layout (e.g., 2A, 1B).
\item \textbf{User} – owner or co-owner managing locations.
\end{mpitemlist}
&
\textbf{Components} \newline
\begin{mpitemlist}
\item \textbf{Location UI} – manage creation, editing, or deletion of locations.
\item \textbf{Location comp.} – validates and updates records.
\item \textbf{Auth comp.} – verifies management permissions.
\end{mpitemlist}
\\ \hline
\textbf{User Actions} \newline
\begin{mpitemlist}
\item Go to “Locations”.
\item Add, edit, or delete a location.
\item Enter details and seating plan.
\item Save or confirm action.
\end{mpitemlist}
&
\textbf{System Reactions} \newline
\begin{mpitemlist}
\item Validate input and permissions.
\item Save, update, or remove records.
\item Show success or error message.
\item Refresh location list.
\end{mpitemlist}
\\ \hline
\end{tabular}
\end{center}



\subsubsubsection{Task Assignment Business Flow}

The task assignment flow (\ref{fig:task_assignment_business_flow}) uses (\ref{1010}) GUI.

\begin{figure}[H]
    \centering
    \includegraphics[width=0.6\textwidth]{docs/ps-design/design-document/images/diagrams/business/task_assignment.png}
    \caption{Task Assignment Business Flow Diagram}
    \label{fig:task_assignment_business_flow}
\end{figure}

\begin{center}
\setlength{\tabcolsep}{8pt}
\renewcommand{\arraystretch}{1.3}
\begin{tabular}{|p{0.48\linewidth}|p{0.48\linewidth}|}
\hline
\textbf{Tables / Entities} \newline
\begin{mpitemlist}
\item \textbf{Task} – defines assignments with title, description, and deadline.
\item \textbf{Employee} – users receiving assigned tasks.
\item \textbf{Manager} – owners or co-owners assigning tasks.
\item \textbf{Business} – links tasks to the business.
\item \textbf{Task Status} – tracks progress updates.
\end{mpitemlist}
&
\textbf{Components} \newline
\begin{mpitemlist}
\item \textbf{Task UI} – create, assign, and track tasks.
\item \textbf{Task comp.} – validates and updates task data.
\item \textbf{Auth comp.} – checks manager permissions.
\end{mpitemlist}
\\ \hline
\textbf{User Actions} \newline
\begin{mpitemlist}
\item Open “Tasks” page.
\item Create or edit a task.
\item Assign employees.
\item Monitor progress.
\end{mpitemlist}
&
\textbf{System Reactions} \newline
\begin{mpitemlist}
\item Verify permissions.
\item Save or update task data.
\item Update task status.
\item Refresh task list view.
\end{mpitemlist}
\\ \hline
\end{tabular}
\end{center}



\subsubsubsection{Peer Feedback \& Ratings Business Flow}

The peer feedback and ratings flow (\ref{fig:feedback_business_flow}) uses (\ref{fig:77}) and (\ref{fig:99}) GUIs.

\begin{figure}
    \centering
    \includegraphics[width=0.8\textwidth]{docs/ps-design/design-document/images/diagrams/business/feedback.png}
    \caption{Feedback Business Flow Diagram}
    \label{fig:feedback_business_flow}
\end{figure}

\begin{center}
\setlength{\tabcolsep}{8pt}
\renewcommand{\arraystretch}{1.3}
\begin{tabular}{|p{0.48\linewidth}|p{0.48\linewidth}|}
\hline
\textbf{Tables / Entities} \newline
\begin{mpitemlist}
\item \textbf{Feedback} – stores comments and ratings.
\item \textbf{Employee} – includes both reviewer and reviewee.
\item \textbf{Rating} – numeric evaluation score.
\item \textbf{Anonymous Flag} – indicates if feedback is anonymous.
\item \textbf{Business} – links feedback to the business context.
\end{mpitemlist}
&
\textbf{Components} \newline
\begin{mpitemlist}
\item \textbf{Feedback UI} – for selecting colleagues and submitting reviews.
\item \textbf{Feedback comp.} – validates, anonymizes, and saves feedback.
\item \textbf{Employee comp.} – ensures both employees belong to the same business.
\end{mpitemlist}
\\ \hline
\textbf{User Actions} \newline
\begin{mpitemlist}
\item Go to “Feedback” page.
\item Select a colleague and add rating/comment.
\item Choose to post anonymously.
\item Submit feedback.
\end{mpitemlist}
&
\textbf{System Reactions} \newline
\begin{mpitemlist}
\item Validate and save feedback.
\item Apply anonymity if selected.
\item Update employee’s average rating.
\item Show success or error message.
\end{mpitemlist}
\\ \hline
\end{tabular}
\end{center}




\section{Package diagram}

\begin{figure}[H]
    \centering
    \includegraphics[width=0.9\textwidth]{docs/ps-design/design-document/images/package/package2.png}
    \caption{Package Diagram}
    \label{fig:package_diagram}
\end{figure}

\subsection*{Internal packages overview:}

\begin{itemize}
    \item \textbf{User Management System} – Generic user account logic (e.g. user, invites)
    \item \textbf{Authentication} – Account authentication specific logic.
    \item \textbf{Business Management System} – Core logic for client business.
    \item \textbf{Employee Management System} – Client business employee logic
    \item \textbf{Order Based System} – Order-based (e.g. restaurant, bar) specific logic
    \item \textbf{Service Based System} – Appointment-based (e.g. spa, massage) specific logic
    \item \textbf{Pricing System} – Shared price and discount logic
    \item \textbf{Payment System} – All payment logic, integrates with Stripe
    \item \textbf{Application Data} - Shared "static" configuration data (e.g user types, permission definitions) 
    \item \textbf{Version Data} – Shared logic for data versioning (historical records)
\end{itemize}

\subsection*{External packages overview:}

\begin{itemize}
    \item \textbf{SMS Provider} - Service used to send messages to people (AWS SMS)
    \item \textbf{Stripe} – Payment gateway
\end{itemize}

\section{Data model}


 We would highly recommend to use the Draw.io link at the introduction of our document (\ref{links}) for better viewing and connection tracing. We will be going through all diagrams in the sequence (red arrows) shown below:

\begin{figure}[H]
    \centering
    \includegraphics[width=\textwidth]{docs/ps-design/design-document/images/datamodel/sequence.png}
    \caption{Menu}
    \label{fig:menu_data}
\end{figure}



We used the violet boxes (\ref{fig:gui_ref}) to indicate references to GUIs in Figma, green box (\ref{fig:package_nam}) to indicate package name. You can also check our color codes in \ref{fig:color_codes} to see how we indicated keys and unique values.

\begin{figure}[H]
    \centering
    \includegraphics[width=0.3\textwidth]{docs/ps-design/design-document/images/datamodel/Screenshot_20251020_084416.png}
    \caption{GUI reference}
    \label{fig:gui_ref}
\end{figure}

\begin{figure}[H]
    \centering
    \includegraphics[width=0.3\textwidth]{docs/ps-design/design-document/images/datamodel/Screenshot_20251020_084446.png}
    \caption{Package name}
    \label{fig:package_nam}
\end{figure}

\begin{figure}[H]
    \centering
    \includegraphics[width=0.7\textwidth]{docs/ps-design/design-document/images/datamodel/Screenshot_20251020_104255.png}
    \caption{Color codes}
    \label{fig:color_codes}
\end{figure}

\subsection{The diagrams}

\begin{figure}[H]
    \centering
    \includegraphics[width=\textwidth]{docs/ps-design/design-document/images/datamodel/Application-Data.png}
    \caption{Application data}
    \label{fig:application_data}
\end{figure}


\vspace{0.3cm}
\noindent
Arrows going to application data (\ref{fig:application_data}):

\noindent
\begin{tabular}{|p{0.3\textwidth}|p{0.3\textwidth}|p{0.3\textwidth}|}
\hline
Business data (\ref{fig:business_data}) & Business owner (\ref{fig:business_owner_data}) & \\
\hline
\end{tabular}

\vspace{0.3cm}
\noindent
Arrows going from application data (\ref{fig:application_data}):

\noindent
\begin{tabular}{|p{0.3\textwidth}|p{0.3\textwidth}|p{0.3\textwidth}|}
\hline
Version data (\ref{fig:version_data}) & & \\
\hline
\end{tabular}

\begin{figure}[H]
    \centering
    \includegraphics[width=0.8\textwidth]{docs/ps-design/design-document/images/datamodel/user_data.png}
    \caption{User}
    \label{fig:user_data}
\end{figure}

\noindent
Arrows going to user (\ref{fig:user_data}):

\noindent
\begin{tabular}{|p{0.3\textwidth}|p{0.3\textwidth}|p{0.3\textwidth}|}
\hline
Notification (\ref{fig:notification_data}) & Invite (\ref{fig:invite_data}) & Employee data (\ref{fig:employee_data}) \\
\hline
Business owner (\ref{fig:business_owner_data}) & Message (\ref{fig:message_data}) & Appointment (\ref{fig:appointment_data}) \\
\hline
Version data (\ref{fig:version_data}) & Inventory (\ref{fig:inventory_data}) & Payment (\ref{fig:payment_data}) \\\hline 
Refresh token (\ref{fig:authentication}) & & \\ 
\hline
\end{tabular}

\vspace{0.3cm}
\noindent
Arrows going from user (\ref{fig:user_data}):

\noindent
\begin{tabular}{|p{0.3\textwidth}|p{0.3\textwidth}|p{0.3\textwidth}|}
\hline
Application data (\ref{fig:application_data}) & & \\
\hline
\end{tabular}


\begin{figure}[H]
    \centering
    \includegraphics[width=0.6\textwidth]{docs/ps-design/design-document/images/datamodel/authentication.png}
    \caption{Refresh Token}
    \label{fig:authentication}
\end{figure}

\noindent
Arrows going to refresh token (\ref{fig:message_data}):

\noindent
\begin{tabular}{|p{0.3\textwidth}|p{0.3\textwidth}|p{0.3\textwidth}|}
\hline
- & & \\
\hline
\end{tabular}

\vspace{0.3cm}
\noindent
Arrows going from refresh token (\ref{fig:user_data}):

\noindent
\begin{tabular}{|p{0.3\textwidth}|p{0.3\textwidth}|p{0.3\textwidth}|}
\hline
User (\ref{fig:user_data}) & & \\
\hline
\end{tabular}


\begin{figure}[H]
    \centering
    \includegraphics[width=\textwidth]{docs/ps-design/design-document/images/datamodel/messages.png}
    \caption{Message}
    \label{fig:message_data}
\end{figure}

\noindent
Arrows going to message (\ref{fig:message_data}):

\noindent
\begin{tabular}{|p{0.3\textwidth}|p{0.3\textwidth}|p{0.3\textwidth}|}
\hline
- & & \\
\hline
\end{tabular}

\vspace{0.3cm}
\noindent
Arrows going from message (\ref{fig:message_data}):

\noindent
\begin{tabular}{|p{0.3\textwidth}|p{0.3\textwidth}|p{0.3\textwidth}|}
\hline
User (\ref{fig:user_data}) & & \\
\hline
\end{tabular}

\begin{figure}[H]
    \centering
    \includegraphics[width=\textwidth]{docs/ps-design/design-document/images/datamodel/Business-Data.png}
    \caption{Business}
    \label{fig:business_data}
\end{figure}

\noindent
Arrows going to business (\ref{fig:business_data}):

\noindent
\begin{tabular}{|p{0.3\textwidth}|p{0.3\textwidth}|p{0.3\textwidth}|}
\hline
Business owner data (\ref{fig:business_owner_data}) & Employee (\ref{fig:employee_data}) & Job (\ref{fig:job_data}) \\
\hline
Menu (\ref{fig:menu_data}) & Inventory (\ref{fig:inventory_data}) & Discount (\ref{fig:discount_data}) \\
\hline
Location (\ref{fig:location_data}) & Service (\ref{fig:service_data}) & \\
\hline
\end{tabular}

\vspace{0.3cm}
\noindent
Arrows going from business (\ref{fig:business_data}):

\noindent
\begin{tabular}{|p{0.3\textwidth}|p{0.3\textwidth}|p{0.3\textwidth}|}
\hline
Application data (\ref{fig:application_data}) & Version data (\ref{fig:version_data}) & \\
\hline
\end{tabular}

\begin{figure}[H]
    \centering
    \includegraphics[width=\textwidth]{docs/ps-design/design-document/images/datamodel/Location.png}
    \caption{Location}
    \label{fig:location_data}
\end{figure}

\noindent
Arrows going to location (\ref{fig:location_data}):

\noindent
\begin{tabular}{|p{0.3\textwidth}|p{0.3\textwidth}|p{0.3\textwidth}|}
\hline
Inventory data (\ref{fig:inventory_data}) & Employee (\ref{fig:employee_data}) & Employee tasks (\ref{fig:employee_tasks}) \\
\hline
Menu (\ref{fig:menu_data}) & Order (\ref{fig:order_data}) & Discount (\ref{fig:discount_data}) \\
\hline
Service (\ref{fig:service_data}) & & \\
\hline
\end{tabular}

\vspace{0.3cm}
\noindent
Arrows going from location (\ref{fig:location_data}):

\noindent
\begin{tabular}{|p{0.3\textwidth}|p{0.3\textwidth}|p{0.3\textwidth}|}
\hline
Application data (\ref{fig:application_data}) & Version data (\ref{fig:version_data}) & Business (\ref{fig:business_data}) \\
\hline
\end{tabular}

\begin{figure}[H]
    \centering
    \includegraphics[width=\textwidth]{docs/ps-design/design-document/images/datamodel/Service.png}
    \caption{Service}
    \label{fig:service_data}
\end{figure}

\noindent
Arrows going to service (\ref{fig:service_data}):

\noindent
\begin{tabular}{|p{0.3\textwidth}|p{0.3\textwidth}|p{0.3\textwidth}|}
\hline
Appointment (\ref{fig:appointment_data}) & & \\
\hline
\end{tabular}

\vspace{0.3cm}
\noindent
Arrows going from service (\ref{fig:service_data}):

\noindent
\begin{tabular}{|p{0.3\textwidth}|p{0.3\textwidth}|p{0.3\textwidth}|}
\hline
Business (\ref{fig:business_data}) & Location (\ref{fig:location_data}) & Employee data (\ref{fig:employee_data}) \\
\hline
Price data (\ref{fig:price_data}) & Version data (\ref{fig:version_data}) & \\
\hline
\end{tabular}

\begin{figure}[H]
    \centering
    \includegraphics[width=\textwidth]{docs/ps-design/design-document/images/datamodel/appointment.png}
    \caption{Appointment}
    \label{fig:appointment_data}
\end{figure}

The \texttt{discount\_id} in appointment bill table is used for whole bill discounts. In this case it may be not used as specific location service uses \texttt{price\_id} which also has \texttt{discount\_id} attribute. 


\noindent
Arrows going to appointment (\ref{fig:appointment_data}):

\noindent
\begin{tabular}{|p{0.3\textwidth}|p{0.3\textwidth}|p{0.3\textwidth}|}
\hline
Payment (\ref{fig:payment_data}) & & \\
\hline
\end{tabular}

\vspace{0.3cm}
\noindent
Arrows going from appointment (\ref{fig:appointment_data}):

\noindent
\begin{tabular}{|p{0.3\textwidth}|p{0.3\textwidth}|p{0.3\textwidth}|}
\hline
Service (\ref{fig:service_data}) & Discount (\ref{fig:discount_data}) & Application data (\ref{fig:application_data}) \\
\hline
User (actioned by) & & \\
\hline
\end{tabular}

\begin{figure}[H]
    \centering
    \includegraphics[width=0.8\textwidth]{docs/ps-design/design-document/images/datamodel/Version-Data.png}
    \caption{Version data}
    \label{fig:version_data}
\end{figure}

\noindent
Arrows going to version data (\ref{fig:version_data}):

\noindent
\begin{tabular}{|p{0.3\textwidth}|p{0.3\textwidth}|p{0.3\textwidth}|}
\hline
Discount (\ref{fig:discount_data}) & Service  (\ref{fig:service_data}) & Menu (\ref{fig:menu_data}) \\
\hline
Recipe  (\ref{fig:recipe_data}) & Price  (\ref{fig:price_data}) & Location  (\ref{fig:location_data}) \\
\hline
Business data (\ref{fig:business_data}) & Business owner  (\ref{fig:business_owner_data}) & Employee  (\ref{fig:employee_data}) \\
\hline
\end{tabular}

\vspace{0.3cm}
\noindent
Arrows going from version data (\ref{fig:version_data}):

\noindent
\begin{tabular}{|p{0.3\textwidth}|p{0.3\textwidth}|p{0.3\textwidth}|}
\hline
User  (created by, updated by) & & \\
\hline
\end{tabular}

\begin{figure}[H]
    \centering
    \includegraphics[width=\textwidth]{docs/ps-design/design-document/images/datamodel/Discount-Data.png}
    \caption{Discount data}
    \label{fig:discount_data}
\end{figure}

\noindent
Arrows going to discount  (\ref{fig:discount_data}):

\noindent
\begin{tabular}{|p{0.3\textwidth}|p{0.3\textwidth}|p{0.3\textwidth}|}
\hline
Price  (\ref{fig:price_data}) & Order  (\ref{fig:order_data}) & Appointment  (\ref{fig:appointment_data}) \\
\hline
\end{tabular}

\vspace{0.3cm}
\noindent
Arrows going from discount  (\ref{fig:discount_data}):

\noindent
\begin{tabular}{|p{0.3\textwidth}|p{0.3\textwidth}|p{0.3\textwidth}|}
\hline
Version data (\ref{fig:version_data}) & Location  (\ref{fig:location_data}) & Business  (\ref{fig:business_data}) \\
\hline
\end{tabular}

\begin{figure}[H]
    \centering
    \includegraphics[width=\textwidth]{docs/ps-design/design-document/images/datamodel/Order-Data.png}
    \caption{Order }
    \label{fig:order_data}
\end{figure}

The \texttt{discount\_id} in order bill table is used for whole bill discounts.

\noindent
Arrows going to order  (\ref{fig:order_data}):

\noindent
\begin{tabular}{|p{0.3\textwidth}|p{0.3\textwidth}|p{0.3\textwidth}|}
\hline
Payment  (\ref{fig:payment_data}) & & \\
\hline
\end{tabular}

\vspace{0.3cm}
\noindent
Arrows going from order  (\ref{fig:order_data}):

\noindent
\begin{tabular}{|p{0.3\textwidth}|p{0.3\textwidth}|p{0.3\textwidth}|}
\hline
Location  (\ref{fig:location_data}) & User  (created by) & Application data (\ref{fig:application_data}) \\
\hline
Discount  (\ref{fig:discount_data}) & Menu  (\ref{fig:menu_data}) & \\
\hline
\end{tabular}

\begin{figure}[H]
    \centering
    \includegraphics[width=\textwidth]{docs/ps-design/design-document/images/datamodel/inventory.png}
    \caption{Inventory }
    \label{fig:inventory_data}
\end{figure}

\noindent
Arrows going to inventory  (\ref{fig:inventory_data}):

\noindent
\begin{tabular}{|p{0.3\textwidth}|p{0.3\textwidth}|p{0.3\textwidth}|}
\hline
Menu  (\ref{fig:menu_data}) & & \\
\hline
\end{tabular}

\vspace{0.3cm}
\noindent
Arrows going from inventory  (\ref{fig:inventory_data}):

\noindent
\begin{tabular}{|p{0.3\textwidth}|p{0.3\textwidth}|p{0.3\textwidth}|}
\hline
Business  (\ref{fig:business_data}) & Location (\ref{fig:location_data}) & User  (created by) \\
\hline
\end{tabular}

\begin{figure}[H]
    \centering
    \includegraphics[width=\textwidth]{docs/ps-design/design-document/images/datamodel/Business-Owner-Data.png}
    \caption{Business owner}
    \label{fig:business_owner_data}
\end{figure}

\noindent
Arrows going to business owner  (\ref{fig:business_owner_data}):

\noindent
\begin{tabular}{|p{0.3\textwidth}|p{0.3\textwidth}|p{0.3\textwidth}|}
\hline
- & & \\
\hline
\end{tabular}

\vspace{0.3cm}
\noindent
Arrows going from business owner  (\ref{fig:business_owner_data}):

\noindent
\begin{tabular}{|p{0.3\textwidth}|p{0.3\textwidth}|p{0.3\textwidth}|}
\hline
Application  (\ref{fig:application_data}) & User  (\ref{fig:user_data}) & Business  (\ref{fig:business_data}) \\
\hline
Version data (\ref{fig:version_data}) & & \\
\hline
\end{tabular}

\begin{figure}[H]
    \centering
    \includegraphics[width=\textwidth]{docs/ps-design/design-document/images/datamodel/Invite.png}
    \caption{Invite }
    \label{fig:invite_data}
\end{figure}

\noindent
Arrows going to invite  (\ref{fig:invite_data}):

\noindent
\begin{tabular}{|p{0.3\textwidth}|p{0.3\textwidth}|p{0.3\textwidth}|}
\hline
- & & \\
\hline
\end{tabular}

\vspace{0.3cm}
\noindent
Arrows going from invite  (\ref{fig:invite_data}):

\noindent
\begin{tabular}{|p{0.3\textwidth}|p{0.3\textwidth}|p{0.3\textwidth}|}
\hline
User  (\ref{fig:user_data}) & & \\
\hline
\end{tabular}

\begin{figure}[H]
    \centering
    \includegraphics[width=\textwidth]{docs/ps-design/design-document/images/datamodel/Notification-Data.png}
    \caption{Notification }
    \label{fig:notification_data}
\end{figure}

\noindent
Arrows going to notification  (\ref{fig:notification_data}):

\noindent
\begin{tabular}{|p{0.3\textwidth}|p{0.3\textwidth}|p{0.3\textwidth}|}
\hline
- & & \\
\hline
\end{tabular}

\vspace{0.3cm}
\noindent
Arrows going from notification  (\ref{fig:notification_data}):

\noindent
\begin{tabular}{|p{0.3\textwidth}|p{0.3\textwidth}|p{0.3\textwidth}|}
\hline
User  (\ref{fig:user_data}) & & \\
\hline
\end{tabular}

\begin{figure}[H]
    \centering
    \includegraphics[width=0.7\textwidth]{docs/ps-design/design-document/images/datamodel/employee.png}
    \caption{Employee}
    \label{fig:employee_data}
\end{figure}

\noindent
Arrows going to employee  (\ref{fig:employee_data}):

\noindent
\begin{tabular}{|p{0.3\textwidth}|p{0.3\textwidth}|p{0.3\textwidth}|}
\hline
Employee feedback (\ref{fig:employee_feedback}) & Employee tasks (\ref{fig:employee_tasks}) & Service  (\ref{fig:service_data}) \\
\hline
\end{tabular}

\vspace{0.3cm}
\noindent
Arrows going from employee  (\ref{fig:employee_data}):

\noindent
\begin{tabular}{|p{0.3\textwidth}|p{0.3\textwidth}|p{0.3\textwidth}|}
\hline
Job  (\ref{fig:job_data}) & Application data (\ref{fig:application_data}) & Version data (\ref{fig:version_data}) \\
\hline
Business  (\ref{fig:business_data}) & Location  (\ref{fig:location_data}) & \\
\hline
\end{tabular}

\begin{figure}[H]
    \centering
    \includegraphics[width=\textwidth]{docs/ps-design/design-document/images/datamodel/Menu.png}
    \caption{Menu data}
    \label{fig:menu_data}
\end{figure}

\noindent
Arrows going to menu  (\ref{fig:menu_data}):

\noindent
\begin{tabular}{|p{0.3\textwidth}|p{0.3\textwidth}|p{0.3\textwidth}|}
\hline
Order  (\ref{fig:order_data}) & Recipe  (\ref{fig:recipe_data}) & \\
\hline
\end{tabular}

\vspace{0.3cm}
\noindent
Arrows going from menu  (\ref{fig:menu_data}):

\noindent
\begin{tabular}{|p{0.3\textwidth}|p{0.3\textwidth}|p{0.3\textwidth}|}
\hline
Business  (\ref{fig:business_data}) & Location  (\ref{fig:location_data}) & Price  (\ref{fig:price_data}) \\
\hline
Inventory  (\ref{fig:inventory_data}) & Version  (\ref{fig:version_data}) & \\
\hline
\end{tabular}

\begin{figure}[H]
    \centering
    \includegraphics[width=\textwidth]{docs/ps-design/design-document/images/datamodel/Payment-Data.png}
    \caption{Payment }
    \label{fig:payment_data}
\end{figure}

\noindent
Arrows going to payment  (\ref{fig:payment_data}):

\noindent
\begin{tabular}{|p{0.3\textwidth}|p{0.3\textwidth}|p{0.3\textwidth}|}
\hline
- & & \\
\hline
\end{tabular}

\vspace{0.3cm}
\noindent
Arrows going from payment (\ref{fig:payment_data}):

\noindent
\begin{tabular}{|p{0.3\textwidth}|p{0.3\textwidth}|p{0.3\textwidth}|}
\hline
Order  (\ref{fig:order_data}) & Appointment  (\ref{fig:appointment_data}) & User  (actioned by) \\
\hline
Application data (\ref{fig:application_data}) & & \\
\hline
\end{tabular}

\begin{figure}[H]
    \centering
    \includegraphics[width=\textwidth]{docs/ps-design/design-document/images/datamodel/Price.png}
    \caption{Price }
    \label{fig:price_data}
\end{figure}

\noindent
Arrows going to price  (\ref{fig:price_data}):

\noindent
\begin{tabular}{|p{0.3\textwidth}|p{0.3\textwidth}|p{0.3\textwidth}|}
\hline
Menu  (\ref{fig:menu_data}) & Service  (\ref{fig:service_data}) & \\
\hline
\end{tabular}

\vspace{0.3cm}
\noindent
Arrows going from price  (\ref{fig:price_data}):

\noindent
\begin{tabular}{|p{0.3\textwidth}|p{0.3\textwidth}|p{0.3\textwidth}|}
\hline
Version data (\ref{fig:version_data}) & Application data (\ref{fig:application_data}) & Discount  (\ref{fig:discount_data}) \\
\hline
\end{tabular}

\begin{figure}[H]
    \centering
    \includegraphics[width=\textwidth]{docs/ps-design/design-document/images/datamodel/Recipe.png}
    \caption{Recipe}
    \label{fig:recipe_data}
\end{figure}

\noindent
Arrows going to recipe data (\ref{fig:recipe_data}):

\noindent
\begin{tabular}{|p{0.3\textwidth}|p{0.3\textwidth}|p{0.3\textwidth}|}
\hline
- & & \\
\hline
\end{tabular}

\vspace{0.3cm}
\noindent
Arrows going from recipe  (\ref{fig:recipe_data}):

\noindent
\begin{tabular}{|p{0.3\textwidth}|p{0.3\textwidth}|p{0.3\textwidth}|}
\hline
Menu (\ref{fig:menu_data}) & Version data (\ref{fig:version_data}) & \\
\hline
\end{tabular}

\begin{figure}[H]
    \centering
    \includegraphics[width=\textwidth]{docs/ps-design/design-document/images/datamodel/Job-Data.png}
    \caption{Job}
    \label{fig:job_data}
\end{figure}

\noindent
Arrows going to job  (\ref{fig:job_data}):

\noindent
\begin{tabular}{|p{0.3\textwidth}|p{0.3\textwidth}|p{0.3\textwidth}|}
\hline
Employee  (\ref{fig:employee_data}) & & \\
\hline
\end{tabular}

\vspace{0.3cm}
\noindent
Arrows going from job  (\ref{fig:job_data}):

\noindent
\begin{tabular}{|p{0.3\textwidth}|p{0.3\textwidth}|p{0.3\textwidth}|}
\hline
Application data (\ref{fig:application_data}) & Business data (\ref{fig:business_data}) & \\
\hline
\end{tabular}

\begin{figure}[H]
    \centering
    \includegraphics[width=\textwidth]{docs/ps-design/design-document/images/datamodel/Employee-Tasks.png}
    \caption{Employee tasks}
    \label{fig:employee_tasks}
\end{figure}

\noindent
Arrows going to employee tasks (\ref{fig:employee_tasks}):

\noindent
\begin{tabular}{|p{0.3\textwidth}|p{0.3\textwidth}|p{0.3\textwidth}|}
\hline
- & & \\
\hline
\end{tabular}

\vspace{0.3cm}
\noindent
Arrows going from employee tasks (\ref{fig:employee_tasks}):

\noindent
\begin{tabular}{|p{0.3\textwidth}|p{0.3\textwidth}|p{0.3\textwidth}|}
\hline
Location (\ref{fig:location_data}) & Employee data (\ref{fig:employee_data}) & \\
\hline
\end{tabular}

\begin{figure}[H]
    \centering
    \includegraphics[width=0.7\textwidth]{docs/ps-design/design-document/images/datamodel/Employee-Feedback.png}
    \caption{Employee feedback}
    \label{fig:employee_feedback}
\end{figure}

\noindent
Arrows going to employee feedback (\ref{fig:employee_feedback}):

\noindent
\begin{tabular}{|p{0.3\textwidth}|p{0.3\textwidth}|p{0.3\textwidth}|}
\hline
- & & \\
\hline
\end{tabular}

\vspace{0.3cm}
\noindent
Arrows going from employee feedback (\ref{fig:employee_feedback}):

\noindent
\begin{tabular}{|p{0.3\textwidth}|p{0.3\textwidth}|p{0.3\textwidth}|}
\hline
Employee data (\ref{fig:employee_data}) & & \\
\hline
\end{tabular}

\section{APIs}

The API contracts are presented as a single \texttt{openapi.yaml} file. To visualise the contracts, use \url{https://editor.swagger.io/} or any other visual API editor.

For authentication, we chose a email-and-password authentication with short-lived access JWTs and longer-lived refresh tokens stored in HttpOnly cookies.
 This lets users stay logged in for long periods (like employees on shift without need to relogin), and allows token revocation if needed. We chose this over a single JWT (which would force frequent logins), traditional server sessions (which need extra server storage), or Keycloak (which adds infrastructure complexity) because it’s simple, secure enough, and works well for our needs right now.


\end{document}