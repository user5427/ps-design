%%%%%
%%%%%  Use LUALATEX, not LATEX.
%%%%%
%%%%
\documentclass[]{VUMIFTemplateClass}

\usepackage{indentfirst}
\usepackage{amsmath, amsthm, amssymb, amsfonts}
\usepackage{mathtools}
\usepackage{physics}
\usepackage{graphicx}
\usepackage{verbatim}
\usepackage[hidelinks]{hyperref}
\usepackage{color,algorithm,algorithmic}
\usepackage[nottoc]{tocbibind}
\usepackage{tocloft}

\usepackage{titlesec}
\newcommand{\sectionbreak}{\clearpage}

\makeatletter
\renewcommand{\fnum@algorithm}{\thealgorithm}
\makeatother
\renewcommand\thealgorithm{\arabic{algorithm} algorithm}

\usepackage{biblatex}
\bibliography{bibliografija}
%% to change the numbering (numeric or alphabetic) of bibliographic sources, make the change in VUMIFTemplateClass.cls, line 139

% Author's MACROS
\newcommand{\EE}{\mathbb{E}\,} % Mean
\newcommand{\ee}{{\mathrm e}}  % nice exponent
\newcommand{\RR}{\mathbb{R}}


\studyprogramme{Software Engineering} %Write your study programme (example – Software engineering, Financial and Actuarial Mathematics, etc.)
\worktype{PS Software Design} % Bachelor's thesis or Master's thesis
\worktitle{PS Sofware Design documentation}
% \secondworktitle{Work Title in Lithuanian}
\workauthor{Tadas Riksas, Darius Spruogis, Gustas Mickus Julius Jauga}

%There may be more than one author, in which case each author is written from a new line which is added in Titlepage.tex or LongerTitlePage.tex
%\secondauthor{Name Surname} %If present, otherwise delete

\supervisor{Vasilij Savin}
% \reviewer{pedagogical/scientific title Name Surname} %If present, otherwise delete
% \scientificadvisor{pedagogical/scientific title Name Surname} %If present, otherwise delete

\begin{document}
\selectlanguage{english}

\onehalfspacing
\begin{titlepage}
\vskip 20pt
\begin{center}
\includegraphics[scale=0.55]{images/MIF.png}
\end{center}

\makeatletter

\vskip 20pt
\centerline{\bf \large \textbf{VILNIUS UNIVERSITY}}
\vskip 10pt
\centerline{\large \textbf{FACULTY OF MATHEMATICS AND INFORMATICS}}
\vskip 10pt
\centerline{\large \textbf{\MakeUppercase{\@studyprogramme \space study programme}}}

\vskip 80pt
\centerline{\Large \@worktype}
\vskip 20pt
\begin{center}
    {\bf \LARGE \@worktitle}
\end{center}
\begin{center}
    {\bf \Large \@secondworktitle}
\end{center}
\vskip 80pt

\centering{\Large \@workauthor}
\@ifundefined{@secondauthor}{}
{
\vskip 10pt
\centering{\Large \@secondauthor}
}
\vskip 20pt

\centering{
    \begin{tabular}{rcp{.7\textwidth}}
        {\Large Supervisor} & {\Large :} & {\Large \@supervisor}\\[10pt]
        \@ifundefined{@scientificadvisor}{}
            {
                {\Large Scientific advisor} & {\Large :} & {\Large \@scientificadvisor}\\[10pt]
            }
        \@ifundefined{@reviewer}{}
            {
                {\Large Reviewer} & {\Large :} & {\Large \@reviewer}\\[10pt]
            }
    \end{tabular}}


\vskip 110pt

\centerline{\large \textbf{Vilnius}}
\centerline{\large \textbf{\the\year{}}}

\makeatother

\newpage
\end{titlepage}
%\newgeometry{top=2cm,bottom=2cm,right=2cm,left=3cm}
\setcounter{page}{2}


%% Acknowledgements Section
\sectionnonumnocontent{Acknowledgements}
The author is thankful the Information Technology Research Center, Faculty of Mathematics and Informatics, Vilnius University, for providing the High-Performance Computing (HPC) resources for this research.
%%
%%
%%      If you have used IT resources (CPU-h, GPU-h, other IT resources) provided by MIF for your thesis research, please leave the acknowledgement; if you have not, you can delete it.
%%
%%

You can also add here acknowledgements for various other things, such as your supervisor, university, company, etc.

%%Summary
% \sectionnonum{Team agreement}

\section*{Team Information}
\textbf{Team name:} Fantastic 4 \\[4pt]
\textbf{Team members:} Tadas, Darius, Gustas, Julius \\[4pt]
\textbf{Team lead:} Tadas \\[2pt]
\textit{Reason: Showed most motivation} \\[8pt]

\section*{Team Lead Responsibilities}
\begin{itemize}
    \item Plan the meetings and consultations with lecturer
    \item Communicate with lecturer
    \item Create a timeline for the tasks that need to be completed
    \item Resolve conflicts
    \item Ensure everyone is putting equal amount of effort
    \item Ensure requirements are met
\end{itemize}

\subsection*{Team Ambition}
Target grade: \textbf{9--10}

\subsection*{Decision-Making Approach}
Everyone is free to express their opinions, and the final decision is made by voting.

\section*{Team Rules}

\subsection*{Communication}
\begin{itemize}
    \item Use one main channel on Discord
    \item Reply to messages promptly
    \item Be respectful and professional
\end{itemize}

\subsection*{Meetings}
\begin{itemize}
    \item Attend all meetings if possible
    \item Give notice if not available
    \item Be prepared to update your teammates
\end{itemize}

\subsection*{Code Standards}
\begin{itemize}
    \item Use ESLint for linting
    \item Use Prettier for automatic code formatting
    \item Perform pull request reviews
    \item Follow additional agreed conventions
\end{itemize}

\subsection*{Respect \& Accountability}
\begin{itemize}
    \item Value everyone's ideas and contributions
    \item Share workload fairly
    \item Stay constructive
\end{itemize}

\subsection*{Penalties}
For missing deadlines, meetings, or work:
\begin{itemize}
    \item Snack tax or beer tax
    \item Extra documentation duty
    \item Bug fix duty
\end{itemize}

\subsection*{Key principles governing work distribution}
\begin{itemize}
    \item Fairness and balance: Workload should be distributed so no one feels overburdened; try to balance difficulty, effort, and time commitment.
    \item Assign tasks according to strengths and preferences of team members.
    \item Be clear with responsibilities: Each task should have a designated owner and defined expectations.
    \item Track work using GitHub Projects; everyone should see what others are working on and provide regular progress updates.
    \item Avoid tasks that block other tasks; strive for parallel work whenever possible.
    \item Members are responsible for their assigned work, but must communicate and seek help if falling behind.
    \item Be willing to help each other; documentation and communication should allow someone else to pick up a task in progress.
\end{itemize}

\subsection*{Capacity Constraints}
\begin{itemize}
    \item Each member reports weekly availability so tasks can be planned realistically.
    \item Hold a weekly sync on [FILL IN], plus short check-ins as needed.
    \item Track capacity: record planned hours versus actual hours spent.
    \item If a member cannot meet their capacity, they must inform the team promptly.
\end{itemize}


\subsection*{Tracking Progress}
\begin{itemize}
    \item Use GitHub Projects to manage tasks and milestones
    \item Hold weekly meetings(Monday evenings) to review progress and plan next week.
\end{itemize}

\printbibliography[title = {References and sources}]


% \appendix
% \renewcommand{\thesection}{Appendix \arabic{section}. }

% \section{\phantom{Appendix} Examples of citations}
% In the document \textit{bibliography.bib}, you need to add all the cited sources and after using the function \textit{\{\textbackslash cite\{name of the cited object\}\}} the corresponding source will be added to the list of literature sources.


% \textit{bibliography.bib} provides examples of some of the most commonly cited types of sources:
% \begin{itemize}
%     \item web pages (\textit{@online}) \cite{PvzInternetinisPuslapis},
%     \item datasets (\textit{@dataset}) \cite{dataset}
%     \item articles (\textit{@article}) \cite{PvzStraipsnLt, PvzStraipsnEn}, 
%     \item articles from conferences (\textit{@inproceedings}) \cite{PvzKonfLt, PvzKonfEn}, 
%     \item books (\textit{@book}) \cite{PvzKnygLt, PvzKnygEn}, 
%     \item theses (\textit{@thesis or mastersthesis/phdthesis} \cite{PvzMagistrLt, PvzPhdEn})
%     \item electronic publications (\textit{@misc}) \cite{PvzElPubLt, PvzElPubEn}
% \end{itemize}

% Examples are also provided for ChatGPT citation, both in general \cite{chatgpt_bendrai} and for a specific conversation \cite{chatgpt_pokalbis}.

\end{document}
